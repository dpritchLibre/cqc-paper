
\section{Discussion}
\label{sec:discussion}

In this paper, We propose a family of classification rules based on the marginal
quantiles of the class features.  The univariate quantile classifier is equal to
the Bayes rule for the optimal choice of quantile level and under some
assumptions.  Motivated by this, we consider univariate quantile classifiers as
a starting point for constructing a multivariate classifier.  The multivariate
classifiers considered in this paper are constructed in a two step process.
First, the marginal quantiles of the data are estimated and an estimate of the
optimal quantile level for each component is calculated.  Secondly, a rule for
combining the information provided by an observation's quantile distances to the
marginal within-class quantile is constructed through a linear combination
obtained using penalized linear regression.  We observe competitive performance
of composite quantile-based classifiers in simulation examples and a spam email
classification application.  The composite quantile-based classifiers decision
rule is computationally efficient to train and the decision rule boundary has a
simple piecewise-linear form that is well-suited for high-dimensional settings.




%%% Local Variables:
%%% mode: latex
%%% TeX-master: "cqc_paper"
%%% End:
