
\documentclass{article}

\usepackage[fleqn]{amsmath}
\usepackage{amssymb}
\usepackage{amsthm}
\usepackage[margin=2.54cm]{geometry}
\usepackage{graphicx}
\usepackage{enumitem}
\usepackage{titlesec}                       % \titlespacing
\usepackage{bbm}                            % mathbbm (for \ind)
\usepackage{booktabs}
\usepackage{bm}
\usepackage{authblk}                        % author, affil
\usepackage{adjustbox}
\usepackage[ruled,linesnumbered]{algorithm2e}
\usepackage{mathtools}

% natbib bibliography style settings
\bibliographystyle{amsplain}


% create custom commands
\renewcommand{\arraystretch}{0.9}
\renewcommand{\vec}{\bm}
\newcommand{\ev}{\mathbb{E}\,}
\newcommand{\prob}{\mathbb{P}\,}
\newcommand{\bigO}{\mathcal{O}}
\DeclareMathOperator*{\argmin}{arg\,min}
\DeclareMathOperator*{\argmax}{arg\,max}
\DeclareMathOperator*{\minimize}{minimize}
\DeclareMathOperator{\sign}{sign}
\newcommand{\ind}{\mathbbm{1}\,}
\newcommand{\normal}{\text{N}}
\newcommand{\convp}{\stackrel{p}{\longrightarrow}}
\newcommand{\limtheta}{\lim_{\vec{\theta} \rightarrow \vec{\theta}^{*}}}
% \newcommand{\colsp}{\extracolsep{5mm}}
% \newcommand{\bigUp}[1]{\text{\kern0.5em\smash{\raisebox{0ex}{\huge #1}}}}
% \newcommand{\bigDown}[1]{\text{\kern-0.5em\smash{\raisebox{-1.5ex}{\huge #1}}}}
\def\r/{\textsf{R}}
\def\matlab/{\textsf{MATLAB}}

\newcommand{\bn}{\fontseries{b}\selectfont} % Bold Number, for highlighting numbers in tables

% Sets the document line spacing; 1.6 is double spacing.  See
% https://www.sharelatex.com/learn/Paragraph_formatting#Line_spacing
\linespread{1.6}


% % allow align environment to break over pages
\allowdisplaybreaks[2]


% Spacing before/after sections and subsections
\titlespacing*{\section}
{0pt}{5.5ex plus 1ex minus .2ex}{2.3ex plus .2ex}
\titlespacing*{\subsection}
{0pt}{7.5ex plus 1ex minus .2ex}{2.3ex plus .2ex}
\titlespacing*{\subsubsection}
{0pt}{5.5ex plus 1ex minus .2ex}{2.3ex plus .2ex}


% see http://www.maths.tcd.ie/~dwilkins/LaTeXPrimer/Theorems.html for these
% environments
\newtheorem{theorem}{Theorem}
\newtheorem{lemma}[theorem]{Lemma}
\newtheorem{corollary}[theorem]{Corollary}
% \newenvironment{proof}[1][Proof]{\begin{trivlist}
% \item[\hskip \labelsep {\bfseries #1}]}{\end{trivlist}}




% document start ---------------------------------------------------------------

\begin{document}

% title and author information
\begin{center}

  {\LARGE Composite Quantile-Based Classifiers} \vspace{5mm}

  {\large David A. Pritchard$^1$ and Yufeng Liu$^{2,3,1}$} \vspace{5mm}

  {\large
    $^1$Department of Biostatistics,
    $^2$Department of Statistics and Operations Research,
    $^3$Department of Genetics,
    University of North Carolina, Chapel Hill, NC 27599, USA
  }

\end{center}

\section{Theoretical proofs}
\label{sec:theoretical-proofs}

\begin{lemma}
  \label{lem:empirical-quantlev}
  Let $x_1, \dots, x_m$ be points on $\mathbb{R}$.  Then a solution for $q$ to
  the expression
  \begin{equation}
    \label{eq:empirical-quantile}
    \argmin_q \left\{
      \theta \sum_{ x_{i} > q } |x_{i} - q| ~+~
      (1 - \theta) \sum_{ x_{i} \leq q } |x_{i} - q|
    \right\}.
  \end{equation}
  providing the $\theta$-th empirical quantile for
  $x_1, \dots, x_m$ is given by $\lceil m \theta \rceil$-th largest value of the
  $x_i$.
\end{lemma}

\begin{proof}
  
We can write the minimization problem
\begin{equation*}
  \min_q \left\{
    \theta \sum_{ x_{i} > q } |x_{i} - q| ~+~
    (1 - \theta) \sum_{ x_{i} \leq q } |x_{i} - q|
  \right\}
\end{equation*}
equivalently as
% \begin{align*}
%   &\min_{q^{+}, q^{-}, \vec{u}, \vec{v}} \left\{
%     \theta \sum_{i=1}^m u_i ~+~
%     (1 - \theta) \sum_{i=1}^m v_i
%   \right\} \\[2ex]
%   & \text{subject to} \hspace{3mm}
%   x_i - (q^{+} - q^{-}) = u_i - v_i, \hspace{5mm} i = 1, \dots, m \\[2ex]
%   & q^{+} \geq 0,~ q^{-} \geq 0,~ \vec{u} \geq \vec{0},~ \vec{v} \geq 0
% \end{align*}
\begin{equation*}
  \arraycolsep=5mm
  \begin{array}{ll}
    \displaystyle
    \minimize_{q^{+}, q^{-}, \vec{u}, \vec{v}}
    & \theta \sum_{i=1}^m u_i ~+~
      (1 - \theta) \sum_{i=1}^m v_i \\[2ex]
    \text{subject to}
    & x_i - (q^{+} - q^{-}) = u_i - v_i, \hspace{5mm} i = 1, \dots, m \\[2ex]
    & q^{+} \geq 0,~ q^{-} \geq 0,~ \vec{u} \geq \vec{0},~ \vec{v} \geq \vec{0} \\
  \end{array}
\end{equation*}
which is seen to be a linear programming problem in standard form.  By rewriting
the equality condition as $q^{+} - q^{-} + u_i - v_i = x_i$ for $i=1, \dots, m$,
we can express the equality condition in matrix form as
\begin{equation*}
  \begin{bmatrix}
    1      & -1     & 1 &        &   & -1 &        &    \\
    \vdots & \vdots &   & \ddots &   &    & \ddots &    \\
    1      & -1     &   &        & 1 &    &        & -1 \\
  \end{bmatrix}
  % \begin{bmatrix}
  %   q^{+} \\ q^{-} \\ u_{1} \\ \vdots \\ u_m \\ v_1 \\ \vdots \\ v_m \\
  % \end{bmatrix}
  \begin{bmatrix}
    q^{+} \\ q^{-} \\ \vec{u} \\ \vec{v} \\
  \end{bmatrix}
  =
  \begin{bmatrix}
    x_1 \\ \vdots \\ x_m \\
  \end{bmatrix}.
\end{equation*}
% \begin{equation*}
%   \begin{bmatrix}
%     1      & -1     & 1         &        &  \bigDown{0} & -1        &        & \bigDown{0} \\
%     \vdots & \vdots &           & \ddots &              &           & \ddots &             \\
%     1      & -1     & \bigUp{0} &        & 1            & \bigUp{0} &        & -1          \\
%   \end{bmatrix}
% \end{equation*}
% \begin{equation*}
%   \begin{bmatrix}
%     1      & -1     & &         &  & &          & \\
%     \vdots & \vdots & & \vec{I} &  & & -\vec{I} & \\
%     1      & -1     & &         &  & &          & \\
%   \end{bmatrix}
% \end{equation*}
% \begin{equation*}
%   \begin{bmatrix}
%     1      & -1     & 1      & 0      & \dots  & 0       & -1     &        &    \\
%     \vdots & \vdots & 0      & \ddots & \ddots & \vdots  &  0     & \ddots & \\
%     \vdots & \vdots & \vdots & \ddots & \ddots & 0       & \vdots & \ddots & \ddots &    \\
%     1      & -1     & 0      & \dots  & 0      & 1       &  0     & \dots  &  0     & -1 \\
%   \end{bmatrix}
% \end{equation*}
Recall that a solution for $(q^{+}, q^{-}, \vec{u}, \vec{v})$ is a basic
solution if and only if there exist indices $B(1), \dots, B(m)$ such that both
(i) the columns of the coefficient matrix with column indices subset by
$B(1), \dots, B(m)$ are linearly independent, and (ii) if an element of
$(q^{+}, q^{-}, \vec{u}, \vec{v})$ does not correspond to one of
$B(1), \dots, B(m)$ then the element must have a value of 0.

We can see that in order to have independent columns from the coefficient
matrix, then for each $i$, no more than one of the columns corresponding to
either $u_i$ or $v_i$ can have an index in $B(1), \dots, B(m)$, and additionally
no more than one of the columns corresponding to either $q^{+}$ or $q^{-}$ can
have an index in $B(1), \dots, B(m)$.  Furthermore, we note that if we have one
column corresponding to $q^{+}$ or $q^{-}$, and one column corresponding to
either $u_i$ or $v_i$ for each $i$, then we have $m + 1$ columns, which is still
one column too many.  Thus we see that there must be either (i) exactly one
column corresponding to either $u_i$ or $v_i$ for each $i$ and no columns
corresponding to either $q^{+}$ or $q^{-}$, or (ii) exactly one column
corresponding to either $u_i$ or $v_i$ for each $i$ less one and exactly one
column corresponding to either $q^{+}$ or $q^{-}$.

% Next we notice that if we include a column corresponding to $q^{+}$ or $q^{-}$,
% then the solution for the corresponding coefficient is either the value of $x_i$
% or $-x_i$, and where the index $i$ corresponds to the only $i$ without a column
% corresponding to either $u_i$ or $v_i$.  Then each

Furthermore, we can infer that if a column corresponding to $q^{+}$ or $q^{-}$
has an index in $B(1), \dots, B(m)$ and the solution is feasible (i.e. $q^{+}$
and $q^{-}$ are both nonnegative), then $q^{+} = (x_i)_{+}$ and
$q^{-} = (-x_i)_{+}$, where the index $i$ corresponds to the only $i$ without a
column corresponding to either $u_i$ or $v_i$, and $(z)_{+} = \max(0, z)$.  Let
$q = q^{+} - q^{-}$, then it follows that a feasible solution for any $j \ne i$
has $u_i = (x_i - q)_{+}$ and $v_i = (q - x_i)_{+}$.  This leads to a set of
basic feasible solutions given by $q \in \{x_1, \dots, x_m\}$.  One last basic
feasible solution is given for $q = 0$ with $u_i = (x_i)_{+}$ and
$v_i = (-x_i)_{+}$ for all $i$.

Next we aim to find the minimizing basic feasible solution.  Suppose that
$\lceil \theta m \rceil = k$ and that $\ell < k$, and let $q = x_k$ and
$q^{\prime} = x_{\ell}$.  Then comparing the objective function evaluated at $q$
and $q^{\prime}$, we have
\begin{align*} 
  \left\{ \theta \sum_{i=\ell+1}^m \right.
  & (x_i - q^{\prime}) +
    \left. (1 - \theta) \sum_{i=1}^{\ell} (q^{\prime} - x_i) \right\} -
    \left\{ \theta \sum_{i=k+1}^m (x_i - q) +
    (1 - \theta) \sum_{i=1}^k (q - x_i) \right\} \\[1ex]
  &= (1 - \theta) \sum_{i=1}^{\ell} \Big\{ (q^{\prime} - x_i) - (q - x_i) \Big\} \\
  &\hspace{8mm}
    + \theta \sum_{i=\ell+1}^k (x_i - q^{\prime})
    - (1 - \theta) \sum_{i=\ell+1}^k (q - x_i) \\
  &\hspace{8mm}
    + \theta \sum_{i=k+1}^m \Big\{ (x_i - q^{\prime})
    - (x_i - q) \Big\} \\[1ex]
  % &= (1 - \theta) \sum_{i=1}^{\ell} \Big\{ (q^{\prime} - x_i) - (q - x_i) \Big\} \\
  % &\hspace{8mm}
  %   + \theta \sum_{i=\ell+1}^k (x_i - q^{\prime})
  %   - (1 - \theta) \sum_{i=\ell+1}^k (q - x_i) \\
  % &\hspace{8mm}
  %   + (1 - \theta) \sum_{i=\ell+1}^k (q^{\prime} - x_i)
  %   - (1 - \theta) \sum_{i=\ell+1}^k (q^{\prime} - x_i) \\
  % &\hspace{8mm}
  %   + \theta \sum_{i=k+1}^m \Big\{ (x_i - q^{\prime})
  %   - (x_i - q) \Big\} \\[1ex]
  &= (1 - \theta) \sum_{i=1}^{\ell} \Big\{ (q^{\prime} - x_i) - (q - x_i) \Big\} \\
  &\hspace{8mm}
    + \theta \sum_{i=\ell+1}^k (x_i - q^{\prime})
    - (1 - \theta) \sum_{i=\ell+1}^k (q - x_i)
    \pm (1 - \theta) \sum_{i=\ell+1}^k (q^{\prime} - x_i) \\
  &\hspace{8mm}
    + \theta \sum_{i=k+1}^m \Big\{ (x_i - q^{\prime})
    - (x_i - q) \Big\} \\[1ex]
  &= -(1 - \theta) \sum_{i=1}^{\ell}\, (q - q^{\prime}) \\
  &\hspace{8mm}
    + \theta \sum_{i=\ell+1}^k (x_i - q^{\prime})
    - (1 - \theta) \sum_{i=\ell+1}^k (q - q^{\prime})
    - (1 - \theta) \sum_{i=\ell+1}^k (q^{\prime} - x_i) \\
  &\hspace{8mm}
    + \theta \sum_{i=k+1}^m (q - q^{\prime}) \\[1ex]
  &= -(1 - \theta) \sum_{i=1}^k\, (q - q^{\prime}) \\
  &\hspace{8mm}
    + \theta \sum_{i=\ell+1}^k (x_i - q^{\prime})
    - (1 - \theta) \sum_{i=\ell+1}^k (q^{\prime} - x_i) \\
  &\hspace{8mm}
    + \theta \sum_{i=k+1}^m (q - q^{\prime}) \\[1ex]
  &= -(1 - \theta) \sum_{i=1}^k\, (q - q^{\prime}) \\
  &\hspace{8mm}
    + \sum_{i=\ell+1}^k (x_i - q^{\prime}) \\
  &\hspace{8mm}
    + \theta \sum_{i=k+1}^m (q - q^{\prime}) \\[1ex]
  & = - (1 - \theta)\, k\, (q - q^{\prime})
    + \sum_{i=\ell+1}^k (x_i - q^{\prime})
    + \theta (m - k) (q - q^{\prime}) \\
  &= \sum_{i=\ell+1}^k (x_i - q^{\prime}) - (k - \theta m) (q - q^{\prime}) \\
  &\geq (x_k - q^{\prime}) - (q - q^{\prime}) \\
  &= x_k - q \\
  &= 0
\end{align*}
The inequality is due to the fact that $x_i \geq q^{\prime}$ for
$i \geq \ell + 1$, and also the fact that $k - \theta m < 1$.  Suppose now that
$\lceil \theta m \rceil = k$ and that $\ell > k$, and let $q = x_k$ and
$q^{\prime} = x_{\ell}$.  Then comparing the objective function evaluated at $q$
and $q^{\prime}$, we have
\begin{align*} 
  \left\{ \theta \sum_{i=\ell+1}^m \right.
  & (x_i - q^{\prime}) +
    \left. (1 - \theta) \sum_{i=1}^{\ell} (q^{\prime} - x_i) \right\} -
    \left\{ \theta \sum_{i=k+1}^m (x_i - q) +
    (1 - \theta) \sum_{i=1}^k (q - x_i) \right\} \\[1ex]
  &= (1 - \theta) \sum_{i=1}^k \Big\{ (q^{\prime} - x_i) - (q - x_i) \Big\} \\
  &\hspace{8mm}
    + \theta \sum_{i=k+1}^{\ell} (x_i - q^{\prime})
    - (1 - \theta) \sum_{i=k+1}^{\ell} (q - x_i) \\
  &\hspace{8mm}
    + \theta \sum_{i=\ell+1}^m \Big\{ (x_i - q^{\prime})
    - (x_i - q) \Big\} \\[1ex]
  &= (1 - \theta) \sum_{i=1}^k\, (q^{\prime} - q) \\
  &\hspace{8mm}
    - \theta \sum_{i=k+1}^{\ell} (q^{\prime} - q)
    + \sum_{i=k+1}^{\ell} (x_i - q) \\
  &\hspace{8mm}
    - \theta \sum_{i=l+1}^m (q^{\prime} - q) \\[1ex]
  % &= (1 - \theta) \sum_{i=1}^k\, (q^{\prime} - q) \\
  % &\hspace{8mm}
  %   + \sum_{i=k+1}^{\ell} (x_i - q) \\
  % &\hspace{8mm}
  %   - \theta \sum_{i=k+1}^m (q^{\prime} - q) \\[1ex]
  &= (1 - \theta) \sum_{i=1}^k\, (q^{\prime} - q)
    + \sum_{i=k+1}^{\ell} (x_i - q)
    - \theta \sum_{i=k+1}^m (q^{\prime} - q) \\
  &= (1 - \theta)\, k\, (q^{\prime} - q)
    + \sum_{i=k+1}^{\ell} (x_i - q)
    - \theta\, (m - k) (q^{\prime} - q) \\
  &= (k - \theta m) (q^{\prime} - q) + \sum_{i=k+1}^{\ell} (x_i - q) \\
  &\geq 0
\end{align*}
There is one basic feasible solution remaining to check, that where $q = 0$.  To
show that this is not the optimal solution except in the case that
$x_{\lceil \theta m \rceil} = 0$, we make the following argument.  Note that if
we add some nonzero value $\tau$ to each $x_i$ and if $q^{*}$ is an optimal
choice for the original problem, then $q^{*} + \tau$ is an optimal choice for
the new problem.  Choose $-\tau$ to be one of the $x_i$ for some $i$: then 0 is
one of the values in the transformed data, which was shown in the previous
results to be no better than the $\lceil \theta m \rceil$-th largest value of
the transformed data, so it follows that the $\lceil \theta m \rceil$-th largest
value is optimal.  Since $\tau$ cannot be better than optimal for the transormed
data, then by the law of the contrapositive 0 cannot be better than optimal for
the original data.




%%% Local Variables:
%%% mode: latex
%%% TeX-master: "cqc_paper"
%%% End:

\end{proof}




\section{Calculating the empirically optimal quantile classifier}
\label{sec:empirically-optimal-algo}

In order to calculate the empirical classification rate as a function of
quantile level, the first step is to obtain the set of valid decision rule
boundaries.  In other words, let
$\phi(\theta;\, \text{data}){:}~ (0, 1) \mapsto \mathbb{R}$ be the function that
maps a quantile level for a given data set to the quantile classifier decision
rule boundary; then we wish to obtain the set
$\big\{ x \in \mathbb{R}{:} \hspace{3mm} \theta \in (0, 1),~ \phi(\theta;\,
\text{data}) = x \big\}$.  The key observation that facilitates the calculation
of this set is that the domain $(0, 1)$ can be partitioned into smaller
intervals for which range of the decision rule boundary function $\phi$ can be
easily calculated.  In more detail, consider an interval
$(\theta_{\scriptscriptstyle\text{low}},\,
\theta_{\scriptscriptstyle\text{high}}]$ such that $\lceil \theta n_0 \rceil$
and $\lceil \theta n_1 \rceil$ are each constant for all $\theta$ in the
interval.  As a consequence of the form of the quantile estimator, it follows
that there are values $a$ and $b$ such that $F_{(0)}^{-1}(\theta) = a$ and
$F_{(1)}^{-1}(\theta) = b$ for all $\theta$ in the interval.  Thus by the result
on the form of the decision rule boundary, we have
$\phi(\theta;\, \text{data}) = \theta a + (1 - \theta) b$ for any $\theta$ in
the interval, and it follows that
$\Big\{ x \in \mathbb{R}{:} \hspace{3mm} \theta \in
(\theta_{\scriptscriptstyle\text{low}},\,
\theta_{\scriptscriptstyle\text{high}}],~ \phi(\theta;\, \text{data}) = x \Big\}
= \Big[ \theta_{\scriptscriptstyle\text{high}}\, a + (1 -
\theta_{\scriptscriptstyle\text{high}})\, b, \hspace{3mm}
\theta_{\scriptscriptstyle\text{low}}\, a + (1 -
\theta_{\scriptscriptstyle\text{low}})\, b\, \Big)$.

Next we consider how to partition $(0, 1)$ into such sets.  For notational
convenience we actually consider $(0, 1]$: even though the 1-th quantile doesn't
make sense theoretically, its estimate is well-defined.  Now
$\lceil \theta n_0 \rceil$ is a step function with the endpoints for each step
taking place at $1, \dots, n_0$ and which occur for values of $\theta$ at
$1 / n_0, 2 / n_0, \dots, n_0 / n_0$.  Similarly, $\lceil \theta n_1 \rceil$ is
a step function with the endpoint for each step taking place at $1, \dots, n_1$
and which occurs for values of $\theta$ at $1 / n_1, 2 / n_1, \dots, n_1 / n_1$.
Thus any interval that doesn't include one of those values of $\theta$ except as
its upper bound is constant for each of these functions.  So we can partition
$(0, 1]$ by letting each interval have an open lower bound be one of the step
function change locations for one of the classes, and having the upper bound be
the next smallest step function change location for either of the classes (and
we also need an interval from 0 to the smallest step function change location).

Consider then one such interval, say
$[x_{\scriptscriptstyle\text{low}}, x_{\scriptscriptstyle\text{high}})$, such
that for each
$x \in [x_{\scriptscriptstyle\text{low}}, x_{\scriptscriptstyle\text{high}})$
there exists a $\theta \in (0, 1]$ with $\phi(\theta;\, \text{data}) = x$.  At
this point we would like to calculate the classification rate for all $x$ in the
interval: we can achieve this by conceptually ``sliding'' the decision rule
boundary across the interval.  Since the classification rate doesn't change as
we slide the decision rule boundary in-between observations, we can find the
values in the data that belong to
$[x_{\scriptscriptstyle\text{low}}, x_{\scriptscriptstyle\text{high}})$ and
partition the interval into sub-intervals that each have a constant
classification rate.  So for example, if $x_k, \dots, x_{k+s}$ are the only
observations in the data that belong to
$[x_{\scriptscriptstyle\text{low}}, x_{\scriptscriptstyle\text{high}})$, then
$[x_{\scriptscriptstyle\text{low}}, x_k),\, [x_k, x_{k+1}),\, \dots,\,
[x_{k+s-1}, x_{k+s}),\, [x_{k+s},\, x_{\scriptscriptstyle\text{high}})$ may be
such a partitioning.

The reason we say that this `may' be a partitioning is that whether the
endpoints of the sub-intervals are open or closed depends on which class we have
arbitrarily decided to classify an observation to in the event of ties.  Note
that when the quantile level $\theta$ corresponds to a decision rule boundary
with a value equal to a point in the data, say $x^{*}$, then the quantile
distance of that point to the populations' empirical $\theta$-th quantiles is
the same for either population, so we have to fall back on our classification
rule in the event of ties.  Whether the decision rule has the same
classification rate as an open interval with upper bound $x^{*}$ or as an open
interval with lower bound $x^{*}$ depends on which class has a smaller empirical
$\theta$-th quantile, and what our tiebreaker rule is.  If the tiebreaker goes
to the class with a larger empirical $\theta$-th quantile, then the
classification rate is the same as an open interval with with lower bound
$x^{*}$, and we get a partitioning of
$[x_{\scriptscriptstyle\text{low}}, x_{\scriptscriptstyle\text{high}})$ with the
form
$[x_{\scriptscriptstyle\text{low}}, x_k),\, [x_k, x_{k+1}),\, \dots,\,
[x_{k+s-1}, x_{k+s}),\, [x_{k+s},\, x_{\scriptscriptstyle\text{high}})$.  If
however, the tiebreaker goes to the class with a larger empirical $\theta$-th
quantile, then the classification rate is the same as an open interval with
upper bound $x^{*}$, and we get a partitioning of
$[x_{\scriptscriptstyle\text{low}}, x_{\scriptscriptstyle\text{high}})$ with the
form
$[x_{\scriptscriptstyle\text{low}}, x_k],\, (x_k, x_{k+1}],\, \dots,\,
(x_{k+s-1}, x_{k+s}],\, (x_{k+s},\, x_{\scriptscriptstyle\text{high}})$.

Finally, we note that we may wish to map back the sets from the decision rule
boundary space to their quantile levels space, for example if we've found a set
with a good empirical classification rate.  Recall that we started with a set of
quantile levels
$(\theta_{\scriptscriptstyle\text{low}},\,
\theta_{\scriptscriptstyle\text{high}}]$ such that $F_{0}^{-1}$ and $F_{1}^{-1}$
are each constant for all $\theta$ in the interval, and that each
$x \in [x_{\scriptscriptstyle\text{low}}, x_{\scriptscriptstyle\text{high}})$
corresponds to the decision rule boundary for a $\theta$ from that interval.
Let us write $F_{(0)}^{-1}(\theta) = a$ and $F_{(1)}^{-1}(\theta) = b$ for all
$\theta$ in the interval, and assume that $a$ is strictly less than $b$.  Then
for
$x \in [x_{\scriptscriptstyle\text{low}}, x_{\scriptscriptstyle\text{high}})$,
we observe the following relation:
\begin{equation}
  \label{eq:map-to-quantile}
  x = \theta\, a + (1 - \theta)\, b
  \hspace{5mm} \Longleftrightarrow \hspace{5mm}
  \theta = \frac{b - x}{b - a}.
\end{equation}
At this point we have all of the concepts that we need to construct an algorithm
to calculate an empirically optimal choice of quantile level.  The corresponding
algorithm is presented as Algorithm \ref{alg:empirically-optimal-classifier}.

\begin{algorithm}[p]
  \label{alg:empirically-optimal-classifier}
  \DontPrintSemicolon
  \BlankLine
  % \KwResult{Write here the result}
  \SetKwInOut{Input}{Data}
  % \SetKwInOut{Output}{Output}
  \Input{$v_1, \dots, v_{n_0}$ from $\Pi_0$ and $w_1, \dots, w_{n_1}$ in
    $\Pi_1$}
  % \Output{Write here the output}
  % \BlankLine
  
  sort observations $v_1, \dots, v_{n_0}$ and $w_1, \dots, w_{n_1}$.  Assume
  that in the case of ties, we classify to $\Pi_1$. \;
  
  sort $v_1, \dots, v_{n_0}, w_1, \dots, w_{n_1}$ and denote the unique values
  as $x_1, \dots, x_r$
  
  Produce a sorted set of quantile levels given by
  $ \Theta = \Big\{ 0 \Big\} \bigcup \left\{ \frac{k}{n_0}: 1 \leq k \leq n_0
  \right\} \bigcup \left\{ \frac{k}{n_1}: 1 \leq k \leq n_1 \right\} $

  \For{$i$ in 2 to $|\Theta|$}{
    
    \BlankLine
    
    calculate $\hat{F}_0^{-1}(\theta_i)$ and $\hat{F}_1^{-1}(\theta_i)$, and
    find
    \begin{equation*}
      a = \min \Big\{ \hat{F}_0^{-1}(\theta_i),~ \hat{F}_1^{-1}(\theta_i) \Big\}
      \hspace{5mm} \text{and} \hspace{5mm}
      b = \max \Big\{ \hat{F}_0^{-1}(\theta_i),~ \hat{F}_1^{-1}(\theta_i) \Big\}
    \end{equation*}

    calculate the interval
    \[
      G_i = \Big[
      \theta_i\, a + (1 - \theta_i)\, b,
      \hspace{3mm}
      \theta_{i-1}\, a + (1 - \theta_{i-1})\, b\,
      \Big) = \big[x_{\scriptscriptstyle\text{low}},\,
      x_{\scriptscriptstyle\text{high}}\big)
    \]
    % Let $x_{\scriptscriptstyle\text{low}}$ and
    % $x_{\scriptscriptstyle\text{high}}$ denote the lower and upper bounds of $G_i$,
    % respectively.  \;

    find the smallest $x_i \in [x_{\scriptscriptstyle\text{low}}, \infty)$.  If
    $x_i \geq x_{\scriptscriptstyle\text{high}}$ then calculate the
    classification rate for $G_i$.

    \BlankLine

    \While{$x_i \in G_i$}{

      calculate the classification rate for the interval with upper and lower
      bounds determined by $x_i$ and
      $\min\{x_{i+1}, x_{\scriptscriptstyle\text{high}}\}$, respectively (for
      whether bounds in interval are open or closed see below)

      \BlankLine

      \uIf{$\hat{F}_0^{-1}(\theta_i) < \hat{F}_1^{-1}(\theta_i)$}{
        % partition $G_i$ into the following intervals each with a constant
        % classification rate \vspace{-2mm}
        calculate classification rate for the interval with bounds $x_i$ and
        $x_{i+1}$ among the following:
        \begin{equation*}
          [x_{\scriptscriptstyle\text{low}}, x_k],\, (x_k, x_{k+1}],\, \dots,\,
          (x_{k+s-1}, x_{k+s}],\, (x_{k+s},\, x_{\scriptscriptstyle\text{high}})
        \end{equation*}
      }
      \Else{
        % partition $G_i$ into the following intervals each with a constant
        % classification rate \vspace{-2mm}
        calculate classification rate for the interval with bounds $x_i$ and
        $x_{i+1}$ among the following:
        \begin{equation*}
          [x_{\scriptscriptstyle\text{low}}, x_k),\, [x_k, x_{k+1}),\, \dots,\,
          [x_{k+s-1}, x_{k+s}),\, [x_{k+s},\, x_{\scriptscriptstyle\text{high}})
        \end{equation*}
      }

      map interval back to corresponding interval in the quantile levels space \;
      
      $x_i \leftarrow x_{i+1}$
      
    }

    % \uIf{$\hat{F}_V^{-1}(\theta_i) < \hat{F}_V^{-1}(\theta_i)$}{
    % partition $G_i$ into the following intervals each with a constant
    % classification rate \vspace{-2mm}
    % \begin{equation*}
    %   [x_{\scriptscriptstyle\text{low}}, x_k],\, (x_k, x_{k+1}],\, \dots,\,
    %   (x_{k+s-1}, x_{k+s}],\, (x_{k+s},\, x_{\scriptscriptstyle\text{high}})
    % \end{equation*}
    % }
    %   \uElse{
    %   partition $G_i$ into the following intervals each with a constant
    %   classification rate \vspace{-2mm}
    %   \begin{equation*}
    %     [x_{\scriptscriptstyle\text{low}}, x_k),\, [x_k, x_{k+1}),\, \dots,\,
    %     [x_{k+s-1}, x_{k+s}),\, [x_{k+s},\, x_{\scriptscriptstyle\text{high}})
    %   \end{equation*}
    % }  
    
  }
  \caption{Calculating the empirically optimal quantile classifier}
\end{algorithm}







\section{Extended simulation results}
\label{sec:simulation-results}

\subsection{Simulated data scenarios}
\label{sec:simulated-data-scenarios}

Consider the following framework for each of the scenarios.  Let
$\vec{Z} = (Z_1, \dots, Z_p) \sim N(\vec{0},\, \vec{\Sigma})$, and let
$\vec{V}_i = (V_{i1}, \dots, V_{ip})$ and $\vec{W}_i = (W_{i1}, \dots, W_{ip})$
be observations from two populations, $i = 1, \dots, n / 2$, and where the
$\vec{V}_i$ and $\vec{W}_i$ are all mutually independent.  There are two choices
of covariance matrices considered for $\vec{\Sigma}$.  In some scenarios we
consider independent data where the covariance matrix is the identity matrix,
and in other scenarios we consider the correlated data with an autoregressive
lag 1 covariance matrix (abbreviated as AR1).  The AR1 matrix is specified with
variance parameters identically 1 on the diagonal, and correlation parameter
0.8.  In the interest of continuity, the scenarios are designed to be similar to
the simulation settings presented in \cite{hennig2016}.

In the first and second scenarios, we consider two classes each with features
drawn from a multivariate Gaussian distribution and with one class given a
location shift.  That is to say that for each feature we consider
$V_{ij} \sim Z_j$ and $W_{ij} \sim Z_j + 0.35$.  In the first scenario we
consider uncorrelated data for the underlying Gaussian distribution, and in the
second scenario we consider autoregressive correlation.

In the third and fourth scenarios, we consider highly skewed data by
exponentiating the components of a Gaussian distribution.  That is to say that
for each feature we consider $V_{ij} \sim \exp(Z_j)$ and
$W_{ij} \sim \exp(Z_j) + 0.35$.  In the third scenario we consider uncorrelated
data for the underlying Gaussian distribution, and in the fourth scenario we
consider autoregressive correlation.

In the fifth and sixth scenarios we consider different distributions for each of
5 blocks of features.  In more detail, the data is sampled from a Gaussian
distribution and then $p$ features are evenly split into 5 blocks, with the
following transformations performed to the various blocks: (i) $V_{ij} \sim Z_j$
and $W_{ij} \sim Z_j + 0.2$, (ii) $V_{ij} \sim \exp(Z_j)$ and
$W_{ij} \sim \exp(Z_j) + 0.2$, (iii) $V_{ij} \sim \log |Z_j|$ and
$W_{ij} \sim \log |Z_j| + 0.1$, (iv) $V_{ij} \sim Z_j^2$ and
$W_{ij} \sim Z_j^2 + 0.2$, and (v) $V_{ij} \sim |Z_j|^{1/2}$ and
$W_{ij} \sim |Z_j|^{1/2} + 0.1$.  In the fifth scenario we consider uncorrelated
data for the underlying Gaussian distribution, and in the sixth scenario we
consider autoregressive correlation.

In the seventh and eight scenarios, we considered data with different
distributional shapes within each feature.  In the seventh scenario we
considered data with independent beta distributed features and in the eight
scenario we considered data with independent gamma distributed features.  The
beta distribution parameters were sampled as follows.  Two shape parameters were
sampled for each feature each from a $\mathit{unif}(0.1, 3)$ and
$\mathit{unif}(0.5, 3)$ distribution for each shape parameter, respectively.
Then for each class and feature, each parameter was transformed by taking the
absolute value of some additive Gaussian random noise.  So for example, suppose
that $\alpha_j$ and $\beta_j$ are the shape parameters drawn for the $j$-th
feature.  Then for each class the distributional parameters are each sampled
from $|\alpha_j + N(0, \sigma_j^2)|$ and $|\beta_j + N(0, \sigma_j^2)|$
distributions.  The parameters $\sigma_1, \dots, \sigma_{50}$ were given by a
fixed sequence each with values between 0.05 and 0.20.  Once the shape
parameters were sampled, the same parameters were used for every replicate and
every simulation study.  Finally, the gamma distribution parameters in the
eighth scenario were sampled in essentially the same manner.


% \subsubsection{Hybrid CQC and quantile-based classifiers}
% \label{sec:hybid-cqc}

% One disadvantage of the CQC approach is that the data is partitioned into two
% parts: one for selecting quantile levels and estimating quantiles, and the other
% for training the linear combination coefficients, so that each of these tasks is
% performed using only a subset of the data.  When data is abundant, there is
% typically not a great loss of efficiency when performing these tasks as compared
% to using all of the available data.  However, when data is limited, losing a
% portion of the data can result in a large loss of efficiency.  As a concrete
% example, suppose $n = 50$ and we split the data into two equal parts.  Then
% estimating the within-class quantiles is limited to 25 points so that the
% quantiles are estimated for at least one class with no more than 12 data points
% - a big difference to estimating the quantiles from 25 data points.  When data
% is limited as in this example, we find that CQC can suffer from instability in
% the choice of quantile levels.  As a result, we suggest using the quantile
% levels as selected by the quantile-based classifiers method and then applying
% the linear combination step on the transformed variables as in the usual CQC
% method.  This has the result in choosing quantile levels that are far more
% stable at the cost of some flexibility.  Our suggestion is that when data is
% relatively small, say less than 100, to perform cross-validation with the
% regular CQC and hybrid approach to select the better model.  It is noted in the
% discussion of the simulations whenever the hybrid approach is used.


\subsection{Classifier implementations and settings}
\label{sec:classifier-implementations}

We compared the misclassification rate from the composite quantile-based
classifiers model with that of nine other classification methods: quantile-based
classifiers \cite{hennig2016}, FANS \cite{fan2016}, penalized linear regression
($\ell_1$ penalty) \cite{park2007}, support vector machine (radial kernel)
\cite{cortes1995}, k-nearest neighbor \cite{cover1967}, naive Bayes
\cite{hastie2009}, nearest shrunken centroids \cite{tibshirani2002}, penalized
LDA \cite{witten2011}, and decision trees \cite{breiman1984}.  

The composite quantiles-based classifier is implemented using the \r/
programming language \cite{rlang}; the source code is available from the authors
upon request.  Quantile-based methods is implemented as the \r/ package
\texttt{quantileDA}.  There are several methods of quantifying distributional
skew provided by the package; we used the default Galton skewness measure.  FANS
is implemented as \matlab/ \cite{matlab} source code and is available upon
request by the authors of \cite{fan2016}.  Results from both the FANS method and
FANS2 method are shown since FANS2 is similar to the augmented version of
composite quantile-based classifiers.  Penalized linear regression is
implemented as the \r/ package \texttt{glmnet} and uses the $\ell_1$ penalty
with 10-fold cross validation to select the penalty parameter.  Support vector
machine is implemented in the \r/ package \texttt{e1071} through the C++ library
\texttt{libsvm}.  The tuning parameters for each simulation were selected by
using the function \texttt{tune.svm} over the kernel coefficient parameter from
among $\{0.001, 0.01, 0.1, 1, 2\}$, and constraints violation cost from among
$\{1, 2, 4, 8, 16\}$.  k-nearest neighbors is implemented in the \r/ package
\texttt{class} and uses leave-one-out cross validation to choose the number of
neighbors considered from among $\{1, \dots, 9\}$.  Naive Bayes is implemented
in the \r/ package \texttt{e1071}.  Nearest shrunken centroids is implemented as
the \r/ package \texttt{pamr} with 10-fold cross validation used to select the
threshold parameter.  The version of penalized linear discriminant analysis
compared in the simulation studies is that proposed in \cite{witten2011}, and is
implemented as the \r/ package \texttt{PenalizedLDA} using 6-fold cross
validation.  Decision trees is implemented as the \r/ package \texttt{rpart}.
All packages used in the numerical analysis are available from the Comprehensive
\r/ Archive Network.


\subsection{Simulation results}
\label{sec:simulation-results}

The composite quantile-based classifier is abbreviated as CQC.  Results for the
second variant of composite quantile-based classifiers where the transformed
quantile distances data is augmented by the original data are also shown and are
abbreviated as CQC augmented.

Simulation results for the Gaussian setting are shown in Table
\ref{tab:gauss-corr0} and Table \ref{tab:gauss-corr08}.  We would expect
composite quantile-based classifiers to be suboptimal in this setting compared
to LDA and similar classifiers since they do not make use of the distributional
information, and are interested in the magnitude of the loss of efficiency.  One
thing to notice is that the augmented version of CQC performs better nearly
everywhere as compared to the non-augmented version.  This is to be expected
since the Bayes rule decision boundary is linear in the original features, and
is also the case for FANS2 as compared to FANS.  In this setting, nearest
shrunken centroids and penalized LDA show the best performance.  When $n$ is
large compared to $p$, marginal quantile-based classifiers perform reasonably
well, although performance deteriorates relative to other methods when
conditions are less ideal.

Simulation results for the exponentiated Gaussian setting are shown in Table
\ref{tab:exp-gauss-corr0} and Table \ref{tab:exp-gauss-corr08}.  In this setting
the 0-th quantile level for all features is optimal and quantile-based perform
extremely well.  CQC performs well also, although never as well as
quantile-based classifiers.  We see this as being due to several reasons.
Firstly, quantile-based classifiers composite quantile-based classifiers
sacrifice part of the data set in order to train the linear combination
coefficients; this has the effect of reducing the sample size to both choose
good quantile levels and to estimate the within-class quantiles.  Secondly,
since the optimal quantile level is the same for every quantile, quantile-based
classifiers is able to borrow information across all of the features to estimate
the optimal quantile level, while composite quantile-based classifiers selects
the quantile level one feature at-a-time.  Thirdly, the freedom to select linear
combination coefficients for the composite quantile-based classifiers method can
result in additional variation.  These points highlight the trade-off between
the two methods.  When the optimal quantile levels and discriminatory
information are the same across the components then quantile-based classifiers
perform better than CQC, and when this is not the case then CQC may perform
better in some settings.

Simulation results for the block transformed Gaussian setting are shown in Table
\ref{tab:block-transformed-corr0} and Table \ref{tab:block-transformed-corr08}.
In contrast to scenarios 1-4, in this setting the optimal quantile levels and
discriminatory information vary across the components, and in this setting CQC
typically performs better than quantile-based classifiers.  FANS and decision
trees are the other best-performing classifiers in this scenario.  The inclusion
of decision trees suggests that there are some relatively discriminative
features.  When the quantity of the training data decreases to $n = 50$ we see a
sharp rise in the misclassification rates for all methods.  It is in this
setting that the hybrid CQC and quantile-based classifiers do well.  For
example, when $n = 50$ and $p = 500$ with uncorrelated data, using the hybrid
approach reduces the misclassification rate from 0.351 to 0.215.  Both CQC and
quantile-based classifiers perform well in this setting compared to other
methods.  We attribute this in part to the relative simplicity of the models in
this data-sparse setting.  Additionally, the reason that the hybrid approach
works well here is that we may have a number of relatively discriminative
features that the quantile-based methods approach of selecting quantile levels
can effectively key in on, and then the additional linear combination step can
help to deal with some of the differences in scale and discriminatory power
between the blocks.

Simulation results for the beta distributed and gamma distributed data settings
are shown in Table \ref{tab:beta} and Table
\ref{tab:gamma}.  In the case of the beta distributed data,
CQC and decision trees performed the best.  We see this as a sign of a few
features having some degree of separability across the two classes.  In this
setting where the optimal quantile level varies across all of the features, we
see that CQC having more flexibility to select quantile levels results in better
performance as compared to quantile-based classifiers.  We see similar results
for the gamma distributed data, although in this case quantile-based classifiers
perform about as well as CQC.  We see this as suggesting that there happen to be
a number of features with similar optimal quantile levels that quantile-based
classifiers can select to achieve similar performance as the more flexible
approach.




\subsection{Real data considerations}

In practice, one often encounters data that has variables with both continuous
as well as categorical data.  In problems such as these categorical data is
typically reformulated into so-called dummy-coded variables.  However, the
quantile-based classifier is inherently ill-conditioned to handle such data.
When categorical variables are present in the data, we propose including them in
the classifier as untransformed dummy-coded variables.

A second type of data can also cause problems:  one where the data is a mixture
of point masses as well as continuous data.  As an example, consider univariate
data where the the quantiles of the underlying populations are given as follows.
\begin{center}
  \begin{tabular}{lrrrrrrrrrrrrr}
    \toprule
    & \multicolumn{13}{c}{\underline{Quantile levels}} \\
    & 0.01 & $\cdots$ & 0.89 & 0.90 & 0.91 & 0.92 & 0.93 & 0.94 & 0.95 & 0.96 & 0.97 & 0.98 & 0.99 \\
    \midrule
    Class 0 \hspace{5mm} & 0.00 & $\cdots$ & 0.00 & 0.02 & 0.14 & 0.23 & 0.33 & 0.42 & 0.55 & 0.68 & 0.87 & 1.05 & 1.41 \\
    Class 1 & 0.00 & $\cdots$ & 0.00 & 0.00 & 0.00 & 0.00 & 0.00 & 0.00 & 0.00 & 0.06 & 0.08 & 0.18 & 0.20 \\
    \bottomrule
  \end{tabular}
\end{center}
This example raises the question as to what the quantile distances would be for
a fixed choice of quantile level.  Suppose first that we consider a quantile
level with nonzero quantiles for both classes, say 0.97.  In this case, the
quantile distances for the data in class 0 with value 0 are larger than the
quantile distances for the data in class 1 with value 0.  As a result, the
quantile-based data will have spuriously introduced differences in the data for
90\% of the data, when in fact there is no discriminative information in those
quantiles.  Conversely, suppose now that we select a quantile level with
quantiles for both classes with value 0.  Doing so has the effect of simply
multiplying the nonzero values in the data by a constant factor (i.e. the chosen
quantile level).  While this is less disastrous then the previous scenario, we
are not using the quantile-based data approach consistent with the rest of the
paper.

What we propose to do for such a scenario is the following.  When mixture data
such as the example given above is present for a variable, we try to separate
the parts of the data that belong to the point mass contribution to the mixture,
and those that belong to the continuous part of the mixture.  Then for the point
mass data, we create a categorical variable that includes a reference category
for the continuous part of the data, and for the continuous part we perform the
quantile-based classification using only the reduced subset of the data.


% Gaussian data ----------------------------------------------------------------

\begin{table}[p]
  \centering
  \caption{Gaussian data, correlation coefficient = 0}
  \label{tab:gauss-corr0}
  \vspace{5mm}
  
  \begin{adjustbox}{max size={\textwidth}{10cm}}
    \begin{tabular}{l@{\extracolsep{15mm}}rrr}
      
      \hline
      & $n=500$ & $n=250$ & $n=50$ \\ 
      \hline
      & $p = 50$ \\
      \hline

      CQC & 0.160 (0.01) & 0.198 (0.02) & 0.286 (0.03) \\ 
      CQC augmented & 0.138 (0.01) & 0.161 (0.02) & 0.263 (0.05) \\
      Quantile-based classifiers & 0.164 (0.01) & 0.186 (0.02) & 0.282 (0.04) \\
      FANS  & 0.189 (0.02) & 0.254 (0.03) & 0.430 (0.04) \\
      FANS2 & 0.141 (0.01) & 0.176 (0.02) & 0.379 (0.07) \\
      Penalized logistic regression & 0.131 (0.01) & 0.161 (0.01) & 0.274 (0.05) \\ 
      Support vector machine & 0.118 (0.01) & 0.129 (0.01) & 0.165 (0.02) \\ 
      k-nearest neighbor & 0.193 (0.01) & 0.214 (0.02) & 0.276 (0.04) \\ 
      Naive Bayes & 0.122 (0.01) & 0.139 (0.01) & 0.213 (0.03) \\ 
      Nearest shrunken centroids & 0.114 (0.01) & 0.126 (0.01) & 0.170 (0.04) \\ 
      Penalized LDA & 0.116 (0.01) & 0.126 (0.01) & 0.161 (0.02) \\ 
      Decision trees & 0.380 (0.01) & 0.388 (0.02) & 0.435 (0.02) \\ [2ex]

      \hline
      & $p = 250$ \\
      \hline

      CQC & 0.196 (0.02) & 0.243 (0.02) & 0.391 (0.03) \\ 
      CQC augmented & 0.158 (0.01) & 0.199 (0.02) & 0.355 (0.03) \\ 
      Quantile-based classifiers & 0.169 (0.03) & 0.179 (0.02) & 0.325 (0.06) \\ 
      FANS  & 0.226 (0.01) & 0.313 (0.02) & 0.464 (0.04) \\
      FANS2 & 0.170 (0.02) & 0.229 (0.03) & 0.459 (0.04) \\
      Penalized logistic regression & 0.151 (0.01) & 0.206 (0.02) & 0.343 (0.04) \\ 
      Support vector machine & 0.143 (0.01) & 0.182 (0.02) & 0.272 (0.02) \\ 
      k-nearest neighbor & 0.190 (0.02) & 0.210 (0.02) & 0.254 (0.02) \\ 
      Naive Bayes & 0.165 (0.02) & 0.204 (0.01) & 0.332 (0.02) \\ 
      Nearest shrunken centroids & 0.120 (0.01) & 0.128 (0.01) & 0.189 (0.03) \\ 
      Penalized LDA & 0.136 (0.01) & 0.167 (0.01) & 0.289 (0.08) \\ 
      Decision trees & 0.397 (0.02) & 0.439 (0.03) & 0.456 (0.02) \\ [2ex]

      \hline
      & $p = 500$ \\
      \hline

      CQC & 0.215 (0.02) & 0.280 (0.02) & 0.423 (0.03) \\ 
      CQC augmented & 0.175 (0.01) & 0.222 (0.02) & 0.405 (0.04) \\  
      Quantile-based classifiers & 0.176 (0.01) & 0.190 (0.03) & 0.343 (0.05) \\ 
      FANS  & 0.243 (0.02) & 0.358 (0.03) & 0.498 (0.02) \\
      FANS2 & 0.192 (0.01) & 0.275 (0.03) & 0.473 (0.04) \\
      Penalized logistic regression & 0.168 (0.01) & 0.224 (0.02) & 0.411 (0.04) \\ 
      Support vector machine & 0.186 (0.02) & 0.222 (0.01) & 0.340 (0.03) \\ 
      k-nearest neighbor & 0.191 (0.02) & 0.218 (0.02) & 0.259 (0.04) \\ 
      Naive Bayes & 0.213 (0.01) & 0.260 (0.02) & 0.379 (0.02) \\ 
      Nearest shrunken centroids & 0.123 (0.01) & 0.134 (0.01) & 0.209 (0.03) \\ 
      Penalized LDA & 0.171 (0.01) & 0.212 (0.02) & 0.333 (0.03) \\ 
      Decision trees & 0.426 (0.03) & 0.425 (0.02) & 0.473 (0.04) \\ 

      \hline
      
    \end{tabular}
  \end{adjustbox}
\end{table}




\begin{table}[p]
  \centering
  \caption{Gaussian data, correlation coefficient = 0.8}
  \label{tab:gauss-corr08}
  \vspace{5mm}
  
  \begin{adjustbox}{max size={\textwidth}{10cm}}
    \begin{tabular}{l@{\extracolsep{15mm}}rrr}
      
      \hline
      & $n=500$ & $n=250$ & $n=50$ \\ 
      \hline
      & $p = 50$ \\
      \hline

      CQC & 0.358 (0.01) & 0.361 (0.02) & 0.402 (0.02) \\ 
      CQC augmented & 0.349 (0.01) & 0.353 (0.02) & 0.391 (0.02) \\ 
      Quantile-based classifiers & 0.349 (0.01) & 0.359 (0.01) & 0.414 (0.04) \\ 
      FANS  & 0.362 (0.01) & 0.352 (0.01) & 0.446 (0.04) \\
      FANS2 & 0.352 (0.01) & 0.356 (0.02) & 0.439 (0.05) \\
      Penalized logistic regression & 0.354 (0.01) & 0.360 (0.02) & 0.396 (0.03) \\ 
      Support vector machine & 0.344 (0.01) & 0.349 (0.02) & 0.389 (0.03) \\ 
      k-nearest neighbor & 0.387 (0.01) & 0.401 (0.04) & 0.429 (0.02) \\ 
      Naive Bayes & 0.339 (0.01) & 0.339 (0.01) & 0.375 (0.03) \\ 
      Nearest shrunken centroids & 0.340 (0.02) & 0.341 (0.01) & 0.381 (0.03) \\ 
      Penalized LDA & 0.337 (0.01) & 0.336 (0.02) & 0.376 (0.05) \\ 
      Decision trees & 0.403 (0.01) & 0.408 (0.02) & 0.433 (0.03) \\ [2ex]

      \hline
      & $p = 250$ \\
      \hline

      CQC & 0.370 (0.01) & 0.380 (0.02) & 0.437 (0.02) \\ 
      CQC augmented & 0.368 (0.01) & 0.379 (0.02) & 0.429 (0.02) \\ 
      Quantile-based classifiers & 0.362 (0.02) & 0.394 (0.03) & 0.428 (0.03) \\ 
      FANS  & 0.375 (0.01) & 0.409 (0.02) & 0.496 (0.01) \\
      FANS2 & 0.360 (0.01) & 0.398 (0.02) & 0.474 (0.03) \\
      Penalized logistic regression & 0.369 (0.01) & 0.395 (0.02) & 0.446 (0.02) \\ 
      Support vector machine & 0.349 (0.02) & 0.370 (0.02) & 0.403 (0.03) \\ 
      k-nearest neighbor & 0.384 (0.01) & 0.392 (0.02) & 0.430 (0.04) \\ 
      Naive Bayes & 0.342 (0.02) & 0.349 (0.01) & 0.401 (0.02) \\ 
      Nearest shrunken centroids & 0.336 (0.01) & 0.344 (0.01) & 0.379 (0.02) \\ 
      Penalized LDA & 0.337 (0.01) & 0.344 (0.01) & 0.374 (0.02) \\ 
      Decision trees & 0.438 (0.03) & 0.447 (0.01) & 0.459 (0.02) \\ [2ex]

      \hline
      & $p = 500$ \\
      \hline

      CQC & 0.369 (0.02) & 0.394 (0.02) & 0.427 (0.03) \\ 
      CQC augmented & 0.362 (0.01) & 0.388 (0.02) & 0.442 (0.04) \\ 
      Quantile-based classifiers & 0.395 (0.01) & 0.402 (0.03) & 0.402 (0.02) \\ 
      FANS  & 0.375 (0.02) & 0.428 (0.03) & 0.483 (0.03) \\
      FANS2 & 0.368 (0.02) & 0.411 (0.04) & 0.468 (0.04) \\
      Penalized logistic regression & 0.360 (0.02) & 0.387 (0.03) & 0.441 (0.04) \\ 
      Support vector machine & 0.349 (0.01) & 0.365 (0.03) & 0.385 (0.03) \\ 
      k-nearest neighbor & 0.386 (0.03) & 0.386 (0.04) & 0.403 (0.04) \\ 
      Naive Bayes & 0.338 (0.01) & 0.360 (0.02) & 0.391 (0.03) \\ 
      Nearest shrunken centroids & 0.331 (0.01) & 0.349 (0.04) & 0.349 (0.03) \\ 
      Penalized LDA & 0.336 (0.01) & 0.346 (0.02) & 0.379 (0.05) \\ 
      Decision trees & 0.434 (0.02) & 0.444 (0.03) & 0.480 (0.03) \\

      \hline
      
    \end{tabular}
  \end{adjustbox}
\end{table}




\begin{table}[p]
  \centering
  \caption{Exponentiated Gaussian data, correlation coefficient = 0}
  \label{tab:exp-gauss-corr0}
  \vspace{5mm}
  
  \begin{adjustbox}{max size={\textwidth}{10cm}}
    \begin{tabular}{l@{\extracolsep{15mm}}rrr}
      
      \hline
      & $n=500$ & $n=250$ & $n=50$ \\ 
      \hline
      & $p = 50$ \\
      \hline

      CQC & 0.012 (0.00) & 0.030 (0.01) & 0.174 (0.04) \\ 
      CQC augmented & 0.012 (0.00) & 0.031 (0.01) & 0.206 (0.03) \\ 
      Quantile-based classifiers & 0.000 (0.00) & 0.001 (0.00) & 0.112 (0.05) \\ 
      FANS  & 0.039 (0.01) & 0.097 (0.02) & 0.372 (0.05) \\
      FANS2 & 0.042 (0.01) & 0.115 (0.03) & 0.384 (0.03) \\
      Penalized logistic regression & 0.293 (0.02) & 0.316 (0.02) & 0.399 (0.03) \\ 
      Support vector machine & 0.223 (0.02) & 0.252 (0.02) & 0.354 (0.03) \\ 
      k-nearest neighbor & 0.376 (0.02) & 0.386 (0.03) & 0.424 (0.03) \\ 
      Naive Bayes & 0.433 (0.01) & 0.439 (0.02) & 0.462 (0.02) \\ 
      Nearest shrunken centroids & 0.283 (0.01) & 0.313 (0.03) & 0.384 (0.04) \\ 
      Penalized LDA & 0.284 (0.01) & 0.308 (0.02) & 0.352 (0.02) \\ 
      Decision trees & 0.114 (0.01) & 0.191 (0.06) & 0.327 (0.02) \\ [2ex]

      \hline
      & $p = 250$ \\
      \hline

      CQC & 0.017 (0.00) & 0.031 (0.01) & 0.198 (0.03) \\ 
      CQC augmented & 0.016 (0.00) & 0.038 (0.01) & 0.222 (0.04) \\ 
      Quantile-based classifiers & 0.001 (0.00) & 0.004 (0.00) & 0.135 (0.04) \\ 
      FANS  & 0.050 (0.01) & 0.123 (0.02) & 0.459 (0.04) \\
      FANS2 & 0.055 (0.01) & 0.132 (0.02) & 0.462 (0.04) \\
      Penalized logistic regression & 0.359 (0.02) & 0.394 (0.02) & 0.461 (0.02) \\ 
      Support vector machine & 0.339 (0.01) & 0.378 (0.02) & 0.429 (0.03) \\ 
      k-nearest neighbor & 0.383 (0.04) & 0.392 (0.03) & 0.428 (0.03) \\ 
      Naive Bayes & 0.440 (0.02) & 0.455 (0.02) & 0.471 (0.01) \\ 
      Nearest shrunken centroids & 0.309 (0.02) & 0.327 (0.01) & 0.404 (0.05) \\ 
      Penalized LDA & 0.348 (0.01) & 0.385 (0.02) & 0.439 (0.04) \\ 
      Decision trees & 0.128 (0.02) & 0.160 (0.02) & 0.375 (0.07) \\ [2ex]

      \hline
      & $p = 500$ \\
      \hline

      CQC & 0.016 (0.00) & 0.036 (0.01) & 0.225 (0.03) \\ 
      CQC augmented & 0.017 (0.01) & 0.043 (0.01) & 0.292 (0.03) \\ 
      Quantile-based classifiers & 0.001 (0.00) & 0.005 (0.00) & 0.148 (0.05) \\ 
      FANS  & 0.051 (0.01) & 0.149 (0.02) & 0.453 (0.04) \\
      FANS2 & 0.057 (0.01) & 0.162 (0.02) & 0.477 (0.04) \\
      Penalized logistic regression & 0.374 (0.01) & 0.422 (0.03) & 0.467 (0.03) \\ 
      Support vector machine & 0.375 (0.01) & 0.410 (0.02) & 0.448 (0.02) \\ 
      k-nearest neighbor & 0.387 (0.01) & 0.400 (0.02) & 0.444 (0.03) \\ 
      Naive Bayes & 0.453 (0.01) & 0.457 (0.01) & 0.480 (0.02) \\ 
      Nearest shrunken centroids & 0.320 (0.02) & 0.324 (0.02) & 0.394 (0.04) \\ 
      Penalized LDA & 0.378 (0.02) & 0.399 (0.02) & 0.465 (0.03) \\ 
      Decision trees & 0.125 (0.02) & 0.164 (0.02) & 0.409 (0.08) \\ 

      \hline
      
    \end{tabular}
  \end{adjustbox}
\end{table}




\begin{table}[p]
  \centering
  \caption{Exponentiated Gaussian data, correlation coefficient = 0.8}
  \label{tab:exp-gauss-corr08}
  \vspace{5mm}
  
  \begin{adjustbox}{max size={\textwidth}{10cm}}
    \begin{tabular}{l@{\extracolsep{15mm}}rrr}
      
      \hline
      & $n=500$ & $n=250$ & $n=50$ \\ 
      \hline
      & $p = 50$ \\
      \hline

      CQC & 0.098 (0.01) & 0.120 (0.01) & 0.223 (0.02) \\ 
      CQC augmented & 0.108 (0.01) & 0.132 (0.01) & 0.232 (0.02) \\ 
      Quantile-based classifiers & 0.061 (0.01) & 0.064 (0.01) & 0.133 (0.03) \\ 
      FANS  & 0.177 (0.02) & 0.232 (0.03) & 0.388 (0.06) \\
      FANS2 & 0.184 (0.02) & 0.235 (0.02) & 0.405 (0.04) \\
      Penalized logistic regression & 0.430 (0.01) & 0.426 (0.01) & 0.430 (0.02) \\ 
      Support vector machine & 0.355 (0.02) & 0.362 (0.02) & 0.417 (0.03) \\ 
      k-nearest neighbor & 0.416 (0.02) & 0.412 (0.02) & 0.446 (0.02) \\ 
      Naive Bayes & 0.436 (0.03) & 0.445 (0.03) & 0.453 (0.02) \\ 
      Nearest shrunken centroids & 0.414 (0.02) & 0.418 (0.03) & 0.432 (0.02) \\ 
      Penalized LDA & 0.405 (0.01) & 0.412 (0.01) & 0.432 (0.04) \\ 
      Decision trees & 0.155 (0.02) & 0.201 (0.03) & 0.327 (0.02) \\ [2ex]

      \hline
      & $p = 250$ \\
      \hline

      CQC & 0.128 (0.01) & 0.164 (0.02) & 0.242 (0.03) \\ 
      CQC augmented & 0.134 (0.01) & 0.167 (0.02) & 0.305 (0.03) \\ 
      Quantile-based classifiers & 0.060 (0.01) & 0.068 (0.01) & 0.157 (0.03) \\ 
      FANS  & 0.197 (0.02) & 0.265 (0.02) & 0.457 (0.05) \\
      FANS2 & 0.197 (0.01) & 0.271 (0.02) & 0.463 (0.05) \\
      Penalized logistic regression & 0.435 (0.02) & 0.458 (0.03) & 0.489 (0.02) \\ 
      Support vector machine & 0.404 (0.02) & 0.433 (0.02) & 0.450 (0.02) \\ 
      k-nearest neighbor & 0.403 (0.02) & 0.417 (0.03) & 0.448 (0.02) \\ 
      Naive Bayes & 0.450 (0.03) & 0.461 (0.03) & 0.465 (0.01) \\ 
      Nearest shrunken centroids & 0.414 (0.02) & 0.434 (0.02) & 0.421 (0.02) \\ 
      Penalized LDA & 0.407 (0.01) & 0.438 (0.04) & 0.464 (0.03) \\ 
      Decision trees & 0.144 (0.02) & 0.208 (0.03) & 0.416 (0.05) \\ [2ex]

      \hline
      & $p = 500$ \\
      \hline

      CQC & 0.135 (0.01) & 0.160 (0.01) & 0.268 (0.02) \\ 
      CQC augmented & 0.138 (0.01) & 0.172 (0.02) & 0.344 (0.05) \\ 
      Quantile-based classifiers & 0.060 (0.01) & 0.065 (0.01) & 0.169 (0.03) \\ 
      FANS  & 0.205 (0.02) & 0.268 (0.03) & 0.453 (0.05) \\
      FANS2 & 0.206 (0.02) & 0.271 (0.03) & 0.476 (0.04) \\
      Penalized logistic regression & 0.461 (0.02) & 0.458 (0.02) & 0.490 (0.01) \\ 
      Support vector machine & 0.441 (0.02) & 0.453 (0.02) & 0.459 (0.03) \\ 
      k-nearest neighbor & 0.415 (0.02) & 0.428 (0.02) & 0.438 (0.04) \\ 
      Naive Bayes & 0.456 (0.03) & 0.458 (0.02) & 0.471 (0.02) \\ 
      Nearest shrunken centroids & 0.426 (0.02) & 0.424 (0.02) & 0.427 (0.02) \\ 
      Penalized LDA & 0.426 (0.02) & 0.436 (0.02) & 0.459 (0.03) \\ 
      Decision trees & 0.156 (0.02) & 0.196 (0.03) & 0.412 (0.07) \\ 

      \hline
      
    \end{tabular}
  \end{adjustbox}
\end{table}



\begin{table}[p]
  \centering
  \caption{Block transformed data, correlation coefficient = 0}
  \label{tab:block-transformed-corr0}
  \vspace{5mm}
  
  \begin{adjustbox}{max size={\textwidth}{10cm}}
    \begin{tabular}{l@{\extracolsep{15mm}}rrr}
      
      \hline
      & $n=500$ & $n=250$ & $n=50$ \\ 
      \hline
      & $p = 50$ \\
      \hline

      CQC & 0.029 (0.01) & 0.043 (0.01) & 0.181 (0.03) \\ 
      CQC augmented & 0.030 (0.01) & 0.048 (0.01) & 0.176 (0.05) \\ 
      Quantile-based classifiers & 0.085 (0.01) & 0.112 (0.02) & 0.223 (0.03) \\ 
      FANS  & 0.027 (0.01) & 0.068 (0.01) & 0.320 (0.08) \\
      FANS2 & 0.028 (0.01) & 0.063 (0.01) & 0.323 (0.06) \\
      Penalized logistic regression & 0.291 (0.02) & 0.322 (0.02) & 0.431 (0.03) \\ 
      Support vector machine & 0.284 (0.01) & 0.306 (0.02) & 0.386 (0.02) \\ 
      k-nearest neighbor & 0.431 (0.02) & 0.448 (0.02) & 0.464 (0.02) \\ 
      Naive Bayes & 0.374 (0.02) & 0.396 (0.02) & 0.452 (0.01) \\ 
      Nearest shrunken centroids & 0.330 (0.02) & 0.339 (0.01) & 0.422 (0.05) \\ 
      Penalized LDA & 0.283 (0.01) & 0.301 (0.02) & 0.374 (0.03) \\ 
      Decision trees & 0.040 (0.01) & 0.078 (0.01) & 0.374 (0.08) \\ [2ex]

      \hline
      & $p = 250$ \\
      \hline

      CQC & 0.045 (0.01) & 0.062 (0.01) & 0.222 (0.08) \\ 
      CQC augmented & 0.047 (0.01) & 0.063 (0.01) & 0.234 (0.08) \\ 
      Quantile-based classifiers & 0.089 (0.01) & 0.115 (0.01) & 0.234 (0.04) \\ 
      FANS  & 0.036 (0.01) & 0.083 (0.01) & 0.374 (0.07) \\
      FANS2 & 0.038 (0.01) & 0.080 (0.01) & 0.398 (0.08) \\
      Penalized logistic regression & 0.343 (0.02) & 0.382 (0.03) & 0.465 (0.03) \\ 
      Support vector machine & 0.344 (0.01) & 0.382 (0.01) & 0.424 (0.02) \\ 
      k-nearest neighbor & 0.451 (0.02) & 0.446 (0.03) & 0.470 (0.03) \\ 
      Naive Bayes & 0.408 (0.01) & 0.426 (0.02) & 0.472 (0.01) \\ 
      Nearest shrunken centroids & 0.303 (0.02) & 0.318 (0.02) & 0.403 (0.05) \\ 
      Penalized LDA & 0.340 (0.01) & 0.372 (0.02) & 0.435 (0.03) \\ 
      Decision trees & 0.038 (0.01) & 0.072 (0.02) & 0.354 (0.07) \\ [2ex]

      \hline
      & $p = 500$ \\
      \hline

      CQC & 0.046 (0.01) & 0.065 (0.01) & 0.215 (0.04) \\ 
      CQC augmented & 0.050 (0.01) & 0.070 (0.01) & 0.267 (0.07) \\ 
      Quantile-based classifiers & 0.093 (0.02) & 0.131 (0.02) & 0.244 (0.04) \\ 
      FANS  & 0.033 (0.01) & 0.102 (0.02) & 0.404 (0.07) \\
      FANS2 & 0.033 (0.01) & 0.094 (0.02) & 0.422 (0.07) \\
      Penalized logistic regression & 0.342 (0.01) & 0.414 (0.04) & 0.472 (0.03) \\ 
      Support vector machine & 0.367 (0.01) & 0.400 (0.01) & 0.451 (0.02) \\ 
      k-nearest neighbor & 0.434 (0.02) & 0.439 (0.02) & 0.464 (0.03) \\ 
      Naive Bayes & 0.420 (0.02) & 0.440 (0.02) & 0.474 (0.02) \\ 
      Nearest shrunken centroids & 0.300 (0.01) & 0.318 (0.01) & 0.406 (0.04) \\ 
      Penalized LDA & 0.358 (0.02) & 0.392 (0.02) & 0.470 (0.03) \\ 
      Decision trees & 0.043 (0.02) & 0.076 (0.03) & 0.406 (0.09) \\ 

      \hline
      
    \end{tabular}
  \end{adjustbox}
\end{table}




\begin{table}[p]
  \centering
  \caption{Block transformed data, correlation coefficient = 0.8}
  \label{tab:block-transformed-corr08}
  \vspace{5mm}  

  \begin{adjustbox}{max size={\textwidth}{10cm}}
    \begin{tabular}{l@{\extracolsep{15mm}}rrr}
      
      \hline
      & $n=500$ & $n=250$ & $n=50$ \\ 
      \hline
      & $p = 50$ \\
      \hline

      CQC & 0.064 (0.01) & 0.081 (0.01) & 0.224 (0.04) \\ 
      CQC augmented & 0.066 (0.01) & 0.086 (0.01) & 0.234 (0.06) \\ 
      Quantile-based classifiers & 0.107 (0.01) & 0.134 (0.02) & 0.280 (0.08) \\ 
      FANS  & 0.076 (0.01) & 0.115 (0.02) & 0.302 (0.06) \\
      FANS2 & 0.076 (0.01) & 0.117 (0.01) & 0.308 (0.06) \\
      Penalized logistic regression & 0.396 (0.02) & 0.409 (0.02) & 0.443 (0.03) \\ 
      Support vector machine & 0.375 (0.03) & 0.384 (0.02) & 0.420 (0.03) \\ 
      k-nearest neighbor & 0.444 (0.02) & 0.450 (0.02) & 0.480 (0.03) \\ 
      Naive Bayes & 0.410 (0.02) & 0.420 (0.02) & 0.454 (0.04) \\ 
      Nearest shrunken centroids & 0.416 (0.03) & 0.422 (0.01) & 0.462 (0.04) \\ 
      Penalized LDA & 0.383 (0.02) & 0.388 (0.02) & 0.430 (0.05) \\ 
      Decision trees & 0.076 (0.01) & 0.115 (0.02) & 0.313 (0.07) \\ [2ex]

      \hline
      & $p = 250$ \\
      \hline

      CQC & 0.091 (0.00) & 0.115 (0.01) & 0.228 (0.03) \\ 
      CQC augmented & 0.092 (0.00) & 0.118 (0.01) & 0.293 (0.05) \\ 
      Quantile-based classifiers & 0.126 (0.03) & 0.154 (0.02) & 0.263 (0.02) \\ 
      FANS  & 0.079 (0.01) & 0.128 (0.01) & 0.428 (0.08) \\
      FANS2 & 0.085 (0.01) & 0.139 (0.01) & 0.402 (0.07) \\
      Penalized logistic regression & 0.421 (0.02) & 0.432 (0.03) & 0.477 (0.03) \\ 
      Support vector machine & 0.416 (0.02) & 0.428 (0.03) & 0.452 (0.02) \\ 
      k-nearest neighbor & 0.452 (0.03) & 0.450 (0.02) & 0.463 (0.02) \\ 
      Naive Bayes & 0.435 (0.02) & 0.436 (0.02) & 0.466 (0.02) \\ 
      Nearest shrunken centroids & 0.392 (0.02) & 0.388 (0.02) & 0.434 (0.02) \\ 
      Penalized LDA & 0.397 (0.02) & 0.403 (0.02) & 0.460 (0.03) \\ 
      Decision trees & 0.081 (0.01) & 0.118 (0.02) & 0.443 (0.07) \\ [2ex]

      \hline
      & $p = 500$ \\
      \hline

      CQC & 0.090 (0.01) & 0.125 (0.01) & 0.238 (0.04) \\ 
      CQC augmented & 0.090 (0.01) & 0.123 (0.01) & 0.312 (0.06) \\ 
      Quantile-based classifiers & 0.115 (0.03) & 0.158 (0.02) & 0.277 (0.03) \\ 
      FANS  & 0.092 (0.01) & 0.146 (0.02) & 0.429 (0.07) \\
      FANS2 & 0.095 (0.01) & 0.148 (0.02) & 0.415 (0.07) \\
      Penalized logistic regression & 0.430 (0.02) & 0.444 (0.02) & 0.480 (0.02) \\ 
      Support vector machine & 0.427 (0.02) & 0.436 (0.02) & 0.460 (0.02) \\ 
      k-nearest neighbor & 0.459 (0.02) & 0.457 (0.01) & 0.487 (0.01) \\ 
      Naive Bayes & 0.436 (0.01) & 0.450 (0.02) & 0.480 (0.03) \\ 
      Nearest shrunken centroids & 0.397 (0.02) & 0.398 (0.03) & 0.423 (0.03) \\ 
      Penalized LDA & 0.411 (0.01) & 0.423 (0.02) & 0.465 (0.04) \\ 
      Decision trees & 0.084 (0.02) & 0.146 (0.03) & 0.421 (0.10) \\ 

      \hline
      
    \end{tabular}
  \end{adjustbox}
\end{table}





\begin{table}[p]
  \centering
  \caption{Beta distributed data}
  \label{tab:beta}
  \vspace{5mm}
  
  \begin{adjustbox}{max size={\textwidth}{10cm}}
    \begin{tabular}{l@{\extracolsep{15mm}}rrr}
      
      \hline
      & $n=500$ & $n=250$ & $n=50$ \\ 
      \hline
      & $p = 50$ \\
      \hline

      CQC & 0.009 (0.01) & 0.016 (0.01) & 0.060 (0.04) \\ 
      CQC augmented & 0.010 (0.01) & 0.015 (0.01) & 0.068 (0.03) \\ 
      Quantile-based classifiers & 0.019 (0.01) & 0.031 (0.01) & 0.158 (0.03) \\ 
      FANS  & 0.015 (0.00) & 0.025 (0.02) & 0.053 (0.03) \\
      FANS2 & 0.016 (0.00) & 0.023 (0.01) & 0.072 (0.05) \\
      Penalized logistic regression & 0.501 (0.02) & 0.511 (0.02) & 0.420 (0.17) \\ 
      Support vector machine & 0.438 (0.02) & 0.475 (0.03) & 0.445 (0.10) \\ 
      k-nearest neighbor & 0.484 (0.01) & 0.491 (0.01) & 0.505 (0.02) \\ 
      Naive Bayes & 0.327 (0.11) & 0.332 (0.13) & 0.195 (0.16) \\ 
      Nearest shrunken centroids & 0.495 (0.02) & 0.512 (0.01) & 0.498 (0.01) \\ 
      Penalized LDA & 0.507 (0.01) & 0.503 (0.01) & 0.420 (0.17) \\ 
      Decision trees & 0.019 (0.00) & 0.029 (0.01) & 0.071 (0.03) \\ [2ex]

      \hline
      & $p = 250$ \\
      \hline

      CQC & 0.012 (0.00) & 0.011 (0.00) & 0.049 (0.02) \\ 
      CQC augmented & 0.007 (0.00) & 0.011 (0.01) & 0.063 (0.03) \\ 
      Quantile-based classifiers & 0.108 (0.01) & 0.173 (0.03) & 0.301 (0.02) \\ 
      FANS  & 0.018 (0.01) & 0.024 (0.01) & 0.059 (0.04) \\
      FANS2 & 0.017 (0.01) & 0.026 (0.01) & 0.059 (0.04) \\
      Penalized logistic regression & 0.505 (0.02) & 0.495 (0.02) & 0.464 (0.11) \\ 
      Support vector machine & 0.494 (0.02) & 0.500 (0.02) & 0.499 (0.02) \\ 
      k-nearest neighbor & 0.490 (0.02) & 0.490 (0.02) & 0.507 (0.02) \\ 
      Naive Bayes & 0.413 (0.10) & 0.331 (0.18) & 0.434 (0.06) \\ 
      Nearest shrunken centroids & 0.504 (0.01) & 0.494 (0.02) & 0.496 (0.01) \\ 
      Penalized LDA & 0.501 (0.02) & 0.500 (0.01) & 0.494 (0.01) \\ 
      Decision trees & 0.016 (0.00) & 0.022 (0.01) & 0.076 (0.06) \\ [2ex]

      \hline
      & $p = 500$ \\
      \hline

      CQC & 0.011 (0.01) & 0.019 (0.01) & 0.075 (0.03) \\ 
      CQC augmented & 0.012 (0.00) & 0.021 (0.01) & 0.068 (0.03) \\ 
      Quantile-based classifiers & 0.164 (0.02) & 0.228 (0.03) & 0.356 (0.02) \\ 
      FANS  & 0.020 (0.01) & 0.028 (0.01) & 0.073 (0.07) \\
      FANS2 & 0.022 (0.01) & 0.031 (0.01) & 0.072 (0.05) \\
      Penalized logistic regression & 0.506 (0.01) & 0.485 (0.02) & 0.460 (0.13) \\ 
      Support vector machine & 0.506 (0.02) & 0.501 (0.02) & 0.497 (0.02) \\ 
      k-nearest neighbor & 0.495 (0.02) & 0.498 (0.01) & 0.499 (0.02) \\ 
      Naive Bayes & 0.459 (0.06) & 0.441 (0.05) & 0.462 (0.03) \\ 
      Nearest shrunken centroids & 0.500 (0.01) & 0.492 (0.02) & 0.497 (0.01) \\ 
      Penalized LDA & 0.498 (0.01) & 0.498 (0.02) & 0.491 (0.02) \\ 
      Decision trees & 0.013 (0.01) & 0.030 (0.01) & 0.060 (0.03) \\ 

      \hline
      
    \end{tabular}
  \end{adjustbox}
\end{table}




\begin{table}[p]
  \centering
  \caption{Gamma distributed data}
  \label{tab:gamma}
  \vspace{5mm}
  
  \begin{adjustbox}{max size={\textwidth}{10cm}}
    \begin{tabular}{l@{\extracolsep{15mm}}rrr}
      
      \hline
      & $n=500$ & $n=250$ & $n=50$ \\ 
      \hline
      & $p = 50$ \\
      \hline

      CQC & 0.035 (0.01) & 0.052 (0.01) & 0.128 (0.03) \\ 
      CQC augmented & 0.034 (0.01) & 0.054 (0.01) & 0.153 (0.04) \\ 
      Quantile-based classifiers & 0.045 (0.01) & 0.062 (0.02) & 0.156 (0.01) \\ 
      FANS  & 0.069 (0.01) & 0.107 (0.02) & 0.389 (0.10) \\
      FANS2 & 0.073 (0.01) & 0.116 (0.02) & 0.363 (0.09) \\
      Penalized logistic regression & 0.501 (0.02) & 0.500 (0.02) & 0.493 (0.01) \\ 
      Support vector machine & 0.424 (0.01) & 0.462 (0.02) & 0.492 (0.01) \\ 
      k-nearest neighbor & 0.486 (0.02) & 0.489 (0.02) & 0.495 (0.01) \\ 
      Naive Bayes & 0.341 (0.02) & 0.369 (0.02) & 0.424 (0.02) \\ 
      Nearest shrunken centroids & 0.503 (0.01) & 0.506 (0.02) & 0.506 (0.02) \\ 
      Penalized LDA & 0.501 (0.01) & 0.506 (0.02) & 0.498 (0.00) \\ 
      Decision trees & 0.038 (0.01) & 0.060 (0.01) & 0.272 (0.11) \\ [2ex]

      \hline
      & $p = 250$ \\
      \hline

      CQC & 0.041 (0.01) & 0.070 (0.02) & 0.228 (0.12) \\ 
      CQC augmented & 0.046 (0.01) & 0.083 (0.03) & 0.272 (0.11) \\ 
      Quantile-based classifiers & 0.055 (0.02) & 0.070 (0.02) & 0.246 (0.13) \\ 
      FANS  & 0.093 (0.03) & 0.146 (0.03) & 0.392 (0.09) \\
      FANS2 & 0.113 (0.05) & 0.148 (0.04) & 0.413 (0.07) \\
      Penalized logistic regression & 0.502 (0.01) & 0.503 (0.01) & 0.503 (0.01) \\ 
      Support vector machine & 0.496 (0.01) & 0.501 (0.01) & 0.495 (0.01) \\ 
      k-nearest neighbor & 0.481 (0.01) & 0.489 (0.02) & 0.492 (0.01) \\ 
      Naive Bayes & 0.491 (0.01) & 0.494 (0.01) & 0.501 (0.01) \\ 
      Nearest shrunken centroids & 0.492 (0.02) & 0.501 (0.01) & 0.493 (0.01) \\ 
      Penalized LDA & 0.500 (0.00) & 0.500 (0.01) & 0.498 (0.01) \\ 
      Decision trees & 0.040 (0.01) & 0.075 (0.03) & 0.296 (0.09) \\ [2ex]

      \hline
      & $p = 500$ \\
      \hline

      CQC & 0.045 (0.01) & 0.070 (0.01) & 0.211 (0.05) \\ 
      CQC augmented & 0.049 (0.01) & 0.081 (0.01) & 0.335 (0.08) \\ 
      Quantile-based classifiers & 0.062 (0.01) & 0.089 (0.02) & 0.192 (0.03) \\ 
      FANS  & 0.083 (0.01) & 0.128 (0.02) & 0.374 (0.11) \\
      FANS2 & 0.091 (0.02) & 0.162 (0.05) & 0.424 (0.09) \\
      Penalized logistic regression & 0.502 (0.01) & 0.503 (0.02) & 0.503 (0.01) \\ 
      Support vector machine & 0.506 (0.01) & 0.496 (0.01) & 0.501 (0.00) \\ 
      k-nearest neighbor & 0.496 (0.02) & 0.490 (0.02) & 0.499 (0.02) \\ 
      Naive Bayes & 0.492 (0.01) & 0.494 (0.01) & 0.498 (0.01) \\ 
      Nearest shrunken centroids & 0.504 (0.01) & 0.504 (0.02) & 0.504 (0.01) \\ 
      Penalized LDA & 0.504 (0.01) & 0.501 (0.01) & 0.496 (0.01) \\ 
      Decision trees & 0.039 (0.01) & 0.110 (0.13) & 0.341 (0.12) \\ 

      \hline
      
    \end{tabular}
  \end{adjustbox}
\end{table}




%%% Local Variables:
%%% mode: latex
%%% TeX-master: "supplemental"
%%% End:


\bibliography{supplemental}

\end{document}




%%% Local Variables:
%%% mode: latex
%%% TeX-master: t
%%% End:
