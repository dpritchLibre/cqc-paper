
\section{Introduction}
\label{sec:intro}

A classifier is a type of supervised learning problem where the goal is to build
a rule for predicting the class membership of an observation based on a set of
features.  The rule is constructed from a training dataset consisting of the
class membership and features for each training sample.  There are numerous
applications for classification, including disease classification, image or
sound recognition, object discrimination, and spam detection, among many others.

Classification has a long and historied place in the literature.  Some important
early methods include naive Bayes, linear and quadratic discriminant analysis,
nearest neighbor methods \cite{cover1967}, and logistic regression.  Examples of
more recent methods include kernel smoothing \cite{mika1999}, neural networks
\cite{ripley1994}, and mixture discriminant analysis \cite{hastie1996}.  These
methods often perform well in the classical low dimensional setting, where the
number of features is smaller than the training data sample size.  However, in
high dimensional settings some of these methods can have identifiability issues,
be computationally demanding, or suffer from poor performance due to the curse
of dimensionality.

Various methods have been proposed for classification in the high-dimensional
setting.  The support vector machine (SVM) \cite{cortes1995} and penalized
logistic regression \cite{park2007} are two well-known approaches that have been
successfully applied in many such settings.  More generally, methods designed
for the high dimensional setting often employ techniques such as shrinkage,
dimension reduction, or distance-based methods, sometimes further restricting
attention to the marginal information provided by the features in the data.
Techniques performing shrinkage include penalized logistic regression and
penalized linear discriminant analysis \cite{tibshirani2002, clemmensen2011,
  witten2011}.  Dimension reduction is another valuable technique that is
employed by methods such as sure independence screening \cite{fan2008} and
supervised principal components analysis \cite{bair2006}.  Nonparametric
distance-based classifiers such as centroid-based classifiers
\cite{tibshirani2002} and median based classifiers \cite{jornsten2004,
  ghosh2005} have been proposed, as well as component-wise median-based
classifiers \cite{hall2012} and quantile-based classifiers \cite{hennig2016}.

In this paper we continue to explore quantile-based classifiers as introduced in
\cite{hennig2016} in the high-dimensional setting.  This family of classifiers
is based upon a comparison of the component-wise distances of the feature vector
of an observation to the within-class quantiles.  The quantile-based family of
methods then classifies an observation as belonging to the class which has the
smaller aggregated component-wise distance from the feature vector of the
observation to the within-class quantiles, where the aggregated distance is
defined as the sum of the individual distances.  The quantile-based classifiers
in \cite{hennig2016} are a single-parameter family of classifiers with the
parameter specifying a common quantile level for each component at which to
compare the component-wise distances of an observation to.  This classification
approach can work well in certain settings, in particular for the setting where
the optimal choice of the quantile level is the same for each component.
However, this restriction on the choice of quantile level can lead to a loss of
efficiency in settings where the optimal choice of the quantile level varies
across components.  This naturally leads to the question of whether there is a
way to allow for more flexibility in the choice of quantile levels in order to
achieve improved performance in these other scenarios.

It was established in \cite{hennig2016} that for univariate data and under some
assuptions, the decision rule based upon the distances of an observation to the
corresponding within-class quantiles for the optimal choice of quantile levels
is the Bayes rule.  This result motivates the methodologies developed in this
paper.  Conceptually, our goal is to use these most powerful univariate
classifiers as building blocks for a multivariate classifier.  In brief, we
propose aggregating the component-wise distances from the feature vector of the
observation to the within-class quantiles corresponding to the one-at-a-time
optimal choices of the quantile level.  Aggregation is performed through an
appropriately chosen linear combination of the component-wise distances.

The remainder of this paper procedes as follows.  In Section
\ref{sec:univariate-classifier}, we first review the quantile classifier, a
univariate distance-based classification method and its sample version.
Properties of the quantile classifier are discussed, and consistency of the
empirically optimal quantile classifier is established.  An algorithm with which
to calculate the estimated classification rate for the family of quantile
classifiers as a function of the quantile level is presented.  In Section
\ref{sec:multivariate-classifier}, the \emph{composite quantile-based} family of
classification methods is presented.  Connections to other related classifiers
are discussed.  Consistency of the empirically optimal composite quantile-based
classifier is established.  A new method for selection of a composite
quantile-based classifier based on training data is proposed, and a
corresponding algorithm is presented.  Properties and the form of composite
quantile-based classifiers are discussed.  In Section
\ref{sec:numerical-results}, the competitive performance of these approaches is
demonstrated using simulation studies as well as on a benchmark email spam
application.  Proofs of theoretical results are presented in the appendix.

% In Section \ref{sec:univariate-classifier}, we present the quantile classifier
% rule as proposed in \cite{hennig2016}.  We describe an algorithm for
% calculating the empirical quantile classifier rule and show that this rule
% converges to the optimal quantile classifier rule (and hence the Bayes rule) in
% the limit and the sample size increases.  In Section
% \ref{sec:multivariate-classifier}, we consider ways of constructing a
% multivariate classifier from the quantile classifier rules based on the
% individual features.  One method that we consider is to sum the component-wise
% distances from a new point to the class quantiles.  However this method has some
% drawbacks, as is discussed in Section \ref{sec:aggregate-classifier}.  To
% alleviate some of these issues, a second approach is proposed, where the
% decision rule is based on a linear combination of the component-wise quantile
% distances.




%%% Local Variables:
%%% mode: latex
%%% TeX-master: "cqc_paper"
%%% End:
