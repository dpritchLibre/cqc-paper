
\section{Appendix}
\label{sec:appendix}

\begin{proof}[Proof of Lemma \ref{lem:univariate-consistency}.]
  The fact that
  $\Psi(\hat{\theta}_n) \stackrel{p}{\longrightarrow} \Psi(\tilde{\theta})$ is a
  special case of Theorem 1 in \cite{hennig2016} for a feature-space of
  dimension 1.  Furthermore, during the proof of that theorem it was shown that
  under Assumptions 1 and 2, $\Psi$ is a continuous function of $\theta$.  To
  show that $\hat{\theta}_n \stackrel{p}{\longrightarrow} \tilde{\theta}$
  suppose that the claim doesn't hold.  Then there exists an $\epsilon > 0$ and
  $\delta > 0$ such that for all $N \in \mathbb{N}$, there exists an $n \geq N$
  such that
  \begin{equation*}
    \prob\Big(
    \big| \hat{\theta}_n - \tilde{\theta} \big|
    > \epsilon \Big)
    \geq \delta.
  \end{equation*}
  Now, because $\Psi$ is continuous and $\Psi(\tilde{\theta})$ is a unique
  maximum, it follows that we can find $\nu > 0$ such that
  \begin{equation*}
    \min \left\{
      \big| \Psi(\Tilde{\theta} - \epsilon) - \Psi(\tilde{\theta}) \big|\,,
      \hspace{2mm}
      \big| \Psi(\Tilde{\theta} + \epsilon) - \Psi(\tilde{\theta}) \big|
    \right\} \geq \nu.
  \end{equation*}
  Therefore, for all $N \in \mathbb{N}$, there exists an $n \geq N$ such that
  \begin{equation*}
    \prob\Big(
    \big| \Psi(\hat{\theta}_n) - \Psi(\tilde{\theta}) \big|
    \geq \nu \Big) \geq
    \prob\Big(
    \big| \hat{\theta}_n - \tilde{\theta} \big|
    \geq \epsilon \Big)
    \geq \delta.
  \end{equation*}
  But this is in contradiction to the fact that
  $\Psi(\hat{\theta}_n) \stackrel{p}{\longrightarrow} \Psi(\tilde{\theta})$.
\end{proof}


\begin{proof}[Proof of Lemma \ref{lem:decision-boundary}.]
  Suppose $F_{(0)}^{-1}(\theta) \ne F_{(1)}^{-1}(\theta)$.  It is clear
  (e.g. see Figure \ref{fig:phi-lambda}) that $\Phi_{(0)}(z, \theta)$ is
  equal to $\Phi_{(1)}(z, \theta)$ at exactly one point, say $\tau$, and that
  the following holds:
  \begin{equation*}
    \left\{
      \begin{array}{lll}
        \Phi_{(0)}(z, \theta) < \Phi_{(1)}(z, \theta), & & z < \tau \\[1ex]
        \Phi_{(0)}(z, \theta) = \Phi_{(1)}(z, \theta), & & z = \tau \\[1ex]
        \Phi_{(0)}(z, \theta) > \Phi_{(1)}(z, \theta), & & z > \tau \\
      \end{array} .
    \right.
  \end{equation*}
  Furthermore, we can infer that
  $F_{(0)}^{-1}(\theta) < \tau < F_{(1)}^{-1}(\theta)$.  Setting the loss
  functions equal for $z$ in this interval yields:
  \begin{align*}
    & \Phi_{(0)}(z, \theta) \stackrel{\mathit{set}}{=} \Phi_{(1)}(z, \theta) \\
    & \hspace{5mm} \Longleftrightarrow \hspace{5mm}
      \ind\!\! \left(z > F_{(0)}^{-1}(\theta)\right) \theta
      \left(z - F_{(0)}^{-1}(\theta)\right) +
      \ind\!\! \left(z \leq F_{(0)}^{-1}(\theta)\right) (1 - \theta)
      \left(F_{(0)}^{-1}(\theta) - z\right) \\
    & \hspace{25mm} =
      \ind\!\! \left(z > F_{(1)}^{-1}(\theta)\right) \theta
      \left(z - F_{(1)}^{-1}(\theta)\right) +
      \ind\!\! \left(z \leq F_{(1)}^{-1}(\theta)\right) (1 - \theta)
      \left(F_{(1)}^{-1}(\theta) - z\right) \\
    & \hspace{5mm} \Longleftrightarrow \hspace{5mm}
      \theta \left(z - F_{(0)}^{-1}(\theta)\right) =
      (1 - \theta) \left(F_{(1)}^{-1}(\theta) - z\right) \\
    & \hspace{5mm} \Longleftrightarrow \hspace{5mm}
      z = \theta\, F_{(0)}^{-1}(\theta) + (1 - \theta)\, F_{(1)}^{-1}(\theta).
  \end{align*}
  It can then be verified that $\Phi_{(0)}(z, \theta) < \Phi_{(1)}(z, \theta)$
  corresponds to classifying $z$ to $\Pi_{(0)}$ and that
  $\Phi_{(0)}(z, \theta) > \Phi_{(1)}(z, \theta)$ corresponds to classifying $z$
  to $\Pi_{(1)}$.  Combining these facts yields the desired result.
\end{proof}

\begin{proof}[Proof of Lemma \ref{lem:empirical-quantlev}.]=
  The proof is relegated to the supplementary materials.  The result is
  essentially a consequence of the fact that the problem can be cast as a linear
  programming problem where the extreme points of the feasible set are the
  values of the $x_i$.
\end{proof}

\begin{proof}[Proof of Lemma \ref{lem:decision-rule-time}.]
  Constructing the $\theta$-th quantile classifier requires merely finding the
  $\lceil \theta n_0 \rceil$-th largest value from the observations that were
  drawn from population $\Pi_0$, and the $\lceil \theta n_1 \rceil$-th largest
  value from the observations that were drawn from population $\Pi_1$, and then
  calculating the decision boundary based on Lemma \ref{lem:decision-boundary}.
  Finding the $k$-th largest value of a set is an $\bigO(n)$ operation, and the
  calculation to obtain the decision boundary using Lemma
  \ref{lem:decision-boundary} is an $\bigO(1)$ operation.
\end{proof}

\begin{proof}[Proof of Proposition \ref{lem:optimal-quantile-time}.]
  Sorting the data is an $\mathcal{O}(n \log n)$ operation.  The operations
  performed in line 2 of Algorithm \ref{alg:empirically-optimal-classifier}
  takes $\mathcal{O}(n)$ time.

  The outer loop beginning on line 3 of Algorithm
  \ref{alg:empirically-optimal-classifier} requires some number of iterations
  that is bounded from above by $n - 1$.  Within each iteration, calculating
  $F_V^{-1}$ and $F_W^{-1}$ and the interval $G_i$ are each constant time
  operations.

  The step in line 6 of Algorithm \ref{alg:empirically-optimal-classifier} is
  really a high-level view of a second loop.  This inner loop requires first
  finding the set of $x_i$'s with values in
  $\big[x_{\scriptscriptstyle\text{low}},\,
  x_{\scriptscriptstyle\text{high}}\big)$ which is an $\mathcal{O}(\log n)$
  operation.  The number of times that the inner loop is performed is determined
  by the number of points with values in the interval and is bounded from above
  by $n - 1$.  Calculating the classification rate the sub-interval in each step
  of the inner loop is a constant time operation, as is mapping the sub-interval
  back to the quantile levels space.

  So combining the worst-case bound for the outer and inner loops, we find that
  the total number of intervals for which we calculate the classification rate
  for has a worst-case bound on the order of $n^2$ intervals, and that each
  calculation is a constant time operation for a total cost of
  $\mathcal{O}(n^2)$ time.  The initial operations performed in lines 1-2 of
  Algorithm \ref{alg:empirically-optimal-classifier} have an aggregate cost with
  $\mathcal{O}(n \log n)$ time, so in total the algorithm runs in
  $\mathcal{O}(n^2)$ time.
\end{proof}

\begin{proof}[Proof of Theorem \ref{thm:multivariate-consistency}.]
  The upper bound on the difference between the component-wise quantile
  distances shown in \ref{eq:quantile-distance-ubnd} was established in the
  proof of Theorem 1 in Hennig and Viroli (2016).  For $z \in \mathbb{R}$ and
  $\theta, \theta^{\prime} \in (0, 1)$ then
  \begin{equation}
    \label{eq:quantile-distance-ubnd}
    \big| \Phi_{ij}(z, \theta) - \Phi_{ij}(z, \theta^{\prime}) \big|
    \leq |z|\, | \theta - \theta^{\prime} | +
    4 | F_{ij}^{-1}(\theta) - F_{ij}^{-1}(\theta^{\prime}) |,
    \hspace{5mm} i = 1, 2,~ j = 1, \dots, p
  \end{equation}
  Since $F_{ij}^{-1}$ is continuous by assumption, it follows that for arbitrary
  fixed $z$, $\Phi_{ij}(z, \theta)$ is continuous in $\theta$ for every
  $\{i, j\}$.  This in turn implies that for arbitrary fixed $\vec{z}$ then
  $\Lambda(\vec{z}, \vec{\theta}) = \alpha_0 + \sum_{j=1}^p \alpha_j
  \Lambda_j(z_j, \theta_j)$ is also continuous in $\vec{\theta}$, since
  $\Lambda$ is a linear combination of the $\Phi_{ij}(z_j, \theta_j)$'s.  Then
  we observe that
  \begin{align}
    \label{eq:multivariate-phi-continuity}
    \begin{split}
      \limtheta \Psi(\vec{\theta})
      & = \limtheta \left\{
        \pi_0 \int \ind\Big( \Lambda(\vec{z}, \vec{\theta}) > 0 \Big)\, dP_0(\vec{z}) +
        \pi_1 \int \ind\Big(\Lambda(\vec{z}, \vec{\theta}) \leq 0 \Big)\, dP_1(\vec{z})
      \right\} \\[1ex]
      & = \pi_0 \int \limtheta \ind\Big( \Lambda(\vec{z}, \vec{\theta}) > 0 \Big)\, dP_0(\vec{z})
      + \pi_1 \int \limtheta \ind\Big(\Lambda(\vec{z}, \vec{\theta}) \leq 0 \Big)\, dP_1(\vec{z})
      \\[1ex]
      & = \pi_0 \int \ind \Big( \limtheta \Lambda(\vec{z}, \vec{\theta}) > 0 \Big)\, dP_0(\vec{z})
      + \pi_1 \int \ind \Big( \limtheta \Lambda(\vec{z}, \vec{\theta}) \leq 0 \Big)\, dP_1(\vec{z})
      \\[1ex]
      & = \pi_0 \int \ind \Big( \Lambda(\vec{z}, \vec{\theta}^{*}) > 0 \Big)\, dP_0(\vec{z})
      + \pi_1 \int \ind \Big( \Lambda(\vec{z}, \vec{\theta}^{*}) \leq 0 \Big)\, dP_1(\vec{z})
      \\[1ex]
      & = \Psi(\vec{\theta}^{*}).
    \end{split}
  \end{align}
  The justification for bringing the limit inside of the integral is due to the
  dominated convergence theorem.  The justification for bringing the limit
  inside of the indicator function is that the indicator function is continuous
  everywhere except at 0, which by Assumption 2 occurs with probability 0, and
  hence does not change the value of the integral.  This result establishes that
  $\Psi$ is continuous in $\vec{\theta}$.

  It was established in Lemma \ref{lem:univariate-consistency} that
  $\hat{\theta}_{jn} \convp \tilde{\theta}_j$, so by Slutsky's theorem
  it follows that $\hat{\vec{\theta}}_n \convp \tilde{\vec{\theta}}$.
  Then by the result obtained in equation
  (\ref{eq:multivariate-phi-continuity}), a second application of Slutsky's
  theorem yields
  $\Psi(\hat{\vec{\theta}}_n) \convp \Psi(\tilde{\vec{\theta}})$.
\end{proof}

\begin{proof}[Proof of Proposition \ref{thm:cqc-runtime}.]
  Consider the operations performed inside the outer-level for loop spanning
  lines 1-10 in Algorithm \ref{alg:classifier}.  Splitting the data into two
  parts is an $\mathcal{O}(n)$ operation.  Next, consider the second-level for
  loop iterating over the features.  We saw in Lemma
  \ref{lem:optimal-quantile-time} that the decision rule for the empirically
  optimal quantile classifier for the $j$-th feature can be obtained in
  $\mathcal{O}(n^2)$ time.  Next, calculating $x_{ij}^{*}$ is a constant-time
  operation for an $\mathcal{O}(n)$ number of calculations, which in total is an
  $\mathcal{O}(n)$ operation.

  Next, we break down the steps required in line 9 of Algorithm
  \ref{alg:classifier} to select the linear combination coefficients via
  penalized logistic regression.  Using the coordinate descent algorithm
  proposed in \cite{friedman2007, friedman2010} yields an $\mathcal{O}(np)$ run
  time for a fixed choice of penalty parameter.  Thus for a grid of size $T$
  penalty parameters and performing $k$-fold cross-validation for a total of $K$
  folds yields a total run time bound of $\mathcal{O}(KTnp)$.

  Then, since the outer loop is performed a total of $L$ times, we obtain the
  following run time bound for a sequential implementation of Algorithm 2:=
  \begin{equation}
    \label{eq:classifier-runtime}
    \mathcal{O}\Big( L \Big[ n + p(n^2 + n) + KTnp \Big] \Big)
    = \mathcal{O}\Big( Lnp \big[ n + KT \big] \Big).
  \end{equation}
  Next, if we have $L$ compute nodes available, we can assign the calculation of
  each sub-model $f_i$ to one of the nodes; this reduces the problem to
  calculating the run time for each of the $f_i$.  If we perform the parallelism
  over first the features (i.e. lines 3-8 of Algorithm \ref{alg:classifier}),
  and then over the cross-validation folds (i.e. line 9 of Algorithm
  \ref{alg:classifier}), then the run time bound for an implementation of
  Algorithm 2 utilizing parallel computing is given by
  \begin{equation}
    \label{eq:classifier-runtime-parallel}
    \mathcal{O}\left( n + \frac{p}{C} (n^2 + n) + \frac{K}{C} Tnp \right) 
    = \mathcal{O}\Big( \frac{np}{C} \big[ n + KT \big] \Big).
  \end{equation}
  This calculation shows that the cost of training the algorithm has a
  polynomial-time complexity.
\end{proof}



%%% Local Variables:
%%% mode: latex
%%% TeX-master: "cqc_paper"
%%% End:
