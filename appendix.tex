
\section{Appendix}
\label{sec:appendix}

\begin{proof}[Proof of Lemma \ref{lem:univariate-consistency}.]
  The fact that
  $\Psi(\hat{\theta}_n) \stackrel{p}{\longrightarrow} \Psi(\tilde{\theta})$ is a
  special case of Theorem 1 in \cite{hennig2016} for a feature-space of
  dimension 1.  Furthermore, during the proof of that theorem it was shown that
  under Assumptions 1 and 2, $\Psi$ is a continuous function of $\theta$.  To
  show that $\hat{\theta}_n \stackrel{p}{\longrightarrow} \tilde{\theta}$
  suppose that the claim doesn't hold.  Then there exists an $\epsilon > 0$ and
  $\delta > 0$ such that for all $N \in \mathbb{N}$, there exists an $n \geq N$
  such that
  \begin{equation*}
    \prob\Big(
    \big| \hat{\theta}_n - \tilde{\theta} \big|
    > \epsilon \Big)
    \geq \delta.
  \end{equation*}
  Now, because $\Psi$ is continuous and $\Psi(\tilde{\theta})$ is a unique
  maximum, it follows that we can find $\nu > 0$ such that
  \begin{equation*}
    \min \left\{
      \big| \Psi(\Tilde{\theta} - \epsilon) - \Psi(\tilde{\theta}) \big|\,,
      \hspace{2mm}
      \big| \Psi(\Tilde{\theta} + \epsilon) - \Psi(\tilde{\theta}) \big|
    \right\} \geq \nu.
  \end{equation*}
  Therefore, for all $N \in \mathbb{N}$, there exists an $n \geq N$ such that
  \begin{equation*}
    \prob\Big(
    \big| \Psi(\hat{\theta}_n) - \Psi(\tilde{\theta}) \big|
    \geq \nu \Big) \geq
    \prob\Big(
    \big| \hat{\theta}_n - \tilde{\theta} \big|
    \geq \epsilon \Big)
    \geq \delta.
  \end{equation*}
  But this is in contradiction to the fact that
  $\Psi(\hat{\theta}_n) \stackrel{p}{\longrightarrow} \Psi(\tilde{\theta})$.
\end{proof}


\begin{proof}[Proof of Lemma \ref{lem:decision-boundary}.]
  Suppose $F_{(0)}^{-1}(\theta) \ne F_{(1)}^{-1}(\theta)$.  It is clear
  (e.g. see Figure \ref{fig:phi-lambda}) that $\Phi_{(0)}(z, \theta)$ is
  equal to $\Phi_{(1)}(z, \theta)$ at exactly one point, say $\tau$, and that
  the following holds:
  \begin{equation*}
    \left\{
      \begin{array}{lll}
        \Phi_{(0)}(z, \theta) < \Phi_{(1)}(z, \theta), & & z < \tau \\[1ex]
        \Phi_{(0)}(z, \theta) = \Phi_{(1)}(z, \theta), & & z = \tau \\[1ex]
        \Phi_{(0)}(z, \theta) > \Phi_{(1)}(z, \theta), & & z > \tau \\
      \end{array}
    \right.
  \end{equation*}
  Furthermore, we can infer that
  $F_{(0)}^{-1}(\theta) < \tau < F_{(1)}^{-1}(\theta)$.  Setting the loss
  functions equal for $z$ in this interval yields:
  \begin{align*}
    & \Phi_{(0)}(z, \theta) \stackrel{\mathit{set}}{=} \Phi_{(1)}(z, \theta) \\
    & \hspace{5mm} \Longleftrightarrow \hspace{5mm}
      \ind\!\! \left(z > F_{(0)}^{-1}(\theta)\right) \theta
      \left(z - F_{(0)}^{-1}(\theta)\right) +
      \ind\!\! \left(z \leq F_{(0)}^{-1}(\theta)\right) (1 - \theta)
      \left(F_{(0)}^{-1}(\theta) - z\right) \\
    & \hspace{25mm} =
      \ind\!\! \left(z > F_{(1)}^{-1}(\theta)\right) \theta
      \left(z - F_{(1)}^{-1}(\theta)\right) +
      \ind\!\! \left(z \leq F_{(1)}^{-1}(\theta)\right) (1 - \theta)
      \left(F_{(1)}^{-1}(\theta) - z\right) \\
    & \hspace{5mm} \Longleftrightarrow \hspace{5mm}
      \theta \left(z - F_{(0)}^{-1}(\theta)\right) =
      (1 - \theta) \left(F_{(1)}^{-1}(\theta) - z\right) \\
    & \hspace{5mm} \Longleftrightarrow \hspace{5mm}
      z = \theta\, F_{(0)}^{-1}(\theta) + (1 - \theta)\, F_{(1)}^{-1}(\theta).
  \end{align*}
  It can then be verified that $\Phi_{(0)}(z, \theta) < \Phi_{(1)}(z, \theta)$
  corresponds to classifying $z$ to $\Pi_{(0)}$ and that
  $\Phi_{(0)}(z, \theta) > \Phi_{(1)}(z, \theta)$ corresponds to classifying $z$
  to $\Pi_{(1)}$.  Combining these facts yields the desired result.
\end{proof}

\begin{proof}[Proof of Lemma \ref{lem:empirical-quantlev}.]
  % 
We can write the minimization problem
\begin{equation*}
  \min_q \left\{
    \theta \sum_{ x_{i} > q } |x_{i} - q| ~+~
    (1 - \theta) \sum_{ x_{i} \leq q } |x_{i} - q|
  \right\}
\end{equation*}
equivalently as
% \begin{align*}
%   &\min_{q^{+}, q^{-}, \vec{u}, \vec{v}} \left\{
%     \theta \sum_{i=1}^m u_i ~+~
%     (1 - \theta) \sum_{i=1}^m v_i
%   \right\} \\[2ex]
%   & \text{subject to} \hspace{3mm}
%   x_i - (q^{+} - q^{-}) = u_i - v_i, \hspace{5mm} i = 1, \dots, m \\[2ex]
%   & q^{+} \geq 0,~ q^{-} \geq 0,~ \vec{u} \geq \vec{0},~ \vec{v} \geq 0
% \end{align*}
\begin{equation*}
  \arraycolsep=5mm
  \begin{array}{ll}
    \displaystyle
    \minimize_{q^{+}, q^{-}, \vec{u}, \vec{v}}
    & \theta \sum_{i=1}^m u_i ~+~
      (1 - \theta) \sum_{i=1}^m v_i \\[2ex]
    \text{subject to}
    & x_i - (q^{+} - q^{-}) = u_i - v_i, \hspace{5mm} i = 1, \dots, m \\[2ex]
    & q^{+} \geq 0,~ q^{-} \geq 0,~ \vec{u} \geq \vec{0},~ \vec{v} \geq \vec{0} \\
  \end{array}
\end{equation*}
which is seen to be a linear programming problem in standard form.  By rewriting
the equality condition as $q^{+} - q^{-} + u_i - v_i = x_i$ for $i=1, \dots, m$,
we can express the equality condition in matrix form as
\begin{equation*}
  \begin{bmatrix}
    1      & -1     & 1 &        &   & -1 &        &    \\
    \vdots & \vdots &   & \ddots &   &    & \ddots &    \\
    1      & -1     &   &        & 1 &    &        & -1 \\
  \end{bmatrix}
  % \begin{bmatrix}
  %   q^{+} \\ q^{-} \\ u_{1} \\ \vdots \\ u_m \\ v_1 \\ \vdots \\ v_m \\
  % \end{bmatrix}
  \begin{bmatrix}
    q^{+} \\ q^{-} \\ \vec{u} \\ \vec{v} \\
  \end{bmatrix}
  =
  \begin{bmatrix}
    x_1 \\ \vdots \\ x_m \\
  \end{bmatrix}.
\end{equation*}
% \begin{equation*}
%   \begin{bmatrix}
%     1      & -1     & 1         &        &  \bigDown{0} & -1        &        & \bigDown{0} \\
%     \vdots & \vdots &           & \ddots &              &           & \ddots &             \\
%     1      & -1     & \bigUp{0} &        & 1            & \bigUp{0} &        & -1          \\
%   \end{bmatrix}
% \end{equation*}
% \begin{equation*}
%   \begin{bmatrix}
%     1      & -1     & &         &  & &          & \\
%     \vdots & \vdots & & \vec{I} &  & & -\vec{I} & \\
%     1      & -1     & &         &  & &          & \\
%   \end{bmatrix}
% \end{equation*}
% \begin{equation*}
%   \begin{bmatrix}
%     1      & -1     & 1      & 0      & \dots  & 0       & -1     &        &    \\
%     \vdots & \vdots & 0      & \ddots & \ddots & \vdots  &  0     & \ddots & \\
%     \vdots & \vdots & \vdots & \ddots & \ddots & 0       & \vdots & \ddots & \ddots &    \\
%     1      & -1     & 0      & \dots  & 0      & 1       &  0     & \dots  &  0     & -1 \\
%   \end{bmatrix}
% \end{equation*}
Recall that a solution for $(q^{+}, q^{-}, \vec{u}, \vec{v})$ is a basic
solution if and only if there exist indices $B(1), \dots, B(m)$ such that both
(i) the columns of the coefficient matrix with column indices subset by
$B(1), \dots, B(m)$ are linearly independent, and (ii) if an element of
$(q^{+}, q^{-}, \vec{u}, \vec{v})$ does not correspond to one of
$B(1), \dots, B(m)$ then the element must have a value of 0.

We can see that in order to have independent columns from the coefficient
matrix, then for each $i$, no more than one of the columns corresponding to
either $u_i$ or $v_i$ can have an index in $B(1), \dots, B(m)$, and additionally
no more than one of the columns corresponding to either $q^{+}$ or $q^{-}$ can
have an index in $B(1), \dots, B(m)$.  Furthermore, we note that if we have one
column corresponding to $q^{+}$ or $q^{-}$, and one column corresponding to
either $u_i$ or $v_i$ for each $i$, then we have $m + 1$ columns, which is still
one column too many.  Thus we see that there must be either (i) exactly one
column corresponding to either $u_i$ or $v_i$ for each $i$ and no columns
corresponding to either $q^{+}$ or $q^{-}$, or (ii) exactly one column
corresponding to either $u_i$ or $v_i$ for each $i$ less one and exactly one
column corresponding to either $q^{+}$ or $q^{-}$.

% Next we notice that if we include a column corresponding to $q^{+}$ or $q^{-}$,
% then the solution for the corresponding coefficient is either the value of $x_i$
% or $-x_i$, and where the index $i$ corresponds to the only $i$ without a column
% corresponding to either $u_i$ or $v_i$.  Then each

Furthermore, we can infer that if a column corresponding to $q^{+}$ or $q^{-}$
has an index in $B(1), \dots, B(m)$ and the solution is feasible (i.e. $q^{+}$
and $q^{-}$ are both nonnegative), then $q^{+} = (x_i)_{+}$ and
$q^{-} = (-x_i)_{+}$, where the index $i$ corresponds to the only $i$ without a
column corresponding to either $u_i$ or $v_i$, and $(z)_{+} = \max(0, z)$.  Let
$q = q^{+} - q^{-}$, then it follows that a feasible solution for any $j \ne i$
has $u_i = (x_i - q)_{+}$ and $v_i = (q - x_i)_{+}$.  This leads to a set of
basic feasible solutions given by $q \in \{x_1, \dots, x_m\}$.  One last basic
feasible solution is given for $q = 0$ with $u_i = (x_i)_{+}$ and
$v_i = (-x_i)_{+}$ for all $i$.

Next we aim to find the minimizing basic feasible solution.  Suppose that
$\lceil \theta m \rceil = k$ and that $\ell < k$, and let $q = x_k$ and
$q^{\prime} = x_{\ell}$.  Then comparing the objective function evaluated at $q$
and $q^{\prime}$, we have
\begin{align*} 
  \left\{ \theta \sum_{i=\ell+1}^m \right.
  & (x_i - q^{\prime}) +
    \left. (1 - \theta) \sum_{i=1}^{\ell} (q^{\prime} - x_i) \right\} -
    \left\{ \theta \sum_{i=k+1}^m (x_i - q) +
    (1 - \theta) \sum_{i=1}^k (q - x_i) \right\} \\[1ex]
  &= (1 - \theta) \sum_{i=1}^{\ell} \Big\{ (q^{\prime} - x_i) - (q - x_i) \Big\} \\
  &\hspace{8mm}
    + \theta \sum_{i=\ell+1}^k (x_i - q^{\prime})
    - (1 - \theta) \sum_{i=\ell+1}^k (q - x_i) \\
  &\hspace{8mm}
    + \theta \sum_{i=k+1}^m \Big\{ (x_i - q^{\prime})
    - (x_i - q) \Big\} \\[1ex]
  % &= (1 - \theta) \sum_{i=1}^{\ell} \Big\{ (q^{\prime} - x_i) - (q - x_i) \Big\} \\
  % &\hspace{8mm}
  %   + \theta \sum_{i=\ell+1}^k (x_i - q^{\prime})
  %   - (1 - \theta) \sum_{i=\ell+1}^k (q - x_i) \\
  % &\hspace{8mm}
  %   + (1 - \theta) \sum_{i=\ell+1}^k (q^{\prime} - x_i)
  %   - (1 - \theta) \sum_{i=\ell+1}^k (q^{\prime} - x_i) \\
  % &\hspace{8mm}
  %   + \theta \sum_{i=k+1}^m \Big\{ (x_i - q^{\prime})
  %   - (x_i - q) \Big\} \\[1ex]
  &= (1 - \theta) \sum_{i=1}^{\ell} \Big\{ (q^{\prime} - x_i) - (q - x_i) \Big\} \\
  &\hspace{8mm}
    + \theta \sum_{i=\ell+1}^k (x_i - q^{\prime})
    - (1 - \theta) \sum_{i=\ell+1}^k (q - x_i)
    \pm (1 - \theta) \sum_{i=\ell+1}^k (q^{\prime} - x_i) \\
  &\hspace{8mm}
    + \theta \sum_{i=k+1}^m \Big\{ (x_i - q^{\prime})
    - (x_i - q) \Big\} \\[1ex]
  &= -(1 - \theta) \sum_{i=1}^{\ell}\, (q - q^{\prime}) \\
  &\hspace{8mm}
    + \theta \sum_{i=\ell+1}^k (x_i - q^{\prime})
    - (1 - \theta) \sum_{i=\ell+1}^k (q - q^{\prime})
    - (1 - \theta) \sum_{i=\ell+1}^k (q^{\prime} - x_i) \\
  &\hspace{8mm}
    + \theta \sum_{i=k+1}^m (q - q^{\prime}) \\[1ex]
  &= -(1 - \theta) \sum_{i=1}^k\, (q - q^{\prime}) \\
  &\hspace{8mm}
    + \theta \sum_{i=\ell+1}^k (x_i - q^{\prime})
    - (1 - \theta) \sum_{i=\ell+1}^k (q^{\prime} - x_i) \\
  &\hspace{8mm}
    + \theta \sum_{i=k+1}^m (q - q^{\prime}) \\[1ex]
  &= -(1 - \theta) \sum_{i=1}^k\, (q - q^{\prime}) \\
  &\hspace{8mm}
    + \sum_{i=\ell+1}^k (x_i - q^{\prime}) \\
  &\hspace{8mm}
    + \theta \sum_{i=k+1}^m (q - q^{\prime}) \\[1ex]
  & = - (1 - \theta)\, k\, (q - q^{\prime})
    + \sum_{i=\ell+1}^k (x_i - q^{\prime})
    + \theta (m - k) (q - q^{\prime}) \\
  &= \sum_{i=\ell+1}^k (x_i - q^{\prime}) - (k - \theta m) (q - q^{\prime}) \\
  &\geq (x_k - q^{\prime}) - (q - q^{\prime}) \\
  &= x_k - q \\
  &= 0
\end{align*}
The inequality is due to the fact that $x_i \geq q^{\prime}$ for
$i \geq \ell + 1$, and also the fact that $k - \theta m < 1$.  Suppose now that
$\lceil \theta m \rceil = k$ and that $\ell > k$, and let $q = x_k$ and
$q^{\prime} = x_{\ell}$.  Then comparing the objective function evaluated at $q$
and $q^{\prime}$, we have
\begin{align*} 
  \left\{ \theta \sum_{i=\ell+1}^m \right.
  & (x_i - q^{\prime}) +
    \left. (1 - \theta) \sum_{i=1}^{\ell} (q^{\prime} - x_i) \right\} -
    \left\{ \theta \sum_{i=k+1}^m (x_i - q) +
    (1 - \theta) \sum_{i=1}^k (q - x_i) \right\} \\[1ex]
  &= (1 - \theta) \sum_{i=1}^k \Big\{ (q^{\prime} - x_i) - (q - x_i) \Big\} \\
  &\hspace{8mm}
    + \theta \sum_{i=k+1}^{\ell} (x_i - q^{\prime})
    - (1 - \theta) \sum_{i=k+1}^{\ell} (q - x_i) \\
  &\hspace{8mm}
    + \theta \sum_{i=\ell+1}^m \Big\{ (x_i - q^{\prime})
    - (x_i - q) \Big\} \\[1ex]
  &= (1 - \theta) \sum_{i=1}^k\, (q^{\prime} - q) \\
  &\hspace{8mm}
    - \theta \sum_{i=k+1}^{\ell} (q^{\prime} - q)
    + \sum_{i=k+1}^{\ell} (x_i - q) \\
  &\hspace{8mm}
    - \theta \sum_{i=l+1}^m (q^{\prime} - q) \\[1ex]
  % &= (1 - \theta) \sum_{i=1}^k\, (q^{\prime} - q) \\
  % &\hspace{8mm}
  %   + \sum_{i=k+1}^{\ell} (x_i - q) \\
  % &\hspace{8mm}
  %   - \theta \sum_{i=k+1}^m (q^{\prime} - q) \\[1ex]
  &= (1 - \theta) \sum_{i=1}^k\, (q^{\prime} - q)
    + \sum_{i=k+1}^{\ell} (x_i - q)
    - \theta \sum_{i=k+1}^m (q^{\prime} - q) \\
  &= (1 - \theta)\, k\, (q^{\prime} - q)
    + \sum_{i=k+1}^{\ell} (x_i - q)
    - \theta\, (m - k) (q^{\prime} - q) \\
  &= (k - \theta m) (q^{\prime} - q) + \sum_{i=k+1}^{\ell} (x_i - q) \\
  &\geq 0
\end{align*}
There is one basic feasible solution remaining to check, that where $q = 0$.  To
show that this is not the optimal solution except in the case that
$x_{\lceil \theta m \rceil} = 0$, we make the following argument.  Note that if
we add some nonzero value $\tau$ to each $x_i$ and if $q^{*}$ is an optimal
choice for the original problem, then $q^{*} + \tau$ is an optimal choice for
the new problem.  Choose $-\tau$ to be one of the $x_i$ for some $i$: then 0 is
one of the values in the transformed data, which was shown in the previous
results to be no better than the $\lceil \theta m \rceil$-th largest value of
the transformed data, so it follows that the $\lceil \theta m \rceil$-th largest
value is optimal.  Since $\tau$ cannot be better than optimal for the transormed
data, then by the law of the contrapositive 0 cannot be better than optimal for
the original data.




%%% Local Variables:
%%% mode: latex
%%% TeX-master: "cqc_paper"
%%% End:

  The proof is relegated to the supplementary materials.  The result is
  essentially a consequence of the fact that the problem can be cast as a linear
  programming problem where the extreme points of the feasible set are the
  values of the $x_i$.
\end{proof}

\begin{proof}[Proof of Lemma \ref{lem:decision-rule-time}.]
  Constructing the $\theta$-th quantile classifier requires merely finding the
  $\lceil \theta n_0 \rceil$-th largest value from the observations that were
  drawn from population $\Pi_0$, and the $\lceil \theta n_1 \rceil$-th largest
  value from the observations that were drawn from population $\Pi_1$, and then
  calculating the decision boundary based on Lemma \ref{lem:decision-boundary}.
  Finding the $k$-th largest value of a set is an $\bigO(n)$ operation, and the
  calculation to obtain the decision boundary using Lemma
  \ref{lem:decision-boundary} is an $\bigO(1)$ operation.
\end{proof}

\begin{proof}[Proof of Lemma \ref{lem:optimal-quantile-time}.]
  Sorting the data is an $\mathcal{O}(n \log n)$ operation.  The operations
  performed in line 2 of Algorithm \ref{alg:empirically-optimal-classifier}
  takes $\mathcal{O}(n)$ time.

  The outer loop beginning on line 3 of Algorithm
  \ref{alg:empirically-optimal-classifier} requires some number of iterations
  that is bounded from above by $n - 1$.  Within each iteration, calculating
  $F_V^{-1}$ and $F_W^{-1}$ and the interval $G_i$ are each constant time
  operations.

  The step in line 6 of Algorithm \ref{alg:empirically-optimal-classifier} is
  really a high-level view of a second loop.  This inner loop requires first
  finding the set of $x_i$'s with values in
  $\big[x_{\scriptscriptstyle\text{low}},\,
  x_{\scriptscriptstyle\text{high}}\big)$ which is an $\mathcal{O}(\log n)$
  operation.  The number of times that the inner loop is performed is determined
  by the number of points with values in the interval and is bounded from above
  by $n - 1$.  Calculating the classification rate the sub-interval in each step
  of the inner loop is a constant time operation, as is mapping the sub-interval
  back to the quantile levels space.

  So combining the worst-case bound for the outer and inner loops, we find that
  the total number of intervals for which we calculate the classification rate
  for has a worst-case bound on the order of $n^2$ intervals, and that each
  calculation is a constant time operation for a total cost of
  $\mathcal{O}(n^2)$ time.  The initial operations performed in lines 1-2 of
  Algorithm \ref{alg:empirically-optimal-classifier} have an aggregate cost with
  $\mathcal{O}(n \log n)$ time, so in total the algorithm runs in
  $\mathcal{O}(n^2)$ time.
\end{proof}

\begin{proof}[Proof of Theorem \ref{thm:multivariate-consistency}.]
  The following upper bound on the difference between the component-wise
  quantile distances was established in the proof of Theorem 1 in Hennig and
  Viroli (2016).  For $z \in \mathbb{R}$ and
  $\theta, \theta^{\prime} \in (0, 1)$ then
  \begin{equation}
    \label{eq:quantile-distance-ubnd}
    \big| \Phi_{ij}(z, \theta) - \Phi_{ij}(z, \theta^{\prime}) \big|
    \leq |z|\, | \theta - \theta^{\prime} | +
    4 | F_{ij}^{-1}(\theta) - F_{ij}^{-1}(\theta^{\prime}) |,
    \hspace{5mm} i = 1, 2,~ j = 1, \dots, p
  \end{equation}
  Since $F_{ij}^{-1}$ is continuous by assumption, it follows that for arbitrary
  fixed $z$, $\Phi_{ij}(z, \theta)$ is continuous in $\theta$ for every
  $\{i, j\}$.  This in turn implies that for arbitrary fixed $\vec{z}$,
  $\Lambda(\vec{z}, \vec{\theta})$ is also continuous in $\vec{\theta}$, since
  $\Lambda$ is just the sum of the differences of the
  $\Phi_{ij}(z_j, \theta_j)$'s.  Then we observe that
  \begin{align}
    \label{eq:multivariate-phi-continuity}
    \begin{split}
      \limtheta \Psi(\vec{\theta})
      & = \limtheta \left\{
        \pi_0 \int \ind\Big( \Lambda(\vec{z}, \vec{\theta}) > 0 \Big)\, dP_0(\vec{z}) +
        \pi_1 \int \ind\Big(\Lambda(\vec{z}, \vec{\theta}) \leq 0 \Big)\, dP_1(\vec{z})
      \right\} \\[1ex]
      & = \pi_0 \int \limtheta \ind\Big( \Lambda(\vec{z}, \vec{\theta}) > 0 \Big)\, dP_0(\vec{z})
      + \pi_1 \int \limtheta \ind\Big(\Lambda(\vec{z}, \vec{\theta}) \leq 0 \Big)\, dP_1(\vec{z})
      \\[1ex]
      & = \pi_0 \int \ind \Big( \limtheta \Lambda(\vec{z}, \vec{\theta}) > 0 \Big)\, dP_0(\vec{z})
      + \pi_1 \int \ind \Big( \limtheta \Lambda(\vec{z}, \vec{\theta}) \leq 0 \Big)\, dP_1(\vec{z})
      \\[1ex]
      & = \pi_0 \int \ind \Big( \Lambda(\vec{z}, \vec{\theta}^{*}) > 0 \Big)\, dP_0(\vec{z})
      + \pi_1 \int \ind \Big( \Lambda(\vec{z}, \vec{\theta}^{*}) \leq 0 \Big)\, dP_1(\vec{z})
      \\[1ex]
      & = \Psi(\vec{\theta}^{*})
    \end{split}
  \end{align}
  The justification for bringing the limit inside of the integral is due to the
  dominated convergence theorem.  The justification for bringing the limit
  inside of the indicator function is that the indicator function is continuous
  everywhere except at 0, which by Assumption 2 occurs with probability 0, and
  hence does not change the value of the integral.  This result establishes that
  $\Psi$ is continuous in $\vec{\theta}$.

  It was established in Lemma \ref{lem:univariate-consistency} that
  $\hat{\theta}_{jn} \convp \tilde{\theta}_j$, so by Slutsky's theorem
  it follows that $\hat{\vec{\theta}}_n \convp \tilde{\vec{\theta}}$.
  Then by the result obtained in equation
  (\ref{eq:multivariate-phi-continuity}), a second application of Slutsky's
  theorem yields
  $\Psi(\hat{\vec{\theta}}_n) \convp \Psi(\tilde{\vec{\theta}})$.
\end{proof}

\begin{proof}[Proof of Proposition \ref{thm:cqc-runtime}.]
  Consider the operations performed inside the outer-level for loop spanning
  lines 1-10 in Algorithm \ref{alg:classifier}.  Splitting the data into two
  parts is an $\mathcal{O}(n)$ operation.  Next, consider the second-level for
  loop iterating over the features.  We saw in Lemma
  \ref{lem:optimal-quantile-time} that the decision rule for the empirically
  optimal quantile classifier for the $j$-th feature can be obtained in
  $\mathcal{O}(n^2)$ time.  Next, calculating $x_{ij}^{*}$ is a constant-time
  operation for an $\mathcal{O}(n)$ number of calculations, which in total is an
  $\mathcal{O}(n)$ operation.

  Next, we break down the steps required in line 9 of Algorithm
  \ref{alg:classifier} to select the linear combination coefficients via
  penalized logistic regression.  Using the coordinate descent algorithm
  proposed in \cite{friedman2007, friedman2010} yields an $\mathcal{O}(np)$ run
  time for a fixed choice of penalty parameter.  Thus for a grid of size $T$
  penalty parameters and performing $k$-fold cross-validation for a total of $K$
  folds yields a total run time bound of $\mathcal{O}(KTnp)$.

  Then, since the outer loop is performed a total of $L$ times, we obtain the
  following run time bound:
  % \begin{align}
      %       \label{eq:classifier-runtime}
      %       \begin{split} \MoveEqLeft
      %       \mathcal{O}\Big( L \Big[ n + p(n^2 + n) + KTnp \Big] \Big) \\
      %    &= \mathcal{O}\Big( L \Big[ n + n^2 p + np + KTnp \Big] \Big) \\
      %            &= \mathcal{O}\Big( L \Big[ n^2 p + KTnp \Big] \Big) \\
      %            &= \mathcal{O}\Big( Lnp \Big[ n + KT \Big] \Big).
                     %                      \end{split}
                     %     \end{align}
  \begin{equation}
    \label{eq:classifier-runtime}
    \mathcal{O}\Big( L \Big[ n + p(n^2 + n) + KTnp \Big] \Big)
    = \mathcal{O}\Big( Lnp \Big[ n + KT \Big] \Big).
  \end{equation}
  Furthermore, if we consider $L$ and $T$ to be constants then the bound further
  simplifies to $\mathcal{O}(n^2 p)$.
\end{proof}

\begin{proof}[Proof of Proposition \ref{cor:parallel-runtime}.]
  Since by assumption we have no fewer than $L$ compute nodes available, we can
  assign the calculation of each sub-model $f_i$ to one of the nodes; this
  reduces the problem to calculating the run time for each of the $f_i$.  If we
  perform the parallelism over first the features (i.e. lines 3-8 of Algorithm
  \ref{alg:classifier}), and then over the cross-validation folds (i.e. line 9
  of Algorithm \ref{alg:classifier}), then the run time bound is as follows:
  \begin{equation}
    \label{eq:classifier-runtime-parallel}
    \mathcal{O}\left( n + \frac{p}{C} (n^2 + n) + \frac{K}{C} Tnp \right) 
    = \mathcal{O}\Big( \frac{np}{C} (n + KT) \Big).
  \end{equation}
  Then by replacing $C$ with $n/T$ under the assumption that $CT \geq n$ we
  obtain the expression in (\ref{eq:parallel-runtime-ass2}).  We further note
  that since a penalized logistic regression model has an $\mathcal{O}(npT)$
  complexity to fit, it follows that when adequate computing resources are
  available calculating the composite quantile-based classifier model is of
  similar complexity as a (non-parallelized) penalized logistic regression model
  fit.
\end{proof}



%%% Local Variables:
%%% mode: latex
%%% TeX-master: "cqc_paper"
%%% End:
