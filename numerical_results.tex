

\section{Numerical results}
\label{sec:numerical-results}

\subsection{Real data considerations}

In practice, one often encounters data that has variables with both continuous
as well as categorical data.  In problems such as these categorical data is
typically reformulated into so-called dummy-coded variables.  However, the
quantile-based classifier is inherently ill-conditioned to handle such data.
When categorical variables are present in the data, we propose including them in
the classifier as untransformed dummy-coded variables.

A second type of data can also cause problems:  one where the data is a mixture
of point masses as well as continuous data.  As an example, consider univariate
data where the the quantiles of the underlying populations are given as follows.
\begin{center}
  \begin{tabular}{lrrrrrrrrrrr}
    \toprule
    & \multicolumn{11}{c}{\underline{Quantile levels}} \\
    & 0.89 & 0.90 & 0.91 & 0.92 & 0.93 & 0.94 & 0.95 & 0.96 & 0.97 & 0.98 & 0.99 \\
    \midrule
    Class 0 & 0.00 & 0.02 & 0.14 & 0.23 & 0.33 & 0.42 & 0.55 & 0.68 & 0.87 & 1.05 & 1.41 \\
    Class 1 & 0.00 & 0.00 & 0.00 & 0.00 & 0.00 & 0.00 & 0.00 & 0.06 & 0.08 & 0.18 & 0.20 \\
    \bottomrule
  \end{tabular}
\end{center}
This example raises the question as to what the quantile distances would be for
a fixed choice of quantile level.  Suppose first that we consider a quantile
level with nonzero quantiles for both classes, say 0.97.  In this case, the
quantile distances for the data in class 0 with value 0 than the data quantile
distances for the data in class 1 with value 0.  As a result, the quantile-based
data will have spuriously introduced differences in the data for 90\% of the
data, when in fact there is no discriminative data in those quantiles.
Conversely, suppose now that we select a quantile level with quantiles for both
classes with value 0.  Doing so has the effect of simply multiplying the nonzero
values in the data by a constant factor.  While this is less disastrous then the
previous scenario, we are not using the quantile-based data approach consistent
with the rest of the paper.

What we propose to do for such a scenario is the following.  When mixture data
such as the example given above is present for a variable, we try to separate
the parts of the data that belong to the point mass contribution to the mixture,
and those that belong to the continuous part of the mixture.  Then for the point
mass data, we create a categorical variable that includes a reference category
for the continuous part of the data, and for the continuous part we perform the
quantile-based classification using only the reduced subset of the data.




\subsection{Simulation study}
\label{sec: simulation}

The performance of the marginal quantile-based classifier is compared to other
classification methods through the use of simulation studies.  A number of
scenarios are considered as follows.  In the first scenario, we consider two
classes, each with features drawn from a multivariate Gaussian distribution,
with one class given a location shift.  In the second scenario, we consider
highly skewed data by sampling from a multivariate Gaussian distribution, and
then exponentiating the data, with one class given a location shift.  A third
scenario is considered, with features drawn from a multivariate Gaussian
distribution, and then blocks of the data are transformed through
exponentiation, log transforms, squaring, and square root transforms.  Finally,
features were was sampled from beta distributions and gamma distributions with
different parameterizations for each class.

The methods proposed in this paper are abbreviated as MQC / Aggregate and MQC /
PLR to signify whether the decision rule was obtained using varying coefficients
or not.  Additionally, a third variant abbreviated as MQC / PLR + Aug is
considered (short for augmented), which means that the quantile distances data
is augmented by the original data when constructing the decision rule.

In the Gaussian setting, nearest shrunken centroids and penalized LDA show the
best performance.  When $n$ is large compared to $p$, marginal quantile-based
classifiers perform reasonably well, although performance deteriorates relative
to other methods when conditions are less ideal.

In the exponentiated Gaussian setting, Hennig and Viroli's quantile-based
classifiers perform far better than any other competitor in every setting.  This
is the setting where this method is really ideal, in that the optimal quantile
level is the same for every feature.  This allows the method to in some sense,
``borrow'' information across the features when selecting the empirically
optimal quantile.  The aggregate MQC method will perform equally as well as the
quantile-based classifiers in the limit, but it is clear that for finite samples
the performance is not nearly as good.  The MQC / PLR methods do well compared
to other methods, although still far behind the quantile-based classifiers.

The MQC / PLR approach did well for the block transformed data simulations and
was the best method for many settings.  Interestingly, decision trees was a
close second for many settings.  Quantile-based classifiers was another
effective method that performed the best when $n$ was small compared to $p$.

For the beta and gamma distributed data, MQC, quantile-based classifiers and,
decision trees exhibited the best performance.  We expect that this is due to a
few features having a high level of discriminatory information, and each of
these methods being able to make good use of these features for classification.




\subsection{Real data study}
\label{sec:real-data-study}

For a real data study, we considered a spam data with a total of 4,601
observations and 57 features.  The attributes are, for example, the percentage
of specific words or phrases in the email, the average and maximum run lengths
of uppercase letters, and the total number of uppercase letters.  The proportion
of data used to train the methods is varied from 5\% to 80\%.  We see that for
this data set, MQC performs at least as well as the best at every level, with
SVM the next best competitor.



% Gaussian data ----------------------------------------------------------------

\begin{table}[p]
  \centering
  Gaussian data \\
  Correlation coefficient = 0 \\[2ex]
  \begin{adjustbox}{max size={\textwidth}{10cm}}
    \begin{tabular}{l@{\extracolsep{15mm}}rrr}
      
      \hline
      & $n=500$ & $n=250$ & $n=50$ \\ 
      \hline
      & $p = 50$ \\
      \hline

      CQC & 0.160 (0.01) & 0.198 (0.02) & 0.286 (0.03) \\ 
      CQC augmented & 0.138 (0.01) & 0.161 (0.02) & 0.263 (0.05) \\
      Quantile-based classifiers & 0.164 (0.01) & 0.186 (0.02) & 0.282 (0.04) \\
      FANS  & 0.189 (0.02) & 0.254 (0.03) & 0.430 (0.04) \\
      FANS2 & 0.141 (0.01) & 0.176 (0.02) & 0.379 (0.07) \\
      Penalized logistic regression & 0.131 (0.01) & 0.161 (0.01) & 0.274 (0.05) \\ 
      Support vector machine & 0.118 (0.01) & 0.129 (0.01) & 0.165 (0.02) \\ 
      k-nearest neighbor & 0.193 (0.01) & 0.214 (0.02) & 0.276 (0.04) \\ 
      Naive Bayes & 0.122 (0.01) & 0.139 (0.01) & 0.213 (0.03) \\ 
      Nearest shrunken centroids & 0.114 (0.01) & 0.126 (0.01) & 0.170 (0.04) \\ 
      Penalized LDA & 0.116 (0.01) & 0.126 (0.01) & 0.161 (0.02) \\ 
      Decision trees & 0.380 (0.01) & 0.388 (0.02) & 0.435 (0.02) \\ [2ex]

      \hline
      & $p = 250$ \\
      \hline

      CQC & 0.196 (0.02) & 0.243 (0.02) & 0.391 (0.03) \\ 
      CQC augmented & 0.158 (0.01) & 0.199 (0.02) & 0.355 (0.03) \\ 
      Quantile-based classifiers & 0.169 (0.03) & 0.179 (0.02) & 0.325 (0.06) \\ 
      FANS  & 0.226 (0.01) & 0.313 (0.02) & 0.464 (0.04) \\
      FANS2 & 0.170 (0.02) & 0.229 (0.03) & 0.459 (0.04) \\
      Penalized logistic regression & 0.151 (0.01) & 0.206 (0.02) & 0.343 (0.04) \\ 
      Support vector machine & 0.143 (0.01) & 0.182 (0.02) & 0.272 (0.02) \\ 
      k-nearest neighbor & 0.190 (0.02) & 0.210 (0.02) & 0.254 (0.02) \\ 
      Naive Bayes & 0.165 (0.02) & 0.204 (0.01) & 0.332 (0.02) \\ 
      Nearest shrunken centroids & 0.120 (0.01) & 0.128 (0.01) & 0.189 (0.03) \\ 
      Penalized LDA & 0.136 (0.01) & 0.167 (0.01) & 0.289 (0.08) \\ 
      Decision trees & 0.397 (0.02) & 0.439 (0.03) & 0.456 (0.02) \\ [2ex]

      \hline
      & $p = 500$ \\
      \hline

      CQC & 0.215 (0.02) & 0.280 (0.02) & 0.423 (0.03) \\ 
      CQC augmented & 0.175 (0.01) & 0.222 (0.02) & 0.405 (0.04) \\  
      Quantile-based classifiers & 0.176 (0.01) & 0.190 (0.03) & 0.343 (0.05) \\ 
      FANS  & 0.243 (0.02) & 0.358 (0.03) & 0.498 (0.02) \\
      FANS2 & 0.192 (0.01) & 0.275 (0.03) & 0.473 (0.04) \\
      Penalized logistic regression & 0.168 (0.01) & 0.224 (0.02) & 0.411 (0.04) \\ 
      Support vector machine & 0.186 (0.02) & 0.222 (0.01) & 0.340 (0.03) \\ 
      k-nearest neighbor & 0.191 (0.02) & 0.218 (0.02) & 0.259 (0.04) \\ 
      Naive Bayes & 0.213 (0.01) & 0.260 (0.02) & 0.379 (0.02) \\ 
      Nearest shrunken centroids & 0.123 (0.01) & 0.134 (0.01) & 0.209 (0.03) \\ 
      Penalized LDA & 0.171 (0.01) & 0.212 (0.02) & 0.333 (0.03) \\ 
      Decision trees & 0.426 (0.03) & 0.425 (0.02) & 0.473 (0.04) \\ 

      \hline
      
    \end{tabular}
  \end{adjustbox}
\end{table}




\begin{table}[p]
  \centering
  Gaussian data \\
  Correlation coefficient = 0.8 \\[2ex]
  \begin{adjustbox}{max size={\textwidth}{10cm}}
    \begin{tabular}{l@{\extracolsep{15mm}}rrr}
      
      \hline
      & $n=500$ & $n=250$ & $n=50$ \\ 
      \hline
      & $p = 50$ \\
      \hline

      CQC & 0.358 (0.01) & 0.361 (0.02) & 0.402 (0.02) \\ 
      CQC augmented & 0.349 (0.01) & 0.353 (0.02) & 0.391 (0.02) \\ 
      Quantile-based classifiers & 0.349 (0.01) & 0.359 (0.01) & 0.414 (0.04) \\ 
      FANS  & 0.362 (0.01) & 0.352 (0.01) & 0.446 (0.04) \\
      FANS2 & 0.352 (0.01) & 0.356 (0.02) & 0.439 (0.05) \\
      Penalized logistic regression & 0.354 (0.01) & 0.360 (0.02) & 0.396 (0.03) \\ 
      Support vector machine & 0.344 (0.01) & 0.349 (0.02) & 0.389 (0.03) \\ 
      k-nearest neighbor & 0.387 (0.01) & 0.401 (0.04) & 0.429 (0.02) \\ 
      Naive Bayes & 0.339 (0.01) & 0.339 (0.01) & 0.375 (0.03) \\ 
      Nearest shrunken centroids & 0.340 (0.02) & 0.341 (0.01) & 0.381 (0.03) \\ 
      Penalized LDA & 0.337 (0.01) & 0.336 (0.02) & 0.376 (0.05) \\ 
      Decision trees & 0.403 (0.01) & 0.408 (0.02) & 0.433 (0.03) \\ [2ex]

      \hline
      & $p = 250$ \\
      \hline

      CQC & 0.370 (0.01) & 0.380 (0.02) & 0.437 (0.02) \\ 
      CQC augmented & 0.368 (0.01) & 0.379 (0.02) & 0.429 (0.02) \\ 
      Quantile-based classifiers & 0.362 (0.02) & 0.394 (0.03) & 0.428 (0.03) \\ 
      FANS  & 0.375 (0.01) & 0.409 (0.02) & 0.496 (0.01) \\
      FANS2 & 0.360 (0.01) & 0.398 (0.02) & 0.474 (0.03) \\
      Penalized logistic regression & 0.369 (0.01) & 0.395 (0.02) & 0.446 (0.02) \\ 
      Support vector machine & 0.349 (0.02) & 0.370 (0.02) & 0.403 (0.03) \\ 
      k-nearest neighbor & 0.384 (0.01) & 0.392 (0.02) & 0.430 (0.04) \\ 
      Naive Bayes & 0.342 (0.02) & 0.349 (0.01) & 0.401 (0.02) \\ 
      Nearest shrunken centroids & 0.336 (0.01) & 0.344 (0.01) & 0.379 (0.02) \\ 
      Penalized LDA & 0.337 (0.01) & 0.344 (0.01) & 0.374 (0.02) \\ 
      Decision trees & 0.438 (0.03) & 0.447 (0.01) & 0.459 (0.02) \\ [2ex]

      \hline
      & $p = 500$ \\
      \hline

      CQC & 0.369 (0.02) & 0.394 (0.02) & 0.427 (0.03) \\ 
      CQC augmented & 0.362 (0.01) & 0.388 (0.02) & 0.442 (0.04) \\ 
      Quantile-based classifiers & 0.395 (0.01) & 0.402 (0.03) & 0.402 (0.02) \\ 
      FANS  & 0.375 (0.02) & 0.428 (0.03) & 0.483 (0.03) \\
      FANS2 & 0.368 (0.02) & 0.411 (0.04) & 0.468 (0.04) \\
      Penalized logistic regression & 0.360 (0.02) & 0.387 (0.03) & 0.441 (0.04) \\ 
      Support vector machine & 0.349 (0.01) & 0.365 (0.03) & 0.385 (0.03) \\ 
      k-nearest neighbor & 0.386 (0.03) & 0.386 (0.04) & 0.403 (0.04) \\ 
      Naive Bayes & 0.338 (0.01) & 0.360 (0.02) & 0.391 (0.03) \\ 
      Nearest shrunken centroids & 0.331 (0.01) & 0.349 (0.04) & 0.349 (0.03) \\ 
      Penalized LDA & 0.336 (0.01) & 0.346 (0.02) & 0.379 (0.05) \\ 
      Decision trees & 0.434 (0.02) & 0.444 (0.03) & 0.480 (0.03) \\

      \hline
      
    \end{tabular}
  \end{adjustbox}
\end{table}




\begin{table}[p]
  \centering
  Exponentiated Gaussian data \\
  Correlation coefficient = 0 \\[2ex]
  \begin{adjustbox}{max size={\textwidth}{10cm}}
    \begin{tabular}{l@{\extracolsep{15mm}}rrr}
      
      \hline
      & $n=500$ & $n=250$ & $n=50$ \\ 
      \hline
      & $p = 50$ \\
      \hline

      CQC & 0.012 (0.00) & 0.030 (0.01) & 0.174 (0.04) \\ 
      CQC augmented & 0.012 (0.00) & 0.031 (0.01) & 0.206 (0.03) \\ 
      Quantile-based classifiers & 0.000 (0.00) & 0.001 (0.00) & 0.112 (0.05) \\ 
      FANS  & 0.039 (0.01) & 0.097 (0.02) & 0.372 (0.05) \\
      FANS2 & 0.042 (0.01) & 0.115 (0.03) & 0.384 (0.03) \\
      Penalized logistic regression & 0.293 (0.02) & 0.316 (0.02) & 0.399 (0.03) \\ 
      Support vector machine & 0.223 (0.02) & 0.252 (0.02) & 0.354 (0.03) \\ 
      k-nearest neighbor & 0.376 (0.02) & 0.386 (0.03) & 0.424 (0.03) \\ 
      Naive Bayes & 0.433 (0.01) & 0.439 (0.02) & 0.462 (0.02) \\ 
      Nearest shrunken centroids & 0.283 (0.01) & 0.313 (0.03) & 0.384 (0.04) \\ 
      Penalized LDA & 0.284 (0.01) & 0.308 (0.02) & 0.352 (0.02) \\ 
      Decision trees & 0.114 (0.01) & 0.191 (0.06) & 0.327 (0.02) \\ [2ex]

      \hline
      & $p = 250$ \\
      \hline

      CQC & 0.017 (0.00) & 0.031 (0.01) & 0.198 (0.03) \\ 
      CQC augmented & 0.016 (0.00) & 0.038 (0.01) & 0.222 (0.04) \\ 
      Quantile-based classifiers & 0.001 (0.00) & 0.004 (0.00) & 0.135 (0.04) \\ 
      FANS  & 0.050 (0.01) & 0.123 (0.02) & 0.459 (0.04) \\
      FANS2 & 0.055 (0.01) & 0.132 (0.02) & 0.462 (0.04) \\
      Penalized logistic regression & 0.359 (0.02) & 0.394 (0.02) & 0.461 (0.02) \\ 
      Support vector machine & 0.339 (0.01) & 0.378 (0.02) & 0.429 (0.03) \\ 
      k-nearest neighbor & 0.383 (0.04) & 0.392 (0.03) & 0.428 (0.03) \\ 
      Naive Bayes & 0.440 (0.02) & 0.455 (0.02) & 0.471 (0.01) \\ 
      Nearest shrunken centroids & 0.309 (0.02) & 0.327 (0.01) & 0.404 (0.05) \\ 
      Penalized LDA & 0.348 (0.01) & 0.385 (0.02) & 0.439 (0.04) \\ 
      Decision trees & 0.128 (0.02) & 0.160 (0.02) & 0.375 (0.07) \\ [2ex]

      \hline
      & $p = 500$ \\
      \hline

      CQC & 0.016 (0.00) & 0.036 (0.01) & 0.225 (0.03) \\ 
      CQC augmented & 0.017 (0.01) & 0.043 (0.01) & 0.292 (0.03) \\ 
      Quantile-based classifiers & 0.001 (0.00) & 0.005 (0.00) & 0.148 (0.05) \\ 
      FANS  & 0.051 (0.01) & 0.149 (0.02) & 0.453 (0.04) \\
      FANS2 & 0.057 (0.01) & 0.162 (0.02) & 0.477 (0.04) \\
      Penalized logistic regression & 0.374 (0.01) & 0.422 (0.03) & 0.467 (0.03) \\ 
      Support vector machine & 0.375 (0.01) & 0.410 (0.02) & 0.448 (0.02) \\ 
      k-nearest neighbor & 0.387 (0.01) & 0.400 (0.02) & 0.444 (0.03) \\ 
      Naive Bayes & 0.453 (0.01) & 0.457 (0.01) & 0.480 (0.02) \\ 
      Nearest shrunken centroids & 0.320 (0.02) & 0.324 (0.02) & 0.394 (0.04) \\ 
      Penalized LDA & 0.378 (0.02) & 0.399 (0.02) & 0.465 (0.03) \\ 
      Decision trees & 0.125 (0.02) & 0.164 (0.02) & 0.409 (0.08) \\ 

      \hline
      
    \end{tabular}
  \end{adjustbox}
\end{table}




\begin{table}[p]
  \centering
  Exponentiated Gaussian data \\
  Correlation coefficient = 0.8 \\[2ex]
  \begin{adjustbox}{max size={\textwidth}{10cm}}
    \begin{tabular}{l@{\extracolsep{15mm}}rrr}
      
      \hline
      & $n=500$ & $n=250$ & $n=50$ \\ 
      \hline
      & $p = 50$ \\
      \hline

      CQC & 0.098 (0.01) & 0.120 (0.01) & 0.223 (0.02) \\ 
      CQC augmented & 0.108 (0.01) & 0.132 (0.01) & 0.232 (0.02) \\ 
      Quantile-based classifiers & 0.061 (0.01) & 0.064 (0.01) & 0.133 (0.03) \\ 
      FANS  & 0.177 (0.02) & 0.232 (0.03) & 0.388 (0.06) \\
      FANS2 & 0.184 (0.02) & 0.235 (0.02) & 0.405 (0.04) \\
      Penalized logistic regression & 0.430 (0.01) & 0.426 (0.01) & 0.430 (0.02) \\ 
      Support vector machine & 0.355 (0.02) & 0.362 (0.02) & 0.417 (0.03) \\ 
      k-nearest neighbor & 0.416 (0.02) & 0.412 (0.02) & 0.446 (0.02) \\ 
      Naive Bayes & 0.436 (0.03) & 0.445 (0.03) & 0.453 (0.02) \\ 
      Nearest shrunken centroids & 0.414 (0.02) & 0.418 (0.03) & 0.432 (0.02) \\ 
      Penalized LDA & 0.405 (0.01) & 0.412 (0.01) & 0.432 (0.04) \\ 
      Decision trees & 0.155 (0.02) & 0.201 (0.03) & 0.327 (0.02) \\ [2ex]

      \hline
      & $p = 250$ \\
      \hline

      CQC & 0.128 (0.01) & 0.164 (0.02) & 0.242 (0.03) \\ 
      CQC augmented & 0.134 (0.01) & 0.167 (0.02) & 0.305 (0.03) \\ 
      Quantile-based classifiers & 0.060 (0.01) & 0.068 (0.01) & 0.157 (0.03) \\ 
      FANS  & 0.197 (0.02) & 0.265 (0.02) & 0.457 (0.05) \\
      FANS2 & 0.197 (0.01) & 0.271 (0.02) & 0.463 (0.05) \\
      Penalized logistic regression & 0.435 (0.02) & 0.458 (0.03) & 0.489 (0.02) \\ 
      Support vector machine & 0.404 (0.02) & 0.433 (0.02) & 0.450 (0.02) \\ 
      k-nearest neighbor & 0.403 (0.02) & 0.417 (0.03) & 0.448 (0.02) \\ 
      Naive Bayes & 0.450 (0.03) & 0.461 (0.03) & 0.465 (0.01) \\ 
      Nearest shrunken centroids & 0.414 (0.02) & 0.434 (0.02) & 0.421 (0.02) \\ 
      Penalized LDA & 0.407 (0.01) & 0.438 (0.04) & 0.464 (0.03) \\ 
      Decision trees & 0.144 (0.02) & 0.208 (0.03) & 0.416 (0.05) \\ [2ex]

      \hline
      & $p = 500$ \\
      \hline

      CQC & 0.135 (0.01) & 0.160 (0.01) & 0.268 (0.02) \\ 
      CQC augmented & 0.138 (0.01) & 0.172 (0.02) & 0.344 (0.05) \\ 
      Quantile-based classifiers & 0.060 (0.01) & 0.065 (0.01) & 0.169 (0.03) \\ 
      FANS  & 0.205 (0.02) & 0.268 (0.03) & 0.453 (0.05) \\
      FANS2 & 0.206 (0.02) & 0.271 (0.03) & 0.476 (0.04) \\
      Penalized logistic regression & 0.461 (0.02) & 0.458 (0.02) & 0.490 (0.01) \\ 
      Support vector machine & 0.441 (0.02) & 0.453 (0.02) & 0.459 (0.03) \\ 
      k-nearest neighbor & 0.415 (0.02) & 0.428 (0.02) & 0.438 (0.04) \\ 
      Naive Bayes & 0.456 (0.03) & 0.458 (0.02) & 0.471 (0.02) \\ 
      Nearest shrunken centroids & 0.426 (0.02) & 0.424 (0.02) & 0.427 (0.02) \\ 
      Penalized LDA & 0.426 (0.02) & 0.436 (0.02) & 0.459 (0.03) \\ 
      Decision trees & 0.156 (0.02) & 0.196 (0.03) & 0.412 (0.07) \\ 

      \hline
      
    \end{tabular}
  \end{adjustbox}
\end{table}



\begin{table}[p]
  \centering
  Block transformed data \\
  Correlation coefficient = 0 \\[2ex]
  \begin{adjustbox}{max size={\textwidth}{10cm}}
    \begin{tabular}{l@{\extracolsep{15mm}}rrr}
      
      \hline
      & $n=500$ & $n=250$ & $n=50$ \\ 
      \hline
      & $p = 50$ \\
      \hline

      CQC & 0.029 (0.01) & 0.043 (0.01) & 0.181 (0.03) \\ 
      CQC augmented & 0.030 (0.01) & 0.048 (0.01) & 0.176 (0.05) \\ 
      Quantile-based classifiers & 0.085 (0.01) & 0.112 (0.02) & 0.223 (0.03) \\ 
      FANS  & 0.027 (0.01) & 0.068 (0.01) & 0.320 (0.08) \\
      FANS2 & 0.028 (0.01) & 0.063 (0.01) & 0.323 (0.06) \\
      Penalized logistic regression & 0.291 (0.02) & 0.322 (0.02) & 0.431 (0.03) \\ 
      Support vector machine & 0.284 (0.01) & 0.306 (0.02) & 0.386 (0.02) \\ 
      k-nearest neighbor & 0.431 (0.02) & 0.448 (0.02) & 0.464 (0.02) \\ 
      Naive Bayes & 0.374 (0.02) & 0.396 (0.02) & 0.452 (0.01) \\ 
      Nearest shrunken centroids & 0.330 (0.02) & 0.339 (0.01) & 0.422 (0.05) \\ 
      Penalized LDA & 0.283 (0.01) & 0.301 (0.02) & 0.374 (0.03) \\ 
      Decision trees & 0.040 (0.01) & 0.078 (0.01) & 0.374 (0.08) \\ [2ex]

      \hline
      & $p = 250$ \\
      \hline

      CQC & 0.045 (0.01) & 0.062 (0.01) & 0.222 (0.08) \\ 
      CQC augmented & 0.047 (0.01) & 0.063 (0.01) & 0.234 (0.08) \\ 
      Quantile-based classifiers & 0.089 (0.01) & 0.115 (0.01) & 0.234 (0.04) \\ 
      FANS  & 0.036 (0.01) & 0.083 (0.01) & 0.374 (0.07) \\
      FANS2 & 0.038 (0.01) & 0.080 (0.01) & 0.398 (0.08) \\
      Penalized logistic regression & 0.343 (0.02) & 0.382 (0.03) & 0.465 (0.03) \\ 
      Support vector machine & 0.344 (0.01) & 0.382 (0.01) & 0.424 (0.02) \\ 
      k-nearest neighbor & 0.451 (0.02) & 0.446 (0.03) & 0.470 (0.03) \\ 
      Naive Bayes & 0.408 (0.01) & 0.426 (0.02) & 0.472 (0.01) \\ 
      Nearest shrunken centroids & 0.303 (0.02) & 0.318 (0.02) & 0.403 (0.05) \\ 
      Penalized LDA & 0.340 (0.01) & 0.372 (0.02) & 0.435 (0.03) \\ 
      Decision trees & 0.038 (0.01) & 0.072 (0.02) & 0.354 (0.07) \\ [2ex]

      \hline
      & $p = 500$ \\
      \hline

      CQC & 0.046 (0.01) & 0.065 (0.01) & 0.215 (0.04) \\ 
      CQC augmented & 0.050 (0.01) & 0.070 (0.01) & 0.267 (0.07) \\ 
      Quantile-based classifiers & 0.093 (0.02) & 0.131 (0.02) & 0.244 (0.04) \\ 
      FANS  & 0.033 (0.01) & 0.102 (0.02) & 0.404 (0.07) \\
      FANS2 & 0.033 (0.01) & 0.094 (0.02) & 0.422 (0.07) \\
      Penalized logistic regression & 0.342 (0.01) & 0.414 (0.04) & 0.472 (0.03) \\ 
      Support vector machine & 0.367 (0.01) & 0.400 (0.01) & 0.451 (0.02) \\ 
      k-nearest neighbor & 0.434 (0.02) & 0.439 (0.02) & 0.464 (0.03) \\ 
      Naive Bayes & 0.420 (0.02) & 0.440 (0.02) & 0.474 (0.02) \\ 
      Nearest shrunken centroids & 0.300 (0.01) & 0.318 (0.01) & 0.406 (0.04) \\ 
      Penalized LDA & 0.358 (0.02) & 0.392 (0.02) & 0.470 (0.03) \\ 
      Decision trees & 0.043 (0.02) & 0.076 (0.03) & 0.406 (0.09) \\ 

      \hline
      
    \end{tabular}
  \end{adjustbox}
\end{table}




\begin{table}[p]
  \centering
  Block transformed data \\
  Correlation coefficient = 0.8 \\[2ex]
  \begin{adjustbox}{max size={\textwidth}{10cm}}
    \begin{tabular}{l@{\extracolsep{15mm}}rrr}
      
      \hline
      & $n=500$ & $n=250$ & $n=50$ \\ 
      \hline
      & $p = 50$ \\
      \hline

      CQC & 0.064 (0.01) & 0.081 (0.01) & 0.224 (0.04) \\ 
      CQC augmented & 0.066 (0.01) & 0.086 (0.01) & 0.234 (0.06) \\ 
      Quantile-based classifiers & 0.107 (0.01) & 0.134 (0.02) & 0.280 (0.08) \\ 
      FANS  & 0.076 (0.01) & 0.115 (0.02) & 0.302 (0.06) \\
      FANS2 & 0.076 (0.01) & 0.117 (0.01) & 0.308 (0.06) \\
      Penalized logistic regression & 0.396 (0.02) & 0.409 (0.02) & 0.443 (0.03) \\ 
      Support vector machine & 0.375 (0.03) & 0.384 (0.02) & 0.420 (0.03) \\ 
      k-nearest neighbor & 0.444 (0.02) & 0.450 (0.02) & 0.480 (0.03) \\ 
      Naive Bayes & 0.410 (0.02) & 0.420 (0.02) & 0.454 (0.04) \\ 
      Nearest shrunken centroids & 0.416 (0.03) & 0.422 (0.01) & 0.462 (0.04) \\ 
      Penalized LDA & 0.383 (0.02) & 0.388 (0.02) & 0.430 (0.05) \\ 
      Decision trees & 0.076 (0.01) & 0.115 (0.02) & 0.313 (0.07) \\ [2ex]

      \hline
      & $p = 250$ \\
      \hline

      CQC & 0.091 (0.00) & 0.115 (0.01) & 0.228 (0.03) \\ 
      CQC augmented & 0.092 (0.00) & 0.118 (0.01) & 0.293 (0.05) \\ 
      Quantile-based classifiers & 0.126 (0.03) & 0.154 (0.02) & 0.263 (0.02) \\ 
      FANS  & 0.079 (0.01) & 0.128 (0.01) & 0.428 (0.08) \\
      FANS2 & 0.085 (0.01) & 0.139 (0.01) & 0.402 (0.07) \\
      Penalized logistic regression & 0.421 (0.02) & 0.432 (0.03) & 0.477 (0.03) \\ 
      Support vector machine & 0.416 (0.02) & 0.428 (0.03) & 0.452 (0.02) \\ 
      k-nearest neighbor & 0.452 (0.03) & 0.450 (0.02) & 0.463 (0.02) \\ 
      Naive Bayes & 0.435 (0.02) & 0.436 (0.02) & 0.466 (0.02) \\ 
      Nearest shrunken centroids & 0.392 (0.02) & 0.388 (0.02) & 0.434 (0.02) \\ 
      Penalized LDA & 0.397 (0.02) & 0.403 (0.02) & 0.460 (0.03) \\ 
      Decision trees & 0.081 (0.01) & 0.118 (0.02) & 0.443 (0.07) \\ [2ex]

      \hline
      & $p = 500$ \\
      \hline

      CQC & 0.090 (0.01) & 0.125 (0.01) & 0.238 (0.04) \\ 
      CQC augmented & 0.090 (0.01) & 0.123 (0.01) & 0.312 (0.06) \\ 
      Quantile-based classifiers & 0.115 (0.03) & 0.158 (0.02) & 0.277 (0.03) \\ 
      FANS  & 0.092 (0.01) & 0.146 (0.02) & 0.429 (0.07) \\
      FANS2 & 0.095 (0.01) & 0.148 (0.02) & 0.415 (0.07) \\
      Penalized logistic regression & 0.430 (0.02) & 0.444 (0.02) & 0.480 (0.02) \\ 
      Support vector machine & 0.427 (0.02) & 0.436 (0.02) & 0.460 (0.02) \\ 
      k-nearest neighbor & 0.459 (0.02) & 0.457 (0.01) & 0.487 (0.01) \\ 
      Naive Bayes & 0.436 (0.01) & 0.450 (0.02) & 0.480 (0.03) \\ 
      Nearest shrunken centroids & 0.397 (0.02) & 0.398 (0.03) & 0.423 (0.03) \\ 
      Penalized LDA & 0.411 (0.01) & 0.423 (0.02) & 0.465 (0.04) \\ 
      Decision trees & 0.084 (0.02) & 0.146 (0.03) & 0.421 (0.10) \\ 

      \hline
      
    \end{tabular}
  \end{adjustbox}
\end{table}





\begin{table}[p]
  \centering
  Beta distributed data \\[2ex]
  \begin{adjustbox}{max size={\textwidth}{10cm}}
    \begin{tabular}{l@{\extracolsep{15mm}}rrr}
      
      \hline
      & $n=500$ & $n=250$ & $n=50$ \\ 
      \hline
      & $p = 50$ \\
      \hline

      CQC & 0.009 (0.01) & 0.016 (0.01) & 0.060 (0.04) \\ 
      CQC augmented & 0.010 (0.01) & 0.015 (0.01) & 0.068 (0.03) \\ 
      Quantile-based classifiers & 0.019 (0.01) & 0.031 (0.01) & 0.158 (0.03) \\ 
      FANS  & 0.015 (0.00) & 0.025 (0.02) & 0.053 (0.03) \\
      FANS2 & 0.016 (0.00) & 0.023 (0.01) & 0.072 (0.05) \\
      Penalized logistic regression & 0.501 (0.02) & 0.511 (0.02) & 0.420 (0.17) \\ 
      Support vector machine & 0.438 (0.02) & 0.475 (0.03) & 0.445 (0.10) \\ 
      k-nearest neighbor & 0.484 (0.01) & 0.491 (0.01) & 0.505 (0.02) \\ 
      Naive Bayes & 0.327 (0.11) & 0.332 (0.13) & 0.195 (0.16) \\ 
      Nearest shrunken centroids & 0.495 (0.02) & 0.512 (0.01) & 0.498 (0.01) \\ 
      Penalized LDA & 0.507 (0.01) & 0.503 (0.01) & 0.420 (0.17) \\ 
      Decision trees & 0.019 (0.00) & 0.029 (0.01) & 0.071 (0.03) \\ [2ex]

      \hline
      & $p = 250$ \\
      \hline

      CQC & 0.012 (0.00) & 0.011 (0.00) & 0.049 (0.02) \\ 
      CQC augmented & 0.007 (0.00) & 0.011 (0.01) & 0.063 (0.03) \\ 
      Quantile-based classifiers & 0.108 (0.01) & 0.173 (0.03) & 0.301 (0.02) \\ 
      FANS  & 0.018 (0.01) & 0.024 (0.01) & 0.059 (0.04) \\
      FANS2 & 0.017 (0.01) & 0.026 (0.01) & 0.059 (0.04) \\
      Penalized logistic regression & 0.505 (0.02) & 0.495 (0.02) & 0.464 (0.11) \\ 
      Support vector machine & 0.494 (0.02) & 0.500 (0.02) & 0.499 (0.02) \\ 
      k-nearest neighbor & 0.490 (0.02) & 0.490 (0.02) & 0.507 (0.02) \\ 
      Naive Bayes & 0.413 (0.10) & 0.331 (0.18) & 0.434 (0.06) \\ 
      Nearest shrunken centroids & 0.504 (0.01) & 0.494 (0.02) & 0.496 (0.01) \\ 
      Penalized LDA & 0.501 (0.02) & 0.500 (0.01) & 0.494 (0.01) \\ 
      Decision trees & 0.016 (0.00) & 0.022 (0.01) & 0.076 (0.06) \\ [2ex]

      \hline
      & $p = 500$ \\
      \hline

      CQC & 0.011 (0.01) & 0.019 (0.01) & 0.075 (0.03) \\ 
      CQC augmented & 0.012 (0.00) & 0.021 (0.01) & 0.068 (0.03) \\ 
      Quantile-based classifiers & 0.164 (0.02) & 0.228 (0.03) & 0.356 (0.02) \\ 
      FANS  & 0.020 (0.01) & 0.028 (0.01) & 0.073 (0.07) \\
      FANS2 & 0.022 (0.01) & 0.031 (0.01) & 0.072 (0.05) \\
      Penalized logistic regression & 0.506 (0.01) & 0.485 (0.02) & 0.460 (0.13) \\ 
      Support vector machine & 0.506 (0.02) & 0.501 (0.02) & 0.497 (0.02) \\ 
      k-nearest neighbor & 0.495 (0.02) & 0.498 (0.01) & 0.499 (0.02) \\ 
      Naive Bayes & 0.459 (0.06) & 0.441 (0.05) & 0.462 (0.03) \\ 
      Nearest shrunken centroids & 0.500 (0.01) & 0.492 (0.02) & 0.497 (0.01) \\ 
      Penalized LDA & 0.498 (0.01) & 0.498 (0.02) & 0.491 (0.02) \\ 
      Decision trees & 0.013 (0.01) & 0.030 (0.01) & 0.060 (0.03) \\ 

      \hline
      
    \end{tabular}
  \end{adjustbox}
\end{table}




\begin{table}[p]
  \centering
  Gamma distributed data \\[2ex]
  \begin{adjustbox}{max size={\textwidth}{10cm}}
    \begin{tabular}{l@{\extracolsep{15mm}}rrr}
      
      \hline
      & $n=500$ & $n=250$ & $n=50$ \\ 
      \hline
      & $p = 50$ \\
      \hline

      CQC & 0.035 (0.01) & 0.052 (0.01) & 0.128 (0.03) \\ 
      CQC augmented & 0.034 (0.01) & 0.054 (0.01) & 0.153 (0.04) \\ 
      Quantile-based classifiers & 0.045 (0.01) & 0.062 (0.02) & 0.156 (0.01) \\ 
      FANS  & 0.069 (0.01) & 0.107 (0.02) & 0.389 (0.10) \\
      FANS2 & 0.073 (0.01) & 0.116 (0.02) & 0.363 (0.09) \\
      Penalized logistic regression & 0.501 (0.02) & 0.500 (0.02) & 0.493 (0.01) \\ 
      Support vector machine & 0.424 (0.01) & 0.462 (0.02) & 0.492 (0.01) \\ 
      k-nearest neighbor & 0.486 (0.02) & 0.489 (0.02) & 0.495 (0.01) \\ 
      Naive Bayes & 0.341 (0.02) & 0.369 (0.02) & 0.424 (0.02) \\ 
      Nearest shrunken centroids & 0.503 (0.01) & 0.506 (0.02) & 0.506 (0.02) \\ 
      Penalized LDA & 0.501 (0.01) & 0.506 (0.02) & 0.498 (0.00) \\ 
      Decision trees & 0.038 (0.01) & 0.060 (0.01) & 0.272 (0.11) \\ [2ex]

      \hline
      & $p = 250$ \\
      \hline

      CQC & 0.041 (0.01) & 0.070 (0.02) & 0.228 (0.12) \\ 
      CQC augmented & 0.046 (0.01) & 0.083 (0.03) & 0.272 (0.11) \\ 
      Quantile-based classifiers & 0.055 (0.02) & 0.070 (0.02) & 0.246 (0.13) \\ 
      FANS  & 0.093 (0.03) & 0.146 (0.03) & 0.392 (0.09) \\
      FANS2 & 0.113 (0.05) & 0.148 (0.04) & 0.413 (0.07) \\
      Penalized logistic regression & 0.502 (0.01) & 0.503 (0.01) & 0.503 (0.01) \\ 
      Support vector machine & 0.496 (0.01) & 0.501 (0.01) & 0.495 (0.01) \\ 
      k-nearest neighbor & 0.481 (0.01) & 0.489 (0.02) & 0.492 (0.01) \\ 
      Naive Bayes & 0.491 (0.01) & 0.494 (0.01) & 0.501 (0.01) \\ 
      Nearest shrunken centroids & 0.492 (0.02) & 0.501 (0.01) & 0.493 (0.01) \\ 
      Penalized LDA & 0.500 (0.00) & 0.500 (0.01) & 0.498 (0.01) \\ 
      Decision trees & 0.040 (0.01) & 0.075 (0.03) & 0.296 (0.09) \\ [2ex]

      \hline
      & $p = 500$ \\
      \hline

      CQC & 0.045 (0.01) & 0.070 (0.01) & 0.211 (0.05) \\ 
      CQC augmented & 0.049 (0.01) & 0.081 (0.01) & 0.335 (0.08) \\ 
      Quantile-based classifiers & 0.062 (0.01) & 0.089 (0.02) & 0.192 (0.03) \\ 
      FANS  & 0.083 (0.01) & 0.128 (0.02) & 0.374 (0.11) \\
      FANS2 & 0.091 (0.02) & 0.162 (0.05) & 0.424 (0.09) \\
      Penalized logistic regression & 0.502 (0.01) & 0.503 (0.02) & 0.503 (0.01) \\ 
      Support vector machine & 0.506 (0.01) & 0.496 (0.01) & 0.501 (0.00) \\ 
      k-nearest neighbor & 0.496 (0.02) & 0.490 (0.02) & 0.499 (0.02) \\ 
      Naive Bayes & 0.492 (0.01) & 0.494 (0.01) & 0.498 (0.01) \\ 
      Nearest shrunken centroids & 0.504 (0.01) & 0.504 (0.02) & 0.504 (0.01) \\ 
      Penalized LDA & 0.504 (0.01) & 0.501 (0.01) & 0.496 (0.01) \\ 
      Decision trees & 0.039 (0.01) & 0.110 (0.13) & 0.341 (0.12) \\ 

      \hline
      
    \end{tabular}
  \end{adjustbox}
\end{table}




%%% Local Variables:
%%% mode: latex
%%% TeX-master: "cqc_paper"
%%% End:






%%% Local Variables:
%%% mode: latex
%%% TeX-master: "mqc_paper"
%%% End:
