

\section{Numerical results}
\label{sec:numerical-results}

In this section we consider the performance of the composite quantile-based
classifiers with that of nine other classification methods: quantile-based
classifiers \cite{hennig2016}, FANS \cite{fan2016}, penalized linear regression
($\ell_1$ penalty) \cite{park2007}, support vector machine (radial kernel)
\cite{cortes1995}, k-nearest neighbor \cite{cover1967}, naive Bayes
\cite{hastie2009}, nearest shrunken centroids \cite{tibshirani2002}, penalized
LDA \cite{witten2011}, and decision trees \cite{breiman1984}.  Detailed
descriptions of the settings and software implementations for each of these
methods is provided in the supplementary materials.


\subsection{Simulated data scenarios}
\label{sec:simulated-data-scenarios}

We consider two simulated data scenarios in the paper; additional simulated data
scenarios and results are provided in the supplementary materials.  Within each
scenario, we consider combinations of sample sizes $n = 500, 250, 50$ and data
dimension $p = 50, 250, 500$.  In every setting we consider equal numbers of
observations for each class, so that the number of samples from each class is
$n / 2$.  Additionally, although the number of features varies for different
simulations, the number of discriminatory features remains fixed at 50
throughout all of the simulations.  That is to say, that for every simulation
there are 50 features such that the distribution of the features varies across
the classes, and the remaining features (if any) are noise variables drawn from
the same distribution for each class.  The noise variables were drawn
independently of each other from a Gaussian distribution for each scenario.

% Consider the following framework for each of the scenarios.  Let
% $\vec{Z} = (Z_1, \dots, Z_p) \sim N(\vec{0},\, \vec{\Sigma})$, and let
% $\vec{V}_i = (V_{i1}, \dots, V_{ip})$ and $\vec{W}_i = (W_{i1}, \dots, W_{ip})$
% be observations from two populations, $i = 1, \dots, n / 2$, and where the
% $\vec{V}_i$ and $\vec{W}_i$ are all mutually independent.  There are two choices
% of covariance matrices considered for $\vec{\Sigma}$.  In some scenarios we
% consider independent data where the covariance matrix is the identity matrix,
% and in other scenarios we consider the correlated data with an autoregressive
% lag 1 covariance matrix (abbreviated as AR1).  The AR1 matrix is specified with
% variance parameters identically 1 on the diagonal, and correlation parameter
% 0.8.  In the interest of continuity, the scenarios are designed to be similar to
% the simulation settings presented in \cite{hennig2016}.

In the first scenario we consider different distributions for each of 5 blocks
of features.  In more detail, the data is sampled from a Gaussian distribution
with components $(Z_1, \dots, Z_{50})$, and then the features are evenly split
into 5 blocks of 10 and the following transformations performed to each block:
(i) $V_{ij} \sim Z_j$ and $W_{ij} \sim Z_j + 0.2$, (ii) $V_{ij} \sim \exp(Z_j)$
and $W_{ij} \sim \exp(Z_j) + 0.2$, (iii) $V_{ij} \sim \log |Z_j|$ and
$W_{ij} \sim \log |Z_j| + 0.1$, (iv) $V_{ij} \sim Z_j^2$ and
$W_{ij} \sim Z_j^2 + 0.2$, and (v) $V_{ij} \sim |Z_j|^{1/2}$ and
$W_{ij} \sim |Z_j|^{1/2} + 0.1$.  In one setting we consider uncorrelated data
for the underlying Gaussian distribution, and in another setting we consider
autoregressive correlation.

In the second scenario, we considered data with different distributional shapes
within each feature.  In one setting we considered data with independent beta
distributed features and in another setting we considered data with independent
gamma distributed features.  The beta distribution parameters were sampled as
follows.  Two shape parameters were sampled for each feature each from a
$\mathit{unif}(0.1, 3)$ and $\mathit{unif}(0.5, 3)$ distribution for each shape
parameter, respectively.  Then for each class and feature, each parameter was
transformed by taking the absolute value of some additive Gaussian random noise.
So for example, suppose that $\alpha_j$ and $\beta_j$ are the shape parameters
drawn for the $j$-th feature.  Then for each class the distributional parameters
are each sampled from $|\alpha_j + N(0, \sigma_j^2)|$ and
$|\beta_j + N(0, \sigma_j^2)|$ distributions.  $\sigma_1, \dots, \sigma_{50}$
were given by a fixed sequence each with values between 0.05 and 0.20.  Once the
shape parameters were sampled, the same parameters were used for every replicate
and every simulation study.  The gamma distribution parameters in the other
setting were sampled in essentially the same manner.


\subsubsection{Hybrid quantile-based classifiers}
\label{sec:hybid-cqc}

One disadvantage of the composite quantile-based classifiers approach is that
the data is partitioned into two parts: one for selecting quantile levels and
estimating quantiles, and the other for training the linear combination
coefficients, so that each of these tasks is performed using only a subset of
the data.  When data is abundant, there is typically not a great loss of
efficiency when performing these tasks as compared to using all of the available
data.  However, when data is limited, losing a portion of the data can result in
a large loss of efficiency.  As a concrete example, suppose $n = 50$ and we
split the data into two equal parts.  Then estimating the within-class quantiles
is limited to 25 points so that the quantiles for one of the classes are
estimated using no more than 12 data points - a big difference to estimating the
quantiles from 25 data points.  When data is limited as in this example, we find
that composite quantile-based classifiers can suffer from instability in the
choice of quantile levels.  As a result, we suggest using the quantile levels as
selected by the quantile-based classifiers method and then applying the linear
combination step on the transformed variables as in the usual composite
quantile-based classifiers method.  This has the result in choosing quantile
levels that are far more stable at the cost of some flexibility.  Our suggestion
is that when data is relatively small, say less than 100, to perform
cross-validation with the regular composite quantile-based classifiers and
hybrid approach to select the better model.  It is noted in the discussion of
the simulations whenever the hybrid approach is used.


% \subsubsection{Classifier implementations and settings}
% \label{sec:classifier-implementations}

% We compared the misclassification rate from the composite quantile-based
% classifiers model with that of nine other classification methods: quantile-based
% classifiers \cite{hennig2016}, FANS \cite{fan2016}, penalized linear regression
% ($\ell_1$ penalty) \cite{park2007}, support vector machine (radial kernel)
% \cite{cortes1995}, k-nearest neighbor \cite{cover1967}, naive Bayes
% \cite{hastie2009}, nearest shrunken centroids \cite{tibshirani2002}, penalized
% LDA \cite{witten2011}, and decision trees \cite{breiman1984}.  

% The composite quantiles-based classifier is implemented using the \r/
% programming language; the source code is available from the authors upon
% request.  Quantile-based methods is implemented as the \r/ package
% \texttt{quantileDA}.  There are several methods of quantifying distributional
% skew provided by the package; we used the default Galton skewness measure.  FANS
% is implemented as \matlab/ \cite{matlab} source code and is available upon
% request by the authors of \cite{fan2016}.  Results from both the FANS method and
% FANS2 method are shown since FANS2 is similar to the augmented version of
% composite quantile-based classifiers.  Penalized linear regression is
% implemented as the \r/ package \texttt{glmnet} and uses the $\ell_1$ penalty
% with 10-fold cross validation to select the penalty parameter.  Support vector
% machine is implemented in the \r/ package \texttt{e1071} through the C++ library
% \texttt{libsvm}.  The tuning parameters for each simulation were selected by
% using the function \texttt{tune.svm} over the kernel coefficient parameter from
% among $\{0.001, 0.01, 0.1, 1, 2\}$, and constraints violation cost from among
% $\{1, 2, 4, 8, 16\}$.  k-nearest neighbors is implemented in the \r/ package
% \texttt{class} and uses leave-one-out cross validation to choose the number of
% neighbors considered from among $\{1, \dots, 9\}$.  Naive Bayes is implemented
% in the \r/ package \texttt{e1071}.  Nearest shrunken centroids is implemented as
% the \r/ package \texttt{pamr} with 10-fold cross validation used to select the
% threshold parameter.  The version of penalized linear discriminant analysis
% compared in the simulation studies is that proposed in \cite{witten2011}, and is
% implemented as the \r/ package \texttt{PenalizedLDA} using 6-fold cross
% validation and 1 discriminant vector.  Decision trees is implemented as the \r/
% package \texttt{rpart}.  All packages used in the numerical analysis are
% available from the Comprehensive \r/ Archive Network.


\subsubsection{Simulation results}
\label{sec:simulation-results}

The classifer proposed in this paper is abbreviated as CQC.  Results for the
second variant of composite quantile-based classifiers where the transformed
quantile distances data is augmented by the original data are also shown and are
abbreviated as CQC augmented.

% Simulation results for the Gaussian setting are shown in Table
% \ref{tab:gauss-corr0} and Table \ref{tab:gauss-corr08}.  We would expect
% composite quantile-based classifiers to be suboptimal in this setting compared
% to LDA and similar classifiers since they do not make use of the distributional
% information, and are interested in the magnitude of the loss of efficiency.  One
% thing to notice is that the augmented version of CQC performs better nearly
% everywhere as compared to the non-augmented version.  This is to be expected
% since the Bayes rule decision boundary is linear in the original features, and
% is also the case for FANS2 as compared to FANS.  In this setting, nearest
% shrunken centroids and penalized LDA show the best performance.  When $n$ is
% large compared to $p$, marginal quantile-based classifiers perform reasonably
% well, although performance deteriorates relative to other methods when
% conditions are less ideal.

% Simulation results for the exponentiated Gaussian setting are shown in Table
% \ref{tab:exp-gauss-corr0} and Table \ref{tab:exp-gauss-corr08}.  As we saw in
% Section \ref{sec:classifier-examples}, that in this setting quantile-based
% classifiers are nearly optimal and this is borne out in the simulation study
% results.  CQC performs well also, although never as well as quantile-based
% classifiers.  We see this as being due to several reasons.  Firstly,
% quantile-based classifiers composite quantile-based classifiers sacrifice part
% of the data set in order to train the linear combination coefficients; this has
% the effect of reducing the sample size to both choose good quantile levels and
% to estimate the within-class quantiles.  Secondly, since the optimal quantile
% level is the same for every quantile, quantile-based classifiers is able to
% borrow information across all of the features to estimate the optimal quantile
% level, while composite quantile-based classifiers selects the quantile level one
% feature at-a-time.  Thirdly, the freedom to select linear combination
% coefficients for the composite quantile-based classifiers method can result in
% additional variation.  These points highlight the trade-off between the two
% methods.  When the optimal quantile levels and discriminatory information are
% the same across the components then quantile-based classifiers perform better
% than CQC, and when this is not the case then CQC may perform better in some
% settings.

Simulation results for the block transformed Gaussian setting are shown in Table
\ref{tab:block-transformed-corr0} and Table \ref{tab:block-transformed-corr08}.
In these settings composite quantile-based classifiers and FANS typically
perform better than other classifiers.  Decision trees also performs well in
some settings, which suggests that there are some relatively discriminative
features.  When the quantity of the training data decreases to $n = 50$ we see a
sharp rise in the misclassification rates for all methods.  It is in this
setting that the hybrid quantile-based classifiers do well.  For example, when
$n = 50$ and $p = 500$ with uncorrelated data, using the hybrid approach reduces
the misclassification rate from 0.351 to 0.215 as compared to composite
quantile-based classifiers.  We believe the reason that the hybrid approach
works well here is that we may have a number of relatively discriminative
features that the quantile-based methods approach of selecting quantile levels
can effectively key in on, and then the additional linear combination step can
help to deal with some of the differences in scale and discriminatory power
between the blocks.

Simulation results for the beta distributed and gamma distributed data settings
are shown in Table \ref{tab:block-transformed-corr0} and Table
\ref{tab:block-transformed-corr08}.  In the case of the beta distributed data,
composite quantile-based classifiers and decision trees performed the best.  We
see this as a sign of a few features having some degree of separability across
the two classes.  In this setting where the optimal quantile level varies across
all of the features, the additional flexibility of composite quantile-based
classifiers to select varying quantile levels results in better performance as
compared to quantile-based classifiers.  Similar results are exhibited for the
gamma distributed data, although in this case quantile-based classifiers perform
about as well as composite quantile-based classifiers.  We see this as
suggesting that there happen to be a number of features with similar optimal
quantile levels that quantile-based classifiers can select to achieve similar
performance as the more flexible approach.




\subsection{Real data study}
\label{sec:real-data-study}

For a real data study, we considered a spam email data set with a total of 4,601
observations and 57 features.  The attributes are, for example, the percentage
of specific words or phrases in the email (e.g. money, free, order), the average
and maximum run lengths of uppercase letters, and the total number of uppercase
letters.  The features for this data are typically highly skewed: often 90\% of
the data have no occurrences of a word while a few emails have a high rate.
39\% of the emails in the data set were spam emails, versus 61\% non-spam
emails.

Various settings were considered for this data as follows.  For each setting, a
certain amount of data was randomly selected to be the training data, and the
misclassification rate was then based on the remaining holdout data.  The number
of observations used to train the classifiers for the various settings was 100,
250, 500, and 1000 observations, respectively.  The misclassification rate for
more than 1,000 training observations was basically constant across the
classifiers and is not shown.

We observe that augmented quantile-based classifiers has nearly the lowest
misclassification rate across all of the settings.  Penalized logistic
regression is one of the better competitors which suggests that a linear
decision rule boundary is a reasonable choice of boundary for this problem.  The
augmented FANS classifier is another strong competitor which makes sense given
that it also uses the original features, and has a close relationship to
composite quantile-based classifiers.


% \subsection{Real data considerations}

% In practice, one often encounters data that has variables with both continuous
% as well as categorical data.  In problems such as these categorical data is
% typically reformulated into so-called dummy-coded variables.  However, the
% quantile-based classifier is inherently ill-conditioned to handle such data.
% When categorical variables are present in the data, we propose including them in
% the classifier as untransformed dummy-coded variables.

% A second type of data can also cause problems:  one where the data is a mixture
% of point masses as well as continuous data.  As an example, consider univariate
% data where the the quantiles of the underlying populations are given as follows.
% \begin{center}
%   \begin{tabular}{lrrrrrrrrrrr}
%     \toprule
%     & \multicolumn{11}{c}{\underline{Quantile levels}} \\
%     & 0.89 & 0.90 & 0.91 & 0.92 & 0.93 & 0.94 & 0.95 & 0.96 & 0.97 & 0.98 & 0.99 \\
%     \midrule
%     Class 0 & 0.00 & 0.02 & 0.14 & 0.23 & 0.33 & 0.42 & 0.55 & 0.68 & 0.87 & 1.05 & 1.41 \\
%     Class 1 & 0.00 & 0.00 & 0.00 & 0.00 & 0.00 & 0.00 & 0.00 & 0.06 & 0.08 & 0.18 & 0.20 \\
%     \bottomrule
%   \end{tabular}
% \end{center}
% This example raises the question as to what the quantile distances would be for
% a fixed choice of quantile level.  Suppose first that we consider a quantile
% level with nonzero quantiles for both classes, say 0.97.  In this case, the
% quantile distances for the data in class 0 with value 0 than the data quantile
% distances for the data in class 1 with value 0.  As a result, the quantile-based
% data will have spuriously introduced differences in the data for 90\% of the
% data, when in fact there is no discriminative data in those quantiles.
% Conversely, suppose now that we select a quantile level with quantiles for both
% classes with value 0.  Doing so has the effect of simply multiplying the nonzero
% values in the data by a constant factor.  While this is less disastrous then the
% previous scenario, we are not using the quantile-based data approach consistent
% with the rest of the paper.

% What we propose to do for such a scenario is the following.  When mixture data
% such as the example given above is present for a variable, we try to separate
% the parts of the data that belong to the point mass contribution to the mixture,
% and those that belong to the continuous part of the mixture.  Then for the point
% mass data, we create a categorical variable that includes a reference category
% for the continuous part of the data, and for the continuous part we perform the
% quantile-based classification using only the reduced subset of the data.

\begin{table}[ht]
  \caption{Misclassification rates for the spam email data set for varying
    numbers of observations used as training data.}
  \label{tab:spam}
  \centering
  \vspace{5mm}

  \begin{adjustbox}{max size={\textwidth}{10cm}}
    \begin{tabular}{l@{\extracolsep{15mm}}rrrr}
      \toprule
      & \multicolumn{4}{c}{Number of observations used to train} \\[1ex]
      \cline{2-5}
      \rule{0mm}{5mm} & $n = 100$ & $n = 250$ & $n = 500$ & $n = 1000$ \\
      \midrule
      CQC & 0.130 (0.04) & \bn{0.088 (0.01)} & 0.080 (0.01) & 0.068 (0.01) \\ 
      CQC augmented & \bn{0.123 (0.03)} & 0.090 (0.01) & \bn{0.079 (0.01)} & \bn{0.064 (0.00)} \\ 
      Quantile-based classifiers & 0.319 (0.03) & 0.307 (0.01) & 0.305 (0.01) & 0.302 (0.01) \\ 
      FANS & 0.153 (0.03) & 0.101 (0.01) & 0.095 (0.01) & 0.079 (0.00) \\ 
      FANS2 & 0.146 (0.02) & 0.102 (0.01) & 0.089 (0.01) & 0.074 (0.00) \\ 
      Penalized logistic regression & 0.140 (0.03) & 0.116 (0.01) & 0.100 (0.01) & 0.081 (0.01) \\ 
      Support vector machine & 0.249 (0.11) & 0.130 (0.05) & 0.100 (0.01) & 0.081 (0.00) \\ 
      k-nearest neighbor & 0.315 (0.01) & 0.299 (0.01) & 0.282 (0.01) & 0.251 (0.01) \\ 
      Naive Bayes & 0.391 (0.03) & 0.352 (0.02) & 0.314 (0.02) & 0.295 (0.02) \\ 
      Nearest shrunken centroids & 0.148 (0.03) & 0.152 (0.02) & 0.153 (0.02) & 0.134 (0.01) \\ 
      Penalized LDA & 0.138 (0.02) & 0.129 (0.01) & 0.121 (0.01) & 0.113 (0.01) \\ 
      Decision trees & 0.204 (0.03) & 0.156 (0.02) & 0.133 (0.02) & 0.108 (0.01) \\ 
      \bottomrule
    \end{tabular}
  \end{adjustbox}
\end{table}



\begin{table}[p]
  \caption{Simulation study: misclassification results for block transformed
    data.}
  \label{tab:block-transformed}

  \begin{subtable}{\textwidth}
    \centering
    \caption{Correlation coefficient = 0 with $p = 50$}
    \label{tab:block-transformed-corr0}
    \vspace{5mm}
    
    \begin{tabular}{l@{\extracolsep{15mm}}rrr}
      
      \hline
      & $n=50$ & $n=250$ & $n=500$ \\ 
      \hline

      CQC                           & 0.181 (0.03)      & \bn{0.043 (0.01)} & \bn{0.029 (0.01)} \\ 
      CQC augmented                 & \bn{0.176 (0.05)} & 0.048 (0.01)      & 0.030 (0.01)      \\ 
      Quantile-based classifiers    & 0.223 (0.03)      & 0.112 (0.02)      & 0.085 (0.01)      \\ 
      FANS                          & 0.320 (0.08)      & 0.068 (0.01)      & 0.027 (0.01)      \\
      FANS2                         & 0.323 (0.06)      & 0.063 (0.01)      & 0.028 (0.01)      \\
      Penalized logistic regression & 0.431 (0.03)      & 0.322 (0.02)      & 0.291 (0.02)      \\ 
      Support vector machine        & 0.386 (0.02)      & 0.306 (0.02)      & 0.284 (0.01)      \\ 
      k-nearest neighbor            & 0.464 (0.02)      & 0.448 (0.02)      & 0.431 (0.02)      \\ 
      Naive Bayes                   & 0.452 (0.01)      & 0.396 (0.02)      & 0.374 (0.02)      \\ 
      Nearest shrunken centroids    & 0.422 (0.05)      & 0.339 (0.01)      & 0.330 (0.02)      \\ 
      Penalized LDA                 & 0.374 (0.03)      & 0.301 (0.02)      & 0.283 (0.01)      \\ 
      Decision trees                & 0.374 (0.08)      & 0.078 (0.01)      & 0.040 (0.01)      \\

      \hline
      
    \end{tabular}
  \end{subtable}
  \vspace{10mm}

  \begin{subtable}{\textwidth}

    \centering
    \caption{Correlation coefficient = 0.8 with $p = 250$}
    \label{tab:block-transformed-corr08}
    \vspace{5mm}
    
    \begin{tabular}{l@{\extracolsep{15mm}}rrr}
      
      \hline
      & $n=50$ & $n=250$ & $n=500$ \\ 
      \hline

      CQC                           & \bn{0.228 (0.03)} & \bn{0.115 (0.01)} & 0.091 (0.00)      \\ 
      CQC augmented                 & 0.293 (0.05)      & 0.118 (0.01)      & 0.092 (0.00)      \\ 
      Quantile-based classifiers    & 0.263 (0.02)      & 0.154 (0.02)      & 0.126 (0.03)      \\ 
      FANS                          & 0.428 (0.08)      & 0.128 (0.01)      & \bn{0.079 (0.01)} \\
      FANS2                         & 0.402 (0.07)      & 0.139 (0.01)      & 0.085 (0.01)      \\
      Penalized logistic regression & 0.477 (0.03)      & 0.432 (0.03)      & 0.421 (0.02)      \\ 
      Support vector machine        & 0.452 (0.02)      & 0.428 (0.03)      & 0.416 (0.02)      \\ 
      k-nearest neighbor            & 0.463 (0.02)      & 0.450 (0.02)      & 0.452 (0.03)      \\ 
      Naive Bayes                   & 0.466 (0.02)      & 0.436 (0.02)      & 0.435 (0.02)      \\ 
      Nearest shrunken centroids    & 0.434 (0.02)      & 0.388 (0.02)      & 0.392 (0.02)      \\ 
      Penalized LDA                 & 0.460 (0.03)      & 0.403 (0.02)      & 0.397 (0.02)      \\ 
      Decision trees                & 0.443 (0.07)      & 0.118 (0.02)      & 0.081 (0.01)      \\

      \hline
      
    \end{tabular}
  \end{subtable}
\end{table}




% \begin{table}[ht]
%   \centering
%   \caption{Block transformed data, correlation coefficient = 0.8}
%   \label{tab:block-transformed-corr08}
%   \vspace{5mm}  

%   \begin{subcaption}{width=\textwidth}
%     \begin{tabular}{l@{\extracolsep{15mm}}rrr}
      
%       \hline
%       & $n=500$ & $n=250$ & $n=50$ \\ 
%       \hline
%       & $p = 250$ \\
%       \hline

%       CQC & 0.091 (0.00) & 0.115 (0.01) & 0.228 (0.03) \\ 
%       CQC augmented & 0.092 (0.00) & 0.118 (0.01) & 0.293 (0.05) \\ 
%       Quantile-based classifiers & 0.126 (0.03) & 0.154 (0.02) & 0.263 (0.02) \\ 
%       FANS  & 0.079 (0.01) & 0.128 (0.01) & 0.428 (0.08) \\
%       FANS2 & 0.085 (0.01) & 0.139 (0.01) & 0.402 (0.07) \\
%       Penalized logistic regression & 0.421 (0.02) & 0.432 (0.03) & 0.477 (0.03) \\ 
%       Support vector machine & 0.416 (0.02) & 0.428 (0.03) & 0.452 (0.02) \\ 
%       k-nearest neighbor & 0.452 (0.03) & 0.450 (0.02) & 0.463 (0.02) \\ 
%       Naive Bayes & 0.435 (0.02) & 0.436 (0.02) & 0.466 (0.02) \\ 
%       Nearest shrunken centroids & 0.392 (0.02) & 0.388 (0.02) & 0.434 (0.02) \\ 
%       Penalized LDA & 0.397 (0.02) & 0.403 (0.02) & 0.460 (0.03) \\ 
%       Decision trees & 0.081 (0.01) & 0.118 (0.02) & 0.443 (0.07) \\

%       \hline
      
%     \end{tabular}
%   \end{subcaption}
% \end{table}


\begin{table}[p]
  \caption{Simulation study: misclassification results for varying within-class
    distributional shapes}
  \label{tab:varying-distributional}

  \begin{subtable}{\textwidth}
    \centering
    \caption{Beta distributed features with $p = 50$}
    \label{tab:beta}
    \vspace{5mm}
    
    \begin{tabular}{l@{\extracolsep{15mm}}rrr}
      
      \hline
      & $n=50$ & $n=250$ & $n=500$ \\ 
      \hline

      CQC                           & 0.075 (0.03)      & \bn{0.019 (0.01)} & \bn{0.011 (0.01)} \\ 
      CQC augmented                 & 0.068 (0.03)      & 0.021 (0.01)      & 0.012 (0.00)      \\ 
      Quantile-based classifiers    & 0.356 (0.02)      & 0.228 (0.03)      & 0.164 (0.02)      \\ 
      FANS                          & 0.073 (0.07)      & 0.028 (0.01)      & 0.020 (0.01)      \\
      FANS2                         & 0.072 (0.05)      & 0.031 (0.01)      & 0.022 (0.01)      \\
      Penalized logistic regression & 0.460 (0.13)      & 0.485 (0.02)      & 0.506 (0.01)      \\ 
      Support vector machine        & 0.497 (0.02)      & 0.501 (0.02)      & 0.506 (0.02)      \\ 
      k-nearest neighbor            & 0.499 (0.02)      & 0.498 (0.01)      & 0.495 (0.02)      \\ 
      Naive Bayes                   & 0.462 (0.03)      & 0.441 (0.05)      & 0.459 (0.06)      \\ 
      Nearest shrunken centroids    & 0.497 (0.01)      & 0.492 (0.02)      & 0.500 (0.01)      \\ 
      Penalized LDA                 & 0.491 (0.02)      & 0.498 (0.02)      & 0.498 (0.01)      \\ 
      Decision trees                & \bn{0.060 (0.03)} & 0.030 (0.01)      & 0.013 (0.01)      \\ 

      \hline
      
    \end{tabular}
  \end{subtable}
  \vspace{10mm}

  \begin{subtable}{\textwidth}

    \centering
    \caption{Gamma distributed features with $p = 250$}
    \label{tab:gamma}
    \vspace{5mm}
    
    \begin{tabular}{l@{\extracolsep{15mm}}rrr}
      
      \hline
      & $n=50$ & $n=250$ & $n=500$ \\ 
      \hline

      CQC                           & \bn{0.128 (0.03)} & \bn{0.052 (0.01)} & 0.035 (0.01)      \\ 
      CQC augmented                 & 0.153 (0.04)      & 0.054 (0.01)      & \bn{0.034} (0.01) \\ 
      Quantile-based classifiers    & 0.156 (0.01)      & 0.062 (0.02)      & 0.045 (0.01)      \\ 
      FANS                          & 0.389 (0.10)      & 0.107 (0.02)      & 0.069 (0.01)      \\
      FANS2                         & 0.363 (0.09)      & 0.116 (0.02)      & 0.073 (0.01)      \\
      Penalized logistic regression & 0.493 (0.01)      & 0.500 (0.02)      & 0.501 (0.02)      \\ 
      Support vector machine        & 0.492 (0.01)      & 0.462 (0.02)      & 0.424 (0.01)      \\ 
      k-nearest neighbor            & 0.495 (0.01)      & 0.489 (0.02)      & 0.486 (0.02)      \\ 
      Naive Bayes                   & 0.424 (0.02)      & 0.369 (0.02)      & 0.341 (0.02)      \\ 
      Nearest shrunken centroids    & 0.506 (0.02)      & 0.506 (0.02)      & 0.503 (0.01)      \\ 
      Penalized LDA                 & 0.498 (0.00)      & 0.506 (0.02)      & 0.501 (0.01)      \\ 
      Decision trees                & 0.272 (0.11)      & 0.060 (0.01)      & 0.038 (0.01)      \\

      \hline
      
    \end{tabular}
  \end{subtable}
\end{table}




% \begin{table}[ht]
%   \centering
%   \caption{Beta distributed data}
%   \label{tab:beta}
%   \vspace{5mm}
  
%   % \begin{adjustbox}{width=\textwidth}
%     \begin{tabular}{l@{\extracolsep{15mm}}rrr}
      
%       \hline
%       & $n=500$ & $n=250$ & $n=50$ \\ 
%       \hline
%       & $p = 500$ \\
%       \hline

%       CQC & 0.011 (0.01) & 0.019 (0.01) & 0.075 (0.03) \\ 
%       CQC augmented & 0.012 (0.00) & 0.021 (0.01) & 0.068 (0.03) \\ 
%       Quantile-based classifiers & 0.164 (0.02) & 0.228 (0.03) & 0.356 (0.02) \\ 
%       FANS  & 0.020 (0.01) & 0.028 (0.01) & 0.073 (0.07) \\
%       FANS2 & 0.022 (0.01) & 0.031 (0.01) & 0.072 (0.05) \\
%       Penalized logistic regression & 0.506 (0.01) & 0.485 (0.02) & 0.460 (0.13) \\ 
%       Support vector machine & 0.506 (0.02) & 0.501 (0.02) & 0.497 (0.02) \\ 
%       k-nearest neighbor & 0.495 (0.02) & 0.498 (0.01) & 0.499 (0.02) \\ 
%       Naive Bayes & 0.459 (0.06) & 0.441 (0.05) & 0.462 (0.03) \\ 
%       Nearest shrunken centroids & 0.500 (0.01) & 0.492 (0.02) & 0.497 (0.01) \\ 
%       Penalized LDA & 0.498 (0.01) & 0.498 (0.02) & 0.491 (0.02) \\ 
%       Decision trees & 0.013 (0.01) & 0.030 (0.01) & 0.060 (0.03) \\ 

%       \hline
      
%     \end{tabular}
%   % \end{adjustbox}
% \end{table}




% \begin{table}[ht]
%   \centering
%   \caption{Gamma distributed data}
%   \label{tab:gamma}
%   \vspace{5mm}
  
%   % \begin{adjustbox}{width=\textwidth}
%     \begin{tabular}{l@{\extracolsep{15mm}}rrr}
      
%       \hline
%       & $n=500$ & $n=250$ & $n=50$ \\ 
%       \hline
%       & $p = 50$ \\
%       \hline

%       CQC & 0.035 (0.01) & 0.052 (0.01) & 0.128 (0.03) \\ 
%       CQC augmented & 0.034 (0.01) & 0.054 (0.01) & 0.153 (0.04) \\ 
%       Quantile-based classifiers & 0.045 (0.01) & 0.062 (0.02) & 0.156 (0.01) \\ 
%       FANS  & 0.069 (0.01) & 0.107 (0.02) & 0.389 (0.10) \\
%       FANS2 & 0.073 (0.01) & 0.116 (0.02) & 0.363 (0.09) \\
%       Penalized logistic regression & 0.501 (0.02) & 0.500 (0.02) & 0.493 (0.01) \\ 
%       Support vector machine & 0.424 (0.01) & 0.462 (0.02) & 0.492 (0.01) \\ 
%       k-nearest neighbor & 0.486 (0.02) & 0.489 (0.02) & 0.495 (0.01) \\ 
%       Naive Bayes & 0.341 (0.02) & 0.369 (0.02) & 0.424 (0.02) \\ 
%       Nearest shrunken centroids & 0.503 (0.01) & 0.506 (0.02) & 0.506 (0.02) \\ 
%       Penalized LDA & 0.501 (0.01) & 0.506 (0.02) & 0.498 (0.00) \\ 
%       Decision trees & 0.038 (0.01) & 0.060 (0.01) & 0.272 (0.11) \\ [2ex]
      
%       \hline
      
%     \end{tabular}
%   % \end{adjustbox}
% \end{table}




%%% Local Variables:
%%% mode: latex
%%% TeX-master: "cqc_paper"
%%% End:






%%% Local Variables:
%%% mode: latex
%%% TeX-master: "cqc_paper"
%%% End:
