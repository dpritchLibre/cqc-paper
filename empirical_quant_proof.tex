
We can write the minimization problem
\begin{equation*}
  \min_q \left\{
    \theta \sum_{ x_{i} > q } |x_{i} - q| ~+~
    (1 - \theta) \sum_{ x_{i} \leq q } |x_{i} - q|
  \right\}
\end{equation*}
equivalently as
% \begin{align*}
%   &\min_{q^{+}, q^{-}, \vec{u}, \vec{v}} \left\{
%     \theta \sum_{i=1}^m u_i ~+~
%     (1 - \theta) \sum_{i=1}^m v_i
%   \right\} \\[2ex]
%   & \text{subject to} \hspace{3mm}
%   x_i - (q^{+} - q^{-}) = u_i - v_i, \hspace{5mm} i = 1, \dots, m \\[2ex]
%   & q^{+} \geq 0,~ q^{-} \geq 0,~ \vec{u} \geq \vec{0},~ \vec{v} \geq 0
% \end{align*}
\begin{equation*}
  \arraycolsep=5mm
  \begin{array}{ll}
    \displaystyle
    \minimize_{q^{+}, q^{-}, \vec{u}, \vec{v}}
    & \theta \sum_{i=1}^m u_i ~+~
      (1 - \theta) \sum_{i=1}^m v_i \\[2ex]
    \text{subject to}
    & x_i - (q^{+} - q^{-}) = u_i - v_i, \hspace{5mm} i = 1, \dots, m \\[2ex]
    & q^{+} \geq 0,~ q^{-} \geq 0,~ \vec{u} \geq \vec{0},~ \vec{v} \geq \vec{0} \\
  \end{array}
\end{equation*}
which is seen to be a linear programming problem in standard form.  By rewriting
the equality condition as $q^{+} - q^{-} + u_i - v_i = x_i$ for $i=1, \dots, m$,
we can express the equality condition in matrix form as
\begin{equation*}
  \begin{bmatrix}
    1      & -1     & 1 &        &   & -1 &        &    \\
    \vdots & \vdots &   & \ddots &   &    & \ddots &    \\
    1      & -1     &   &        & 1 &    &        & -1 \\
  \end{bmatrix}
  % \begin{bmatrix}
  %   q^{+} \\ q^{-} \\ u_{1} \\ \vdots \\ u_m \\ v_1 \\ \vdots \\ v_m \\
  % \end{bmatrix}
  \begin{bmatrix}
    q^{+} \\ q^{-} \\ \vec{u} \\ \vec{v} \\
  \end{bmatrix}
  =
  \begin{bmatrix}
    x_1 \\ \vdots \\ x_m \\
  \end{bmatrix}.
\end{equation*}
% \begin{equation*}
%   \begin{bmatrix}
%     1      & -1     & 1         &        &  \bigDown{0} & -1        &        & \bigDown{0} \\
%     \vdots & \vdots &           & \ddots &              &           & \ddots &             \\
%     1      & -1     & \bigUp{0} &        & 1            & \bigUp{0} &        & -1          \\
%   \end{bmatrix}
% \end{equation*}
% \begin{equation*}
%   \begin{bmatrix}
%     1      & -1     & &         &  & &          & \\
%     \vdots & \vdots & & \vec{I} &  & & -\vec{I} & \\
%     1      & -1     & &         &  & &          & \\
%   \end{bmatrix}
% \end{equation*}
% \begin{equation*}
%   \begin{bmatrix}
%     1      & -1     & 1      & 0      & \dots  & 0       & -1     &        &    \\
%     \vdots & \vdots & 0      & \ddots & \ddots & \vdots  &  0     & \ddots & \\
%     \vdots & \vdots & \vdots & \ddots & \ddots & 0       & \vdots & \ddots & \ddots &    \\
%     1      & -1     & 0      & \dots  & 0      & 1       &  0     & \dots  &  0     & -1 \\
%   \end{bmatrix}
% \end{equation*}
Recall that a solution for $(q^{+}, q^{-}, \vec{u}, \vec{v})$ is a basic
solution if and only if there exist indices $B(1), \dots, B(m)$ such that both
(i) the columns of the coefficient matrix with column indices subset by
$B(1), \dots, B(m)$ are linearly independent, and (ii) if an element of
$(q^{+}, q^{-}, \vec{u}, \vec{v})$ does not correspond to one of
$B(1), \dots, B(m)$ then the element must have a value of 0.

We can see that in order to have independent columns from the coefficient
matrix, then for each $i$, no more than one of the columns corresponding to
either $u_i$ or $v_i$ can have an index in $B(1), \dots, B(m)$, and additionally
no more than one of the columns corresponding to either $q^{+}$ or $q^{-}$ can
have an index in $B(1), \dots, B(m)$.  Furthermore, we note that if we have one
column corresponding to $q^{+}$ or $q^{-}$, and one column corresponding to
either $u_i$ or $v_i$ for each $i$, then we have $m + 1$ columns, which is still
one column too many.  Thus we see that there must be either (i) exactly one
column corresponding to either $u_i$ or $v_i$ for each $i$ and no columns
corresponding to either $q^{+}$ or $q^{-}$, or (ii) exactly one column
corresponding to either $u_i$ or $v_i$ for each $i$ less one and exactly one
column corresponding to either $q^{+}$ or $q^{-}$.

% Next we notice that if we include a column corresponding to $q^{+}$ or $q^{-}$,
% then the solution for the corresponding coefficient is either the value of $x_i$
% or $-x_i$, and where the index $i$ corresponds to the only $i$ without a column
% corresponding to either $u_i$ or $v_i$.  Then each

Furthermore, we can infer that if a column corresponding to $q^{+}$ or $q^{-}$
has an index in $B(1), \dots, B(m)$ and the solution is feasible (i.e. $q^{+}$
and $q^{-}$ are both nonnegative), then $q^{+} = (x_i)_{+}$ and
$q^{-} = (-x_i)_{+}$, where the index $i$ corresponds to the only $i$ without a
column corresponding to either $u_i$ or $v_i$, and $(z)_{+} = \max(0, z)$.  Let
$q = q^{+} - q^{-}$, then it follows that a feasible solution for any $j \ne i$
has $u_i = (x_i - q)_{+}$ and $v_i = (q - x_i)_{+}$.  This leads to a set of
basic feasible solutions given by $q \in \{x_1, \dots, x_m\}$.  One last basic
feasible solution is given for $q = 0$ with $u_i = (x_i)_{+}$ and
$v_i = (-x_i)_{+}$ for all $i$.

Next we aim to find the minimizing basic feasible solution.  Suppose that
$\lceil \theta m \rceil = k$ and that $\ell < k$, and let $q = x_k$ and
$q^{\prime} = x_{\ell}$.  Then comparing the objective function evaluated at $q$
and $q^{\prime}$, we have
\begin{align*} 
  \left\{ \theta \sum_{i=\ell+1}^m \right.
  & (x_i - q^{\prime}) +
    \left. (1 - \theta) \sum_{i=1}^{\ell} (q^{\prime} - x_i) \right\} -
    \left\{ \theta \sum_{i=k+1}^m (x_i - q) +
    (1 - \theta) \sum_{i=1}^k (q - x_i) \right\} \\[1ex]
  &= (1 - \theta) \sum_{i=1}^{\ell} \Big\{ (q^{\prime} - x_i) - (q - x_i) \Big\} \\
  &\hspace{8mm}
    + \theta \sum_{i=\ell+1}^k (x_i - q^{\prime})
    - (1 - \theta) \sum_{i=\ell+1}^k (q - x_i) \\
  &\hspace{8mm}
    + \theta \sum_{i=k+1}^m \Big\{ (x_i - q^{\prime})
    - (x_i - q) \Big\} \\[1ex]
  % &= (1 - \theta) \sum_{i=1}^{\ell} \Big\{ (q^{\prime} - x_i) - (q - x_i) \Big\} \\
  % &\hspace{8mm}
  %   + \theta \sum_{i=\ell+1}^k (x_i - q^{\prime})
  %   - (1 - \theta) \sum_{i=\ell+1}^k (q - x_i) \\
  % &\hspace{8mm}
  %   + (1 - \theta) \sum_{i=\ell+1}^k (q^{\prime} - x_i)
  %   - (1 - \theta) \sum_{i=\ell+1}^k (q^{\prime} - x_i) \\
  % &\hspace{8mm}
  %   + \theta \sum_{i=k+1}^m \Big\{ (x_i - q^{\prime})
  %   - (x_i - q) \Big\} \\[1ex]
  &= (1 - \theta) \sum_{i=1}^{\ell} \Big\{ (q^{\prime} - x_i) - (q - x_i) \Big\} \\
  &\hspace{8mm}
    + \theta \sum_{i=\ell+1}^k (x_i - q^{\prime})
    - (1 - \theta) \sum_{i=\ell+1}^k (q - x_i)
    \pm (1 - \theta) \sum_{i=\ell+1}^k (q^{\prime} - x_i) \\
  &\hspace{8mm}
    + \theta \sum_{i=k+1}^m \Big\{ (x_i - q^{\prime})
    - (x_i - q) \Big\} \\[1ex]
  &= -(1 - \theta) \sum_{i=1}^{\ell}\, (q - q^{\prime}) \\
  &\hspace{8mm}
    + \theta \sum_{i=\ell+1}^k (x_i - q^{\prime})
    - (1 - \theta) \sum_{i=\ell+1}^k (q - q^{\prime})
    - (1 - \theta) \sum_{i=\ell+1}^k (q^{\prime} - x_i) \\
  &\hspace{8mm}
    + \theta \sum_{i=k+1}^m (q - q^{\prime}) \\[1ex]
  &= -(1 - \theta) \sum_{i=1}^k\, (q - q^{\prime}) \\
  &\hspace{8mm}
    + \theta \sum_{i=\ell+1}^k (x_i - q^{\prime})
    - (1 - \theta) \sum_{i=\ell+1}^k (q^{\prime} - x_i) \\
  &\hspace{8mm}
    + \theta \sum_{i=k+1}^m (q - q^{\prime}) \\[1ex]
  &= -(1 - \theta) \sum_{i=1}^k\, (q - q^{\prime}) \\
  &\hspace{8mm}
    + \sum_{i=\ell+1}^k (x_i - q^{\prime}) \\
  &\hspace{8mm}
    + \theta \sum_{i=k+1}^m (q - q^{\prime}) \\[1ex]
  & = - (1 - \theta)\, k\, (q - q^{\prime})
    + \sum_{i=\ell+1}^k (x_i - q^{\prime})
    + \theta (m - k) (q - q^{\prime}) \\
  &= \sum_{i=\ell+1}^k (x_i - q^{\prime}) - (k - \theta m) (q - q^{\prime}) \\
  &\geq (x_k - q^{\prime}) - (q - q^{\prime}) \\
  &= x_k - q \\
  &= 0
\end{align*}
The inequality is due to the fact that $x_i \geq q^{\prime}$ for
$i \geq \ell + 1$, and also the fact that $k - \theta m < 1$.  Suppose now that
$\lceil \theta m \rceil = k$ and that $\ell > k$, and let $q = x_k$ and
$q^{\prime} = x_{\ell}$.  Then comparing the objective function evaluated at $q$
and $q^{\prime}$, we have
\begin{align*} 
  \left\{ \theta \sum_{i=\ell+1}^m \right.
  & (x_i - q^{\prime}) +
    \left. (1 - \theta) \sum_{i=1}^{\ell} (q^{\prime} - x_i) \right\} -
    \left\{ \theta \sum_{i=k+1}^m (x_i - q) +
    (1 - \theta) \sum_{i=1}^k (q - x_i) \right\} \\[1ex]
  &= (1 - \theta) \sum_{i=1}^k \Big\{ (q^{\prime} - x_i) - (q - x_i) \Big\} \\
  &\hspace{8mm}
    + \theta \sum_{i=k+1}^{\ell} (x_i - q^{\prime})
    - (1 - \theta) \sum_{i=k+1}^{\ell} (q - x_i) \\
  &\hspace{8mm}
    + \theta \sum_{i=\ell+1}^m \Big\{ (x_i - q^{\prime})
    - (x_i - q) \Big\} \\[1ex]
  &= (1 - \theta) \sum_{i=1}^k\, (q^{\prime} - q) \\
  &\hspace{8mm}
    - \theta \sum_{i=k+1}^{\ell} (q^{\prime} - q)
    + \sum_{i=k+1}^{\ell} (x_i - q) \\
  &\hspace{8mm}
    - \theta \sum_{i=l+1}^m (q^{\prime} - q) \\[1ex]
  % &= (1 - \theta) \sum_{i=1}^k\, (q^{\prime} - q) \\
  % &\hspace{8mm}
  %   + \sum_{i=k+1}^{\ell} (x_i - q) \\
  % &\hspace{8mm}
  %   - \theta \sum_{i=k+1}^m (q^{\prime} - q) \\[1ex]
  &= (1 - \theta) \sum_{i=1}^k\, (q^{\prime} - q)
    + \sum_{i=k+1}^{\ell} (x_i - q)
    - \theta \sum_{i=k+1}^m (q^{\prime} - q) \\
  &= (1 - \theta)\, k\, (q^{\prime} - q)
    + \sum_{i=k+1}^{\ell} (x_i - q)
    - \theta\, (m - k) (q^{\prime} - q) \\
  &= (k - \theta m) (q^{\prime} - q) + \sum_{i=k+1}^{\ell} (x_i - q) \\
  &\geq 0
\end{align*}
There is one basic feasible solution remaining to check, that where $q = 0$.  To
show that this is not the optimal solution except in the case that
$x_{\lceil \theta m \rceil} = 0$, we make the following argument.  Note that if
we add some nonzero value $\tau$ to each $x_i$ and if $q^{*}$ is an optimal
choice for the original problem, then $q^{*} + \tau$ is an optimal choice for
the new problem.  Choose $-\tau$ to be one of the $x_i$ for some $i$: then 0 is
one of the values in the transformed data, which was shown in the previous
results to be no better than the $\lceil \theta m \rceil$-th largest value of
the transformed data, so it follows that the $\lceil \theta m \rceil$-th largest
value is optimal.  Since $\tau$ cannot be better than optimal for the transormed
data, then by the law of the contrapositive 0 cannot be better than optimal for
the original data.




%%% Local Variables:
%%% mode: latex
%%% TeX-master: "cqc_paper"
%%% End:
